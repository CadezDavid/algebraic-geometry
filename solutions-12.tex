\newcommand{\sheet}{12}
\documentclass{article}
\usepackage[english, german]{babel}
\usepackage{amsthm,amssymb,amsmath,mathrsfs,mathtools}
\usepackage[shortlabels]{enumitem}
\usepackage{hyperref}
\usepackage{biblatex}
\usepackage{tikz}
\usepackage{tikz-cd}

% \usepackage[tmargin=1.25in,bmargin=1.25in,lmargin=1.2in,rmargin=1.2in]{geometry}


\newcommand{\C}{\mathbb{C}}
\newcommand{\R}{\mathbb{R}}
\newcommand{\N}{\mathbb{N}}
\newcommand{\Q}{\mathbb{Q}}
\newcommand{\Z}{\mathbb{Z}}
\newcommand{\Proj}{\mathbb{P}}
\newcommand{\Aff}{\mathbb{A}}

\DeclareMathOperator{\id}{id}
\DeclareMathOperator{\im}{im}
\DeclareMathOperator{\GL}{GL}
\DeclareMathOperator{\sgn}{sgn}
\DeclareMathOperator{\Tor}{Tor}
\DeclareMathOperator{\Sym}{Sym}
\DeclareMathOperator{\coker}{coker}
\DeclareMathOperator{\Quot}{Quot}
\DeclareMathOperator{\supp}{supp}
\DeclareMathOperator{\Hom}{Hom}
\DeclareMathOperator{\Spec}{Spec}
\DeclareMathOperator{\MinSpec}{MinSpec}
\DeclareMathOperator{\MaxSpec}{MaxSpec}
\DeclareMathOperator{\diag}{diag}
\DeclareMathOperator{\BL}{BL}
\DeclareMathOperator{\Ouv}{Ouv}
\DeclareMathOperator{\Sh}{Sh}
\DeclareMathOperator{\PSh}{PSh}
\DeclareMathOperator{\Eq}{Eq}
\DeclareMathOperator{\colim}{colim}
\DeclareMathOperator{\Pic}{Pic}
\DeclareMathOperator{\CL}{CL}
\DeclareMathOperator{\eq}{eq}
\DeclareMathOperator{\codim}{codim}

\newenvironment{exercise}[1] {
  \vspace{0.5cm}
  \noindent \textbf{Exercise~{#1}.}
} {
  \vspace{0.5cm}
}
\newenvironment{claim} {
  \par\noindent\textbf{Claim.}
} { }

\newenvironment{proof_claim} {
  \par\noindent\textbf{Proof of claim.}
} {
    \qed (of claim)
}

\title{Algebraic geometry 1\\Exercise sheet \sheet}
\author{Solutions by: Eric Rudolph and David Čadež}

\date{\today}


\begin{document}

\maketitle{}

\begin{exercise}{2}
    Let $p=(p_1;\dots,p_n)\in X$ and $a_p\subset k[x_1,\dots,x_n]=:A$ its corresponding maximal ideal. Define a linear map
    \begin{align*}
        \phi := A &\to k^n\\
        f&\mapsto ({df}/{dx_1}(p),\dots, {df}/{dx_n}(p)).
    \end{align*}
    We observe that $\phi(x_i-a_i)$ is a basis of $k^n$ and $\phi(f)=0$ if and only if $f\in a_p^2$. Hence,
    \begin{align*}
        \phi^\prime:a_p/a_p^2 \to k^n
    \end{align*}
    is an isomorphism. Define $b=(f_1,\dots, f_n).$ Then the rank of the Jacoby matrix corresponding to $b$
    is exactly the $\dim(\phi(b))$ as a subset of $k^n$. Using the above isomorphism, we see that this is the dimension
    of the subspace $b+a_p^2$ of $a_p+a_p^2$.

    On the other hand, we have 
    \begin{align*}
        m/m^2\cong a_p/(b+a_p^2),
    \end{align*}
    where $m\subset$ is the maximal ideal of $\mathcal{O}_{X,p}.$ 

    Now we can just sum up the dimensions of the vector spaces to get
    \begin{align*}
        \dim(m/m^2)+\rank(J)=n.
    \end{align*}

    Now we can conclude, because $p$ is regular if and only if $\dim(X)=\dim(\mathcal{O}_{X,p})=\dim(m/m^2)$ (I think for the first 
    equality we need that $X$ is irreducible, so that use for example Prop. 5.30 in Görz-Wedhorn),
     or equivalently
    \begin{align*}
        \rank(J)=n-\dim(X).
    \end{align*}
\end{exercise}

\begin{exercise}{3}
    \begin{enumerate}
        \item We want to argue that all local rings $\mathcal{O}_{X,x}$ are regular by showing that their maximal is (non-zero
        and) principal.

        This basically follows immediately from writing down explicitely how the primes in the local rings look like and of course
        we use that their Krull dimension is one. This shows smoothness, because we have for all the local rings $A$ and their
         maximal ideals we have
        \begin{align*}
            \dim_k(m/m^2)=1=\dim(A)
        \end{align*}

        Smoothness can also be seen using the second part of the previous exercise. The Jacobian is of course of rank one.


        Since $\dim(Y)=1$, a theorem from class implies that $Y=\Spec(A)$ is normal and hence $A$ is normal.

        Since $X$ is smooth, all local rings are UFDs. So in this exercise we will show that locally factorial is not affine-local, i.e.
        that $\mathcal{O}_{X,x}$ UFD does not imply $\mathcal{O}_X(\Spec(A))$ for an affine $\Spec(A)$.
        \item We can write every $a\in A$ as 
        \begin{align*}
            a=c_o+c_1a_1(x)+c_2ya_2(x),
        \end{align*}
        so as a polynomial of degree 1 over $R$. Since the sum of two integral elements is integral again (\href{https://stacks.math.columbia.edu/tag/00GO}{StacksProject})
        it is enough to show that $c_2ya(x)$ is integral over $R$. This is clearly true since $(c_2ya(x))^2\in R$.
        \item We need to check that zero gets mapped to zero. This follows from
        \begin{align*}
            (-y)^2-x^3+x=y^2-x^3+x
        \end{align*}
        \item Write an element $a(x,y)\in A$ as 
        \begin{align*}
            a(x,y)=c+a_1(x)+ya_2(x).
        \end{align*}
        We compute 
        \begin{align*}
            N(a)=a(x,y)a(x,-y)=(c+a_1(x))^2-(ya_2(x))^2=(c+a_1(x))^2-(x^3-x)a_2(x)^2\in R.
        \end{align*}
        Multiplicativity of $N$ follows from the properties of ring maps.
        \item Take $a\in A^*$. Then $N(a)\in k[x]^*=k^*$ and since $\deg(a\sigma(a))\leq 1$ implies $\deg(a)\leq 1$ this implies $a\in k^*$.

        We can check irreducibility using the map $N.$

        Assume that $x=fg$ for $f,g\in A$. Then
        \begin{align*}
            x^2=N(x)=N(fg).
        \end{align*}
        Write $fg=a_1(x)$. Then by the calculations from above,
        \begin{align*}
            a_1(x)^2=x^2 \in R.
        \end{align*}
        Therefore $a_1(x)=\pm x,$ but $x$ is irreducible in $R$, so w.l.o.g. $f\in A^*=k^*$. 

        I guess a similar argument works to show that $y$ is irreducible.

        This shows that $y^2=x(x^2-1)$ are two decompositions into irreducible elements, showing that $A$ is not 
        factorial (to me it is not clear that $x^2-1$ is irreducible, but we could just compose it further 
        into it's irreducible factors and on the left side there is no $x$, but on the right side there is).

        First, $A$ cannot be isomorphic to $\Aff^1_k$, because $A$ is not a UFD, but $k[x]$ is.

        Let us now try to show that $A$ can also not be isomorphic to a nontrivial open subset of $\Aff^1_k$.
        Remember that all opens of $\Aff^1_k$ are principal, because $k[x]$ is principal. So it suffices to
        show that $A$ is not isomorphic to a principal open $D(f)\subset\Aff^1_k$. We do this by showing that
        their corresponding rings are not isomorphic. 
        
        This can be seen by comparing units, namely by observing that $(k[x][f^{-1}])^*\supsetneq k^*=A^*$.
    \end{enumerate}
\end{exercise}

\end{document}
