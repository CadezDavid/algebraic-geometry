\newcommand{\sheet}{6}
\documentclass{article}
\usepackage[english, german]{babel}
\usepackage{amsthm,amssymb,amsmath,mathrsfs,mathtools}
\usepackage[shortlabels]{enumitem}
\usepackage{tikz}
\usepackage{tikz-cd}

% \usepackage[tmargin=1.25in,bmargin=1.25in,lmargin=1.2in,rmargin=1.2in]{geometry}


\newcommand{\C}{\mathbb{C}}
\newcommand{\R}{\mathbb{R}}
\newcommand{\N}{\mathbb{N}}
\newcommand{\Q}{\mathbb{Q}}
\newcommand{\Z}{\mathbb{Z}}

\DeclareMathOperator{\id}{id}
\DeclareMathOperator{\im}{im}
\DeclareMathOperator{\GL}{GL}
\DeclareMathOperator{\sgn}{sgn}
\DeclareMathOperator{\Tor}{Tor}
\DeclareMathOperator{\Sym}{Sym}
\DeclareMathOperator{\coker}{coker}
\DeclareMathOperator{\Quot}{Quot}
\DeclareMathOperator{\supp}{supp}
\DeclareMathOperator{\Hom}{Hom}
\DeclareMathOperator{\Spec}{Spec}
\DeclareMathOperator{\MinSpec}{MinSpec}
\DeclareMathOperator{\diag}{diag}


\newenvironment{exercise}[1] {
  \vspace{0.5cm}
  \noindent \textbf{Exercise~{#1}.}
} {
  \vspace{0.5cm}
}
\newenvironment{claim} {
  \noindent \textbf{Claim.}
} {
}

\title{Algebraic geometry 1\\Exercise sheet \sheet}
\author{Solutions by: Eric Rudolph and David Čadež}

\date{\today}


\begin{document}

\maketitle{}

\begin{exercise}{1}
    \begin{enumerate}
        \item By the universal property of the fiber product of locally ringed spaces,
        we have the following commutative diagram
        \begin{center}
            \begin{tikzcd}
                U_i \times_{S_{i,j}} V_j \arrow{r}{\pi_2}\arrow{d}{\pi_1} & V_j\arrow{d}{\psi}\\
                U_i \arrow{r}{\phi} & S_{i,j}\subset X \times_S Y
            \end{tikzcd}
        \end{center}

    Now 
    \begin{align*}
        (\phi \circ \pi_1)^{-1}(S_{i,j})=U_i \times_{S_{i,j}} V_j  =: Z_{i,j}
    \end{align*}
    is open as the preimage of an open set under a continuous map. By the second
    part of exercise sheet 5, this induces a subset
    $(Z_{i,j},\mathcal{O}_{Z_{i,j}})$ of $X \times_S Y$ as locally
    ringed spaces.
    \end{enumerate}
\end{exercise}

\begin{exercise}{3}
    By definition we have to compute a fibred product of $\Spec(B) \to \Spec(A)$
    and $\Spec(k(p)) \to \Spec(A)$ (where $k(p)$ is the residue field of $p \in
    \Spec(A)$ and $\to$ is the canonical inclusion). Since we are dealing with
    affine schemes, we can express it concretely as $\Spec(B \otimes_A k(p))$.
    Note that $B$ has the structure of an $A$-algebra, which is induced
    by the starting morphism of schemes $\Spec(B) \rightarrow \Spec(A)$.
    So this exercise reduces to computing these tensor products.
    
    We also observe that $k[T]$ is a PID, which means every non-zero prime ideal
    is a maximal ideal. This will be handy when computing residue fields,
    because after quotienting with a non-zero ideal we already get a field (we do
    not have to further take the quotient field).

    \begin{enumerate}[a)]
        \item{In the first example we do now even have to calculate the tensor
            product, because we can rewrite $k[T, U]/(TU - 1) = k[T, T^{-1}]$,
            so this is just a localization of $k[T]$. Morphism of spectrums,
            induced by inclusion into localization, is an open immersion, so
            fibers will be singletons if $x \in D(T)$ and empty sets otherwise.
            And the structure sheaf is also clear, it is just the restriction of
            structure sheaf $\mathcal{O}_\Spec(k[T])$.}
        \item{}
        \item{}
        \item{}
    \end{enumerate}
\end{exercise}

\begin{exercise}{4}
    Take $U = D(f)$ for some $f \in A$ and let $U = \cup_i D(f_i)$ be some
    cover. We have to check that 
    \begin{equation*}
        M[f^{-1}] \rightarrow \Eq\left[ \prod_i M[f^{-1}_i] \rightrightarrows
        \prod_{i, j} M[(f_i f_j)^{-1}]  \right]
    \end{equation*}
    is isomorphism.

    This proof is exactly the same as when we proved that
    $\mathcal{O}_{\Spec(A)}$ is a sheaf, after we defined it the basis of
    principal opens.

    Then proved that $A = \Eq\left[ \prod_i A[f^{-1}_i] \rightrightarrows
    \prod_{i, j} A[(f_i f_j)^{-1}]  \right]$ where $\Spec(A) = \cup_i
    D(f_i)$ is a cover.

    We can simply tensor the whole diagram and, since tensor product commute
    with direct limits, we have that
    \begin{equation*}
        M = \Eq\left[ \prod_i M \otimes_A A[f^{-1}_i] \rightrightarrows \prod_{i, j}
        M \otimes_A A[(f_i f_j)^{-1}]  \right].
    \end{equation*}
\end{exercise}

\end{document}
