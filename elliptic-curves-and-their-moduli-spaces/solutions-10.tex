\newcommand{\sheet}{10}
\documentclass{article}
\usepackage[english, german]{babel}
\usepackage{amsthm,amssymb,amsmath,mathrsfs,mathtools}
\usepackage[shortlabels]{enumitem}
\usepackage{tikz}
\usepackage{tikz-cd}

% \usepackage[tmargin=1.25in,bmargin=1.25in,lmargin=1.2in,rmargin=1.2in]{geometry}


\newcommand{\C}{\mathbb{C}}
\newcommand{\R}{\mathbb{R}}
\newcommand{\N}{\mathbb{N}}
\newcommand{\Q}{\mathbb{Q}}
\newcommand{\Z}{\mathbb{Z}}

\DeclareMathOperator{\id}{id}
\DeclareMathOperator{\im}{im}
\DeclareMathOperator{\GL}{GL}
\DeclareMathOperator{\sgn}{sgn}
\DeclareMathOperator{\Tor}{Tor}
\DeclareMathOperator{\Sym}{Sym}
\DeclareMathOperator{\coker}{coker}
\DeclareMathOperator{\Quot}{Quot}
\DeclareMathOperator{\supp}{supp}
\DeclareMathOperator{\Hom}{Hom}
\DeclareMathOperator{\Spec}{Spec}
\DeclareMathOperator{\MinSpec}{MinSpec}
\DeclareMathOperator{\diag}{diag}


\newenvironment{exercise}[1] {
  \vspace{0.5cm}
  \noindent \textbf{Exercise~{#1}.}
} {
  \vspace{0.5cm}
}
\newenvironment{claim} {
  \noindent \textbf{Claim.}
} {
}

\title{Algebraic geometry 1\\Exercise sheet \sheet}
\author{Solutions by: Eric Rudolph and David Čadež}

\date{\today}


\begin{document}

\maketitle{}

\begin{exercise}{1}
    \begin{enumerate}
        \item{
            }
        \item{
                If $\phi$ is divisible in $L$, then it is clearly divisible in
                $\Hom(T_l(E_1), T_l(E_2))$.
                Suppose $\phi$ is divisible by $l$ in $\Hom(T_l(E_1),
                T_l(E_2))$. Then $\phi$ vanishes on $E_1[l]$, so by
                proposition 10.1 from the lectures we have that $\phi$ is
                already divisible by $l$.
            }
        \item{
                If $\End^0(E)$ is a quaternion algebra, then $\End(E)$ is of
                rank $4$ (its always free of rank $\leq 4$).
                The map $\Z_l \otimes_\Z \End(E) \cong (\Z_l)^4 \to \End(T_l(E))
                \cong M_2(\Z_l)$ (we used $T_l(E) \cong \Z_l$) is injective.
                By previous part invertible elements in $\Z_l \otimes_\Z
                \End(E)$ get mapped to invertible elements in $\End(T_l(E))$, so
                the map is also surjective.
            }
    \end{enumerate}
\end{exercise}

\begin{exercise}{2}
    By assumption both $E_1$ and $E_2$ have some nontrivial endomorphism, lets
    denote them by $\tau_1$ and $\tau_2$. By assumption there is an isomorphism
    $\phi \colon \Q(\tau_1) \to \Q(\tau_2)$. It follows that $\phi(\tau_1) = a
    \tau_2 + b$. Since $K$ is an imaginary-quadratic field, $\tau_1$ and
    $\tau_2$ satisfy equations $\tau^2_1 + d_1 = 0$ and $\tau^2_2 + d_2 = 0$ for
    some integers $d_1, d_2 > 0$. 
    \begin{equation*}
    0 = \phi(\tau_1)^2 + d_1 = a^2 \tau^2_2 + 2 a b \tau_2 + b^2 + d_1 =
    - a^2 d_2 + 2 a b \tau_2 + b^2 + d_1
    \end{equation*}
    Therefore $a b = 0$.

    If $a = 0$, then $b^2 + d_1 = 0$ which doesn't have a solution for $b \in
    \Q$. So $b = 0$.
    Then $a^2 d_2 = d_1$. Write $a = \frac{a_1}{a_2} \in \Q$. Then
    $a^2_1 d_2 = a^2_2 d_1$ and $a_2 \tau_1 = \pm a_1 \tau_2$. So there exists a
    nonconstant morphism from $E_1$ to $E_2$, namely multiplication with $a_2
    \in \Z$.

\end{exercise}

\end{document}
