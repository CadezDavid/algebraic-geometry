\newcommand{\sheet}{5}
\documentclass{article}
\usepackage[english, german]{babel}
\usepackage{amsthm,amssymb,amsmath,mathrsfs,mathtools}
\usepackage[shortlabels]{enumitem}
\usepackage{hyperref}
\usepackage{biblatex}
\usepackage{tikz}
\usepackage{tikz-cd}

% \usepackage[tmargin=1.25in,bmargin=1.25in,lmargin=1.2in,rmargin=1.2in]{geometry}


\newcommand{\C}{\mathbb{C}}
\newcommand{\R}{\mathbb{R}}
\newcommand{\N}{\mathbb{N}}
\newcommand{\Q}{\mathbb{Q}}
\newcommand{\Z}{\mathbb{Z}}
\newcommand{\Proj}{\mathbb{P}}
\newcommand{\Aff}{\mathbb{A}}

\DeclareMathOperator{\id}{id}
\DeclareMathOperator{\im}{im}
\DeclareMathOperator{\GL}{GL}
\DeclareMathOperator{\sgn}{sgn}
\DeclareMathOperator{\Tor}{Tor}
\DeclareMathOperator{\Sym}{Sym}
\DeclareMathOperator{\coker}{coker}
\DeclareMathOperator{\Quot}{Quot}
\DeclareMathOperator{\supp}{supp}
\DeclareMathOperator{\Hom}{Hom}
\DeclareMathOperator{\Spec}{Spec}
\DeclareMathOperator{\MinSpec}{MinSpec}
\DeclareMathOperator{\MaxSpec}{MaxSpec}
\DeclareMathOperator{\diag}{diag}
\DeclareMathOperator{\BL}{BL}
\DeclareMathOperator{\Ouv}{Ouv}
\DeclareMathOperator{\Sh}{Sh}
\DeclareMathOperator{\PSh}{PSh}
\DeclareMathOperator{\Eq}{Eq}
\DeclareMathOperator{\colim}{colim}
\DeclareMathOperator{\Pic}{Pic}
\DeclareMathOperator{\CL}{CL}
\DeclareMathOperator{\eq}{eq}
\DeclareMathOperator{\codim}{codim}

\newenvironment{exercise}[1] {
  \vspace{0.5cm}
  \noindent \textbf{Exercise~{#1}.}
} {
  \vspace{0.5cm}
}
\newenvironment{claim} {
  \par\noindent\textbf{Claim.}
} { }

\newenvironment{proof_claim} {
  \par\noindent\textbf{Proof of claim.}
} {
    \qed (of claim)
}

\title{Algebraic geometry 1\\Exercise sheet \sheet}
\author{Solutions by: Eric Rudolph and David Čadež}

\date{\today}


\begin{document}

\maketitle{}

\begin{exercise}{1}
    \begin{enumerate}
        \item{We make a pushout of the diagram $U_1 \leftarrow V_1 \rightarrow
            U_2$, where $V_1 \rightarrow U_1$ is the inclusion and $V_1
            \rightarrow U_2$ and composition of $\varphi$ and inclusion.

            Let $X$ be the pushout in terms of topological spaces and let
            $\alpha_1 \colon U_1 \rightarrow X$ and $\alpha_2 \colon U_2
            \rightarrow X$ be the associated morphisms.

            We define a sheaf $\mathcal{O}_X$ in the following way. Take an open
            subset $Z \subseteq X$. Then $Z \cap \alpha_1(U_1) = Z_1$ and $Z
            \cap \alpha_2(U_2) = Z_2$ are an open cover of $Z$ in $X$. Then let}
    \end{enumerate}

    \begin{enumerate}
        \item Define 
        \begin{align*}
            X:=U_1\coprod U_2/\!{\sim},
        \end{align*}
        where $x\sim y$ if $x=\varphi(y)$ and for $i\in \lbrace
        1,2\rbrace$
        \begin{align*}
            \pi_i&:U_i\to X\\
            x&\mapsto \bar{x}.
        \end{align*}
        We can now give $X$ the structure of a topological 
        space by defining a subset $U\subset X$ to be open
        if $\pi^{-1}(U)\in$ are open in $U_1$.\\
        Notice, that $\pi_i$ are homeomorphic onto open subsets
        of X. This will become important later.
        Next we want to define a structure sheaf on $X$ that
        behaves well with restricting to $U_i.$\\
        For $U\subset X$ open, let 
        \begin{align*}
            \mathcal{O}_X(U):= \ker (\mathcal{O}_{U_1}(\pi^{-1}(U))\oplus
            \mathcal{O}_{U_2}(\pi^{-1}(U)) \to \mathcal{O}_{U_1}(\pi^{-1}(U)
            \cap U_1)\\ (x,y) \mapsto x_{\mid \pi^{-1}(U)\cap
            V_1}-\varphi^\sharp (\pi_2^{-1}(U)\cap V_2)(y_{\mid
            \pi_2^{-1}(U)\cap V_2}) ),
        \end{align*}
            where the substraction in the above term comes from the group
            structure of $\mathcal{O}_{U_1}(\pi^{-1}(U)\cap V_1)$. This is of course
            a group again, as the kernel of a ring map.\\
            We conclude, that $(X,\mathcal{O}_X)$ is a scheme, because 
            $X=\pi_1(U_1)\cup \pi _2(U_2)$  can be covered by affine
            schemes using the cover from $U_1$ and $U_2$ and since by construction
            of the structure sheaf $\mathcal{O}_{X \mid U_1}=\mathcal{O}_{x_i}.$
            Here we finally used, as promised, that $\pi_i$ are homeomorphisms
            onto open subsets of $X$.
    \end{enumerate}
\end{exercise}
\begin{exercise}{3}
    \begin{enumerate}
        \item Remember, that the 
        contravariant functor
         $A\mapsto (Spec(A),\mathcal{O}_{Spec(A)})$
         is an equivalence of categories, meaning that 
         $\mathcal{O}^{op}$ is equivalent to $Rings$. Then
         the affine case follows from the Yoneda embedding.\\
         How to show general case?
         \\
         \\
         We want to be in the situation of the second part of
         this exercise. We choose $\mathcal{C}=Psh, \mathcal{D}=\mathcal{E}=Sh$.
         and $\mathcal{F}$ is sheafification, $\mathcal{\tilde{F}}$ is pullback
         along $f$, with $\mathcal{G}$ just being inclusion and $\tilde{\mathcal{G}}$
         being the pushforeward
         along $f$. So we can conclude that the concatenation of 
         $\mathcal{F}\circ \mathcal{\tilde{F}}\dashv \mathcal{G}\circ \mathcal{\tilde{G}$.
         Now the claim follows using the first part of the exercise.

    \end{enumerate}
\end{exercise}


\begin{exercise}{2}
    \begin{enumerate}
        \item{Take two isomorphic open immersions $(Z, \mathcal{O}_Z)$ and $(W,
            \mathcal{O}_W)$ as schemes over $(Y, \mathcal{O}_Y)$. So we have a
            commutative diagram

            \begin{tikzcd}
                & (Z, \mathcal{O}_Z) \arrow[hookrightarrow]{r} \arrow{d}{\cong}
                & (Y, \mathcal{O}_Y) \\
                & (W, \mathcal{O}_W) \arrow[hookrightarrow]{ur}
            \end{tikzcd}

            from which we get a diagram of topological spaces

            \begin{tikzcd}
                & Z \arrow[hookrightarrow]{r} \arrow{d}{\cong} & Y \\
                & W \arrow[hookrightarrow]{ur}
            \end{tikzcd}

            from which it clearly follows that $Z$ and $W$ must be equal as
            sets.

            For the other way, we want to show that for every open $Z \subseteq
            Y$ there is a unique sheaf $\mathcal{O}_Z$ for which $(\varphi,
            \varphi^\#) \colon (Z, \mathcal{O}_Z) \hookrightarrow (Y,
            \mathcal{O}_Y)$ is an open embedding. Take any two sheaves
            $\mathcal{O}_Z$ and $\mathcal{O}'_Z$ on $Z$ for which $(\mu, \mu^\#)
            \colon (Z, \mathcal{O}'_Z) \hookrightarrow (Y, \mathcal{O}_Y)$ is
            also open embedding. Then by definition of an open embedding we have
            isomorphisms $\mu^{-1}\mathcal{O}_Y \rightarrow \mathcal{O}_Z$ and
            $\varphi^{-1}\mathcal{O}_Y \rightarrow \mathcal{O}'_Z$. But
            $\varphi^{-1}\mathcal{O}_Y$ and $\mu^{-1}\mathcal{O}_Y$ are the
            same, since $\varphi = \mu$, so $\mathcal{O}'_Z \cong
            \mathcal{O}_Z$. As for the existence: there clearly exists such a
            sheaf $\mathcal{O}_Z$ simply by taking a restriction
            $\mathcal{O}_Y\mid_Z$. But (as it says in Davies/Scholze notes) it
            is not obvious. We have to show that we can cover $Z$ with open
            subsets, where each of them is isomorphic to an affine scheme. Let
            $Y = \cup_i Y_i$, where $Y_i \cong \Spec B_i$. Then for every point
            $x \in Z$ we choose $i$ such that $x \in Y_i \cap Z$. That means
            there exists some $f \in B_i$ such that $x \in D_{Y_i}(f) \subseteq
            V_i \cap U$. Since $D_{Y_i}(f) \cong B_[f^{-1}]$, we found a
            neighborhood of $x \in Z$ that is isomorphic to an affince scheme.
            We can do that for every $x \in Z$ and thus cover it. So $Z$ is
            itself a scheme.
            }
    \end{enumerate}
\end{exercise}

\begin{exercise}{4}
    \begin{enumerate}
        \item{Let $F \colon C \rightarrow D$ be a functor with adjoints $G, G'
            \colon D \rightarrow C$. By the definition of adjointness, for every
            arrow $f \colon Fc \rightarrow d$ we have unique arrows $\phi f
            \colon c \rightarrow Gd$ and $\mu f \colon c \rightarrow G'd$, such
            that $\phi$ and $\mu$ are natural. In this case take some $d \in D$
            and $c = Gd$. Then we have a unique arrow $Gd \rightarrow G'd$.

            We just have to show this is natural in $d$, so pick some other $e
            \in D$ and $FGe \rightarrow e$. Same as before we get an arrow $Gb
            \rightarrow G'b$. Using adjointness we have a commutative diagram

            \begin{tikzcd}
                &FGa \arrow{r} \arrow{d} &a \arrow{d} \\
                &FGb \arrow{r} &b
            \end{tikzcd}

            Then, using the naturality of $\mu$ gives that
            \begin{equation*}
                \mu(FGa \rightarrow a \rightarrow b) = Ga \rightarrow G'a
                \rightarrow G'b
            \end{equation*}
            and
            \begin{equation*}
                \mu(FGa \rightarrow FGb \rightarrow b) = Ga \rightarrow Gb
                \rightarrow G'b
            \end{equation*}

            Which proves that $a \mapsto (Ga \rightarrow G'a)$ is natural.
            We could easily construct an inverse $a \mapsto (G'a \rightarrow
            Ga)$ which would compose to identity.}
    \item{
        }
    \end{enumerate}
\end{exercise}

\end{document}
