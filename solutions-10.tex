\newcommand{\sheet}{10}
\documentclass{article}
\usepackage[english, german]{babel}
\usepackage{amsthm,amssymb,amsmath,mathrsfs,mathtools}
\usepackage[shortlabels]{enumitem}
\usepackage{tikz}
\usepackage{tikz-cd}

% \usepackage[tmargin=1.25in,bmargin=1.25in,lmargin=1.2in,rmargin=1.2in]{geometry}


\newcommand{\C}{\mathbb{C}}
\newcommand{\R}{\mathbb{R}}
\newcommand{\N}{\mathbb{N}}
\newcommand{\Q}{\mathbb{Q}}
\newcommand{\Z}{\mathbb{Z}}

\DeclareMathOperator{\id}{id}
\DeclareMathOperator{\im}{im}
\DeclareMathOperator{\GL}{GL}
\DeclareMathOperator{\sgn}{sgn}
\DeclareMathOperator{\Tor}{Tor}
\DeclareMathOperator{\Sym}{Sym}
\DeclareMathOperator{\coker}{coker}
\DeclareMathOperator{\Quot}{Quot}
\DeclareMathOperator{\supp}{supp}
\DeclareMathOperator{\Hom}{Hom}
\DeclareMathOperator{\Spec}{Spec}
\DeclareMathOperator{\MinSpec}{MinSpec}
\DeclareMathOperator{\diag}{diag}


\newenvironment{exercise}[1] {
  \vspace{0.5cm}
  \noindent \textbf{Exercise~{#1}.}
} {
  \vspace{0.5cm}
}
\newenvironment{claim} {
  \noindent \textbf{Claim.}
} {
}

\title{Algebraic geometry 1\\Exercise sheet \sheet}
\author{Solutions by: Eric Rudolph and David Čadež}

\date{\today}


\begin{document}

\maketitle{}

\begin{exercise}{1}
    \begin{enumerate}
        \item{
                Since $X$ is closed and irreducible, it is of the form $X =
                \overline{\{p_0\}}$ for some (Eric thinks unique) $p_0 \in \Aff^n_k$.
                That means $X \cong \Spec(k[x_1, \dots x_n] / p_0)$.
                Denote $A = k[x_1, \dots x_n] / p_0$.

                By assumption there is a chain of specializations $p_0 \subset
                \dots \subset p_d$ inside $X$.

                Let $Z \subseteq X \cap V(f_1, \dots, f_r)$ be a irreducible
                component. Thus it is the closure of a minimal prime ideal in $A
                / (f_1, \dots, f_r)$.

                By Krull's principal ideal theorem we have $\dim(A / (f_1,
                \dots, f_r)) \geq d - r$.

                Denote minimal prime ideals in $A / (f_1, \dots, f_r)$ with
                $q_1, \dots q_l$.

                (Eric thinks that there is a unique generic point here again, since $X$ is sober, so 
                there should only be one of these prime ideals, right?)
                
                We argue that
                \begin{equation*}
                    \dim(A / (f_1, \dots, f_r) / q_j) = \dim(A / (f_1, \dots, f_r)).
                \end{equation*}
                for any $j$.

                That follows from $A$ being catenary.
                If there existed a maximal chain in $A / (f_1, \dots, f_r)$ that
                starts at $q_j$ we could simply extend it below to get a
                maximal chain in $A$. Since all maximal chains in $A$ are of the
                same length, we get that all maximal chains in $A / (f_1, \dots,
                f_r)$ are also of the same length.

                Since $Z$ is an irreducible component, we have $Z =
                \overline{\{q_i\}} \subseteq \Spec(A / (f_1, \dots, f_r))$.

                Therefore any maximal chain in $Z$ is exactly as long as the
                longest chain in $A / (f_1, \dots, f_r)$. And the longest chain
                in $A / (f_1, \dots, f_r)$ is at least of length $d - r$.
            }
        \item{
                The diagonal $\triangle \subseteq \Aff^n_k \times \Aff^n_k$ can
                be defined as $V(x_i \otimes x_i \mid i = 1, \dots n) \subseteq
                \Spec(k[x_1, \dots, x_n] \otimes_\Z k[x_1, \dots, x_n])$.

                (Should there not be a minus instead of $\otimes$ in the above expression?)

                Using exercise above we get that any irreducible component of
                $X \cap Y \cong (X \times Y) \cap V(x_i \otimes x_i \mid i = 1,
                \dots n)$ has dimension at least $d + e - n$.
            }
        \item{
                Let $\tilde{X} = \overline{f^{-1}(X)}$ and $\tilde{Y} = \overline{f^{-1}(Y)}$ as in
                the hint.

                We have $\dim(\tilde{X}) = d + 1$ and $\dim(\tilde{Y}) = e + 1$.
                By the previous exercise we have $\dim(\tilde{X} \cap \tilde{Y})
                \geq d + 1 + e + 1 - (n + 1) = (d + e - n) + 1 \geq 1$.

                Therefore there there exists $0 \not= x \in \tilde{X} \cap
                \tilde{Y}$.

                Questions from Eric:
                
                Why is $\tilde{X}$ irreducible (to be able to use part 2) and why does the dimension 
                increase by 1 when we go to affine space?
            }
    \end{enumerate}
\end{exercise}

\begin{exercise}{2}
    \begin{enumerate}
        \item{It is enough to show that there exists a cover $X=\cup_i Spec(A_i)$ of $X$ by open affines 
        such that $f^{-1}(Spec(A_i))$ is affine for all $i$. Therefore it is enough to show that the hint holds,
        since $X=\cup_{x\in X} U_x$, where each $U_x$ is an open affine with $x\in U_x$.

        Take some $x\in X$. Then $x\in Spec(A_k)$ for some $k$. 
        Now choose some principal open $D(g)\subset Z$ with $f^{-1}(x)\in D(g)$. 

        We can now take a principal open $D(g^\prime) \subset f(D(g))$ such that $D(g^\prime) \subset U_k$ and $x\in D(g^\prime)$.
        Then we can show similarly to exercise 4.1 on sheet 8 that 
        \begin{align*}
            f^{-1}(D(g^\prime))
        \end{align*}
        is a principal open again, so in particular affine.
            }
        \item{
            }
    \end{enumerate}
\end{exercise}

\end{document}
