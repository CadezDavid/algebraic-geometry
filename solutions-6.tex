\newcommand{\sheet}{6}
\documentclass{article}
\usepackage[english, german]{babel}
\usepackage{amsthm,amssymb,amsmath,mathrsfs,mathtools}
\usepackage[shortlabels]{enumitem}
\usepackage{hyperref}
\usepackage{biblatex}
\usepackage{tikz}
\usepackage{tikz-cd}

% \usepackage[tmargin=1.25in,bmargin=1.25in,lmargin=1.2in,rmargin=1.2in]{geometry}


\newcommand{\C}{\mathbb{C}}
\newcommand{\R}{\mathbb{R}}
\newcommand{\N}{\mathbb{N}}
\newcommand{\Q}{\mathbb{Q}}
\newcommand{\Z}{\mathbb{Z}}
\newcommand{\Proj}{\mathbb{P}}
\newcommand{\Aff}{\mathbb{A}}

\DeclareMathOperator{\id}{id}
\DeclareMathOperator{\im}{im}
\DeclareMathOperator{\GL}{GL}
\DeclareMathOperator{\sgn}{sgn}
\DeclareMathOperator{\Tor}{Tor}
\DeclareMathOperator{\Sym}{Sym}
\DeclareMathOperator{\coker}{coker}
\DeclareMathOperator{\Quot}{Quot}
\DeclareMathOperator{\supp}{supp}
\DeclareMathOperator{\Hom}{Hom}
\DeclareMathOperator{\Spec}{Spec}
\DeclareMathOperator{\MinSpec}{MinSpec}
\DeclareMathOperator{\MaxSpec}{MaxSpec}
\DeclareMathOperator{\diag}{diag}
\DeclareMathOperator{\BL}{BL}
\DeclareMathOperator{\Ouv}{Ouv}
\DeclareMathOperator{\Sh}{Sh}
\DeclareMathOperator{\PSh}{PSh}
\DeclareMathOperator{\Eq}{Eq}
\DeclareMathOperator{\colim}{colim}
\DeclareMathOperator{\Pic}{Pic}
\DeclareMathOperator{\CL}{CL}
\DeclareMathOperator{\eq}{eq}
\DeclareMathOperator{\codim}{codim}

\newenvironment{exercise}[1] {
  \vspace{0.5cm}
  \noindent \textbf{Exercise~{#1}.}
} {
  \vspace{0.5cm}
}
\newenvironment{claim} {
  \par\noindent\textbf{Claim.}
} { }

\newenvironment{proof_claim} {
  \par\noindent\textbf{Proof of claim.}
} {
    \qed (of claim)
}

\title{Algebraic geometry 1\\Exercise sheet \sheet}
\author{Solutions by: Eric Rudolph and David Čadež}

\date{\today}


\begin{document}

\maketitle{}

\begin{exercise}{1}
    \begin{enumerate}
        \item By the universal property of the fiber product of locally ringed spaces,
        we have the following commutative diagram
        \begin{center}
%TODO I think U_i should be X and V_j should be Y. Maps XxY -> U_i
            %and XxY -> V_j dont necessarily exist.
            \begin{tikzcd}
                U_i\times_{S_{i,j}}V_j\arrow[dotted]{dr}\arrow[bend right]{ddr}{p}\arrow[bend left]{rrd}{q}\\
                &X \times_S Y \arrow{r}{\pi_2}\arrow{d}{\pi_1} & V_j\arrow{d}{\psi}\\
                &U_i \arrow{r}{\phi} & S_{i,j}
            \end{tikzcd}
        \end{center}
        Therefore, on the level of sets,
        \begin{align*}
            U_i\times_{S_{i,j}}V_j \subset X \times_S Y,
        \end{align*}
        but in exercise 5.2.1, we showed that this induces 
        an open immersion as locally ringed spaces.

        Now observe that 
        \begin{center}
            \begin{tikzcd}
               &\bigcup_{i,j}\left(U_i\times_{S_{i,j}}V_j\right)\arrow{r}\arrow{d} & Y\arrow{d}\\
                &X \arrow{r} & S
            \end{tikzcd}
        \end{center}
        commutes, because $S=\bigcup_{i,j}S_{i,j}$.
        Now by uniqueness of the pullback, 
        \begin{align*}
           \bigcup_{i,j}\left(U_i\times_{S_{i,j}}V_j\right)\cong U_i\times_{S_{i,j}}V_j.
        \end{align*}

        I guess this is a good step in the direction of understanding
        why the pullback in the category of sheaves exists, right? If 
        we assume $X, Y, S$ to be sheaves and $U_i, V_j, S_{i,j}$ to be
        affine schemes, then by the above argument we found a cover
        of $X\times_S Y$ by affine schemes.

    \item[1.\ (alternative)]{
            Let $U \subseteq X$, $V \subseteq Y$ and $W \subseteq S$ open
            subschemes. By the universal property of the fiber product of
            locally ringed spaces, we have the following commutative diagram
            \begin{center}
                \begin{tikzcd}
                    &U\times_W V \arrow[dotted]{dr} \arrow{d}{p} \arrow{r}{q} &V
                    \arrow[hook]{dr} \\
                    &U \arrow[hook]{dr} &X \times_S Y
                    \arrow{r}{\pi_2}\arrow{d}{\pi_1} &Y \arrow{d}{\psi}\\
                    &&X \arrow{r}{\phi} &S
                \end{tikzcd}
            \end{center}
            so we get a unique map
            $U \times_W V \rightarrow X \times_S Y$.

            Observe the concrete space $\pi^{-1}_1(U) \cap \pi^{-1}_2(V)$ with
            inclusion
            \begin{equation*}
                \pi^{-1}_1(U) \cap \pi^{-1}_2(V) \hookrightarrow X \times_S Y
            \end{equation*}
            also satisfies the universal property of being a fibred product
            $U \times_S V$. If $T \to U$ and $T \to V$ such that
            $T \to U \to W = T \to V \to W$, then
            we can create the following diagram
            \begin{center}
                \begin{tikzcd}
                    &T \arrow[dotted]{dr} \arrow{d}{p} \arrow{r}{q} &V
                    \arrow[hook]{dr} \\
                    &U \arrow[hook]{dr} &X \times_S Y
                    \arrow{r}{\pi_2}\arrow{d}{\pi_1} &Y \arrow{d}{\psi}\\
                    &&X \arrow{r}{\phi} &S
                \end{tikzcd}
            \end{center}
            from which we get a unique map $T \to X \times_S Y$. Since its image
            is contained in $\pi^{-1}_1(U) \cap \pi^{-1}_2(V)$, it factors
            uniquely through
            \begin{equation*}
                T \to \pi^{-1}_1(U) \cap \pi^{-1}_2(V) \hookrightarrow X
                \times_S Y
            \end{equation*}

            % also $T \to U \to X \to S = T
            % \to V \to Y \to S$. So by the universal property of fibred product
            % $X \times_S Y$ we get a unique map $T \to X \times_S Y$. Since $T
            % \to U \to X = T \to X \times_S \to X$ and same for $Y$, the image
            % of $T \to X \times_S Y$ is contained in $\pi^{-1}_1(U) \cap
            % \pi^{-1}_2(V)$.

            Therefore, the fibre product $U \times_W V$ can be identified as an
            open subspace $\pi^{-1}_1(U) \cap \pi^{-1}_2(V) \subseteq X \times_S
            Y$.

            Then clearly for coverings $X = \cup_i U_i$, $Y = \cup_i V_i$ and $S =
            \cup_{i, j} S_{i, j}$, we have a covering
            \begin{equation*}
                \bigcup_{i, j} U_i \times_{S_{i, j}} V_j = \bigcup_{i, j}
                (\pi^{-1}_1(U_i) \cap \pi^{-1}_2(V_j)) = X \times_S Y.
            \end{equation*}

            I guess this is a good step in the direction of understanding why the
            pullback in the category of schemes exists, right? If we assume $X, Y,
            S$ to be sheaves and $U_i, V_j, S_{i,j}$ to be affine schemes, then by
            the above argument we found a cover of $X\times_S Y$ by affine schemes.}

        \item{
            Surjectivity follows, because a pullback of schemes in partiular
            makes
            \begin{center}
                \begin{tikzcd}
                    &\mid X\times_S Y\mid \arrow{r}\arrow{d} & \mid X \mid\arrow{d}{\psi}\\
                    &\mid Y\mid \arrow{r}{\phi} & \mid S \mid
                \end{tikzcd}
            \end{center}
            commute for all $\psi, \phi$ from maps of schemes.
        }
        \item[2.\ (alternative)]{
            If $X = \Spec(A), Y = \Spec(B), Z = \Spec(R)$ were affine schemes,
            we would have homeomorphism, since $\Spec(A \otimes_R B) = \Spec(A)
            \times_{\Spec(R)} \Spec(B)$. But in schemes (using that the fiber
            product is also a scheme) we only locally have isomorphisms, which
            is means the map must be surjective.

            Concretely:
            For every $(x, y) \in |X| \times_{|S|} |Y|$ we have $x \in \Spec(A)
            \subseteq |X|$ and $y \in \Spec(B) \subseteq |Y|$ such that $|X
            \times_S Y| \to |X| \times_{|S|} |Y|$ restricted to $\Spec(A
            \otimes_R B)$ will be isomorphism (with $(x, y)$ in its image).
            }
    \end{enumerate}
\end{exercise}

\begin{exercise}{2}
    %TODO add remark about previous exercise
    \begin{enumerate}
        \item{
            First let $f \colon X \rightarrow S$ be open immersion. In this case
            we can directly use previous exercise on the following fibred
            product
                      
        \begin{center}
            \begin{tikzcd}
                &S \arrow{r}{\id} &S \\
                &S \times_S S' \arrow{r}{q} \arrow{u}{p} &S' \arrow{u}{g}
            \end{tikzcd}
        \end{center}
        
            by taking subset of $X \subseteq S$ and immediately getting open
            immersion $X \times_S S' \rightarrow S \times_S S'$, which we
            postcompose with canonical isomorphism $S \times_S S' \rightarrow
            S'$ and get that $X \times_S S' \rightarrow S'$ is open immersion.

            Now suppose $f \colon X \rightarrow S$ is a closed immersion. So we
            have the following diagram

            \begin{center}
            \begin{tikzcd}
                &X \arrow{r}{f} &S \\
                &X \times_S S' \arrow{r}{q} \arrow{u}{p} &S' \arrow{u}{g}
            \end{tikzcd}
            \end{center}

            We want to show $X \times_S S' \rightarrow S'$ is also a closed
            immersion. For that it satisfies to find an open covering of $S'$
            with affine subschemes such that preimages with be also affine schemes
            and induced maps of rings surjective.

            Take $s \in S'$ and a neighborhood $g(s) \in \Spec(R) = U \subseteq
            S$. Preimage $f^{-1}(U) = \Spec(A)$ already is affine, since $f$ is
            closed immersion, and for $g^{-1}(U)$ we have to take some smaller
            affine neighborhood of $s$. So we get $s \in \Spec(B) \subseteq
            g^{-1}(U)$. Then use previous exercise on these open sets and obtain
            open immersion
            \begin{equation*}
                \Spec(A) \times_{\Spec(R)} \Spec(B) = \Spec(A \otimes_R B)
                \rightarrow X \times_S S'.
            \end{equation*}
            By remark at the start we have
            \begin{equation*}
                \Spec(A \otimes_R B) = p^{-1}(\Spec(A)) \cap q^{-1}(\Spec(B)) =
                q^{-1}(\Spec(B)).
            \end{equation*}
            Only thing to argue is why the map $B \rightarrow A \otimes_R B$ is
            surjective. Since $R \twoheadrightarrow A$ surjective, $R/I \cong A$
            and thus $A \otimes_R B = R/I \otimes_R B = B / IB$. Clearly $B
            \rightarrow B/IB$ is then surjective.
        \item{}
    \end{enumerate}
\end{exercise}

\begin{exercise}{3}
    By definition we have to compute a fibred product of $\Spec(B) \to \Spec(A)$
    and $\Spec(k(p)) \to \Spec(A)$ (where $k(p)$ is the residue field of $p \in
    \Spec(A)$ and $\to$ is the canonical inclusion). Since we are dealing with
    affine schemes, we can express it concretely as $\Spec(B \otimes_A k(p))$.
    Note that $B$ has the structure of an $A$-algebra, which is induced
    by the starting morphism of schemes $\Spec(B) \rightarrow \Spec(A)$.
    So this exercise reduces to computing these tensor products.
    
    We also observe that $k[T]$ is a PID, which means every non-zero prime ideal
    is a maximal ideal. This will be handy when computing residue fields,
    because after quotienting with a non-zero ideal we already get a field (we do
    not have to further take the quotient field).

    \begin{enumerate}
        \item{In the first example we do now even have to calculate the tensor
            product, because we can rewrite $k[T, U]/(TU - 1) = k[T, T^{-1}]$,
            so this is just a localization of $k[T]$. Morphism of spectrums,
            induced by inclusion into localization, is an open immersion, so
            fibers will be singletons if $x \in D(T)$ and empty sets otherwise.
            And the structure sheaf is also clear, it is just the restriction of
            structure sheaf $\mathcal{O}_{\Spec(k[T])}$.}
        \item{}
        \item{}
        \item{}
    \end{enumerate}
\end{exercise}

\begin{exercise}{4}
    This proof is the same as when we proved that $\mathcal{O}_{\Spec(A)}$ is a
    sheaf, after we defined it on the basis of principal opens.

    Take $U = D(f)$ for some $f \in A$ and let $U = \cup_i D(f_i)$ be some
    cover. We have to check that 
    \begin{equation*}
        M[f^{-1}] \rightarrow \Eq\left[ \prod_i M[f^{-1}_i] \rightrightarrows
        \prod_{i, j} M[(f_i f_j)^{-1}]  \right]
    \end{equation*}
    is isomorphism.

    We make following simplifications, namely we can set $M := M[f^{-1}]$ and
    assume $I$ is finite, say $I = \{1, \dots, n\}$ (we can do that since
    $\Spec(A)$ is quasi-compact).

    So we are trying to show $M$ is isomorphic to a submodule of $\prod_i
    M[f^{-1}_i]$ defined as $\{ (m_1, \dots, m_n) \in \prod_i
    M[f^{-1}_i] \mid m_i = m_j \in M[(f_i f_j)^{-1}] \}$.

    Injectivity:
    Take $m \in M$. Since $m = 0 \in M[f^{-1}_i]$, we have $f^{k_i}_i m = 0$ for
    every $i$ for some $k_i$. Since $I$ is finite, take $k = \max_i k_i$.
    Since $D(f^k_i)$ is still a cover, we have $1 = \sum_i a_i f^k_i$ for some
    $a_i \in A$. Then $1m = \sum_i a_i f^k_i m = 0$, so $m = 0$.

    Surjectivity:
    Let $(m_1, \dots, m_n) \in \prod_i M[f^{-1}_i]$ with $m_i = m_j \in M[(f_i
    f_j)^{-1}]$ for all $i, j$. Write $m_i = \frac{a_i}{f^{k_i}_i}$. WLOG $k =
    \max_i k_i$. For every pair there exists $l \in \N$ such that $(f_i f_j)^l
    (m_i - m_j) = 0$. Take $l$ again to be maximum over all pairs. Because
    $D(f^l_i)$ is still a cover, we have $1 = \sum_i b_i f^l_i$. Then define $s
    = \sum_i b_i a_i$. Clearly $f^l_j s = \sum_i b_i f^l_j a_i =  \sum_i b_i
    f^l_i a_j = a_j$. So $(m_1, \dots, m_n)$ is the image of $s$.
\end{exercise}

\end{document}
