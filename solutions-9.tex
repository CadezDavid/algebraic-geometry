\newcommand{\sheet}{9}
\documentclass{article}
\usepackage[english, german]{babel}
\usepackage{amsthm,amssymb,amsmath,mathrsfs,mathtools}
\usepackage[shortlabels]{enumitem}
\usepackage{hyperref}
\usepackage{biblatex}
\usepackage{tikz}
\usepackage{tikz-cd}

% \usepackage[tmargin=1.25in,bmargin=1.25in,lmargin=1.2in,rmargin=1.2in]{geometry}


\newcommand{\C}{\mathbb{C}}
\newcommand{\R}{\mathbb{R}}
\newcommand{\N}{\mathbb{N}}
\newcommand{\Q}{\mathbb{Q}}
\newcommand{\Z}{\mathbb{Z}}
\newcommand{\Proj}{\mathbb{P}}
\newcommand{\Aff}{\mathbb{A}}

\DeclareMathOperator{\id}{id}
\DeclareMathOperator{\im}{im}
\DeclareMathOperator{\GL}{GL}
\DeclareMathOperator{\sgn}{sgn}
\DeclareMathOperator{\Tor}{Tor}
\DeclareMathOperator{\Sym}{Sym}
\DeclareMathOperator{\coker}{coker}
\DeclareMathOperator{\Quot}{Quot}
\DeclareMathOperator{\supp}{supp}
\DeclareMathOperator{\Hom}{Hom}
\DeclareMathOperator{\Spec}{Spec}
\DeclareMathOperator{\MinSpec}{MinSpec}
\DeclareMathOperator{\MaxSpec}{MaxSpec}
\DeclareMathOperator{\diag}{diag}
\DeclareMathOperator{\BL}{BL}
\DeclareMathOperator{\Ouv}{Ouv}
\DeclareMathOperator{\Sh}{Sh}
\DeclareMathOperator{\PSh}{PSh}
\DeclareMathOperator{\Eq}{Eq}
\DeclareMathOperator{\colim}{colim}
\DeclareMathOperator{\Pic}{Pic}
\DeclareMathOperator{\CL}{CL}
\DeclareMathOperator{\eq}{eq}
\DeclareMathOperator{\codim}{codim}

\newenvironment{exercise}[1] {
  \vspace{0.5cm}
  \noindent \textbf{Exercise~{#1}.}
} {
  \vspace{0.5cm}
}
\newenvironment{claim} {
  \par\noindent\textbf{Claim.}
} { }

\newenvironment{proof_claim} {
  \par\noindent\textbf{Proof of claim.}
} {
    \qed (of claim)
}

\title{Algebraic geometry 1\\Exercise sheet \sheet}
\author{Solutions by: Eric Rudolph and David Čadež}

\date{\today}


\begin{document}

\maketitle{}

\begin{exercise}{2}
    \begin{enumerate}
        \item{
            Suppose we have a vector bundle of rank $n$ on $\Proj^1_k$. How do
            we construct a matrix $\in \GL_n(k[T^{\pm 1}])$?

            Take a rank $n$ vector bundle $\mathcal{E}$. Since Picard group of
            $U_0$ and $U_1$ are trivial, we have isomorphisms $\alpha, \beta$.
            So on $\Spec(k[T^{\pm 1}]) \subseteq U_0$ we have an isomorphism
            $\Gamma(U_0 \cap U_1, \mathcal{O}^n_{U_0}) = (k[T^{\pm 1}])^n \cong
            \mathcal{E}(U_0 \cap U_1)$.

            Combining this with an isomorphism $\mathcal{E}(U_0 \cap U_1) \cong
            (k[T^{\pm 1}])^n = \Gamma(U_0 \cap U_1, \mathcal{O}^n_{U_1})$, we
            get an isomorpism $(k[T^{\pm 1}])^n \cong (k[T^{\pm 1}])^n$.

            Let $\mathcal{D}$ be another rank $n$ vector bundle on $\Proj^1_k$,
            and let $\varphi \colon \mathcal{E} \to \mathcal{D}$ be an
            isomorphism between them.
            On $U_0$ and $U_1$ we get induced isomorphisms
            \begin{equation*}
                (k[T])^n = \mathcal{E}(U_0) \to \mathcal{D}(U_0) = (k[T])^n
            \end{equation*}
            and
            \begin{equation*}
                (k[T^{-1}])^n = \mathcal{E}(U_1) \to \mathcal{D}(U_1) = (k[T^{-1}])^n
            \end{equation*}
        }
    \item{
            Take $G = \begin{pmatrix} p_1 & p_2 \\ p_3 & p_4 \end{pmatrix} \in
                \GL_n(k[T^{\pm 1}])$. We can writte each $p_i =
                \frac{g_i}{T^{k_i}}$ for some $g_i \in k[T]$. We can take $k_i =
                k$ to be all the same. Then
                \begin{equation}
                    \begin{pmatrix} p_1 & p_2 \\ p_3 & p_4 \end{pmatrix} =
                        \begin{pmatrix} T^{-k} & 0 \\ 0 & T^{-k} \end{pmatrix} 
                    \begin{pmatrix} g_1 & g_2 \\ g_3 & g_4 \end{pmatrix} = 
                \end{equation}
        }
    \item{
        }
    \end{enumerate}
\end{exercise}

\end{document}
