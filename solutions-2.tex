\newcommand{\sheet}{2}
\documentclass{article}
\usepackage[english, german]{babel}
\usepackage{amsthm,amssymb,amsmath,mathrsfs,mathtools}
\usepackage[shortlabels]{enumitem}
\usepackage{hyperref}
\usepackage{biblatex}
\usepackage{tikz}
\usepackage{tikz-cd}

% \usepackage[tmargin=1.25in,bmargin=1.25in,lmargin=1.2in,rmargin=1.2in]{geometry}


\newcommand{\C}{\mathbb{C}}
\newcommand{\R}{\mathbb{R}}
\newcommand{\N}{\mathbb{N}}
\newcommand{\Q}{\mathbb{Q}}
\newcommand{\Z}{\mathbb{Z}}
\newcommand{\Proj}{\mathbb{P}}
\newcommand{\Aff}{\mathbb{A}}

\DeclareMathOperator{\id}{id}
\DeclareMathOperator{\im}{im}
\DeclareMathOperator{\GL}{GL}
\DeclareMathOperator{\sgn}{sgn}
\DeclareMathOperator{\Tor}{Tor}
\DeclareMathOperator{\Sym}{Sym}
\DeclareMathOperator{\coker}{coker}
\DeclareMathOperator{\Quot}{Quot}
\DeclareMathOperator{\supp}{supp}
\DeclareMathOperator{\Hom}{Hom}
\DeclareMathOperator{\Spec}{Spec}
\DeclareMathOperator{\MinSpec}{MinSpec}
\DeclareMathOperator{\MaxSpec}{MaxSpec}
\DeclareMathOperator{\diag}{diag}
\DeclareMathOperator{\BL}{BL}
\DeclareMathOperator{\Ouv}{Ouv}
\DeclareMathOperator{\Sh}{Sh}
\DeclareMathOperator{\PSh}{PSh}
\DeclareMathOperator{\Eq}{Eq}
\DeclareMathOperator{\colim}{colim}
\DeclareMathOperator{\Pic}{Pic}
\DeclareMathOperator{\CL}{CL}
\DeclareMathOperator{\eq}{eq}
\DeclareMathOperator{\codim}{codim}

\newenvironment{exercise}[1] {
  \vspace{0.5cm}
  \noindent \textbf{Exercise~{#1}.}
} {
  \vspace{0.5cm}
}
\newenvironment{claim} {
  \par\noindent\textbf{Claim.}
} { }

\newenvironment{proof_claim} {
  \par\noindent\textbf{Proof of claim.}
} {
    \qed (of claim)
}

\title{Algebraic geometry 1\\Exercise sheet \sheet}
\author{Solutions by: Eric Rudolph and David Čadež}

\date{\today}


\begin{document}

\maketitle

\begin{exercise}{1}
    Let $I = (f_1, \dots, f_r) \subseteq k[x_1, \dots, x_n]$ be an ideal and $X
    = V(I) \subseteq \mathbb{A}^n(k)$ be its vanishing locus.
    \begin{enumerate}
        \item{} We have to show
            \[
                \overline{X} = \bigcap \{V^+(h) \mid h\ \text{homogenous}, h(X) = 0\} =
                V(\{ \tilde{g} \mid g \in I \}).
            \]
            Pick any homogenous $h$ that vanishes on $X$. By letting $x_{n+1} =
            1$ we see that $h(x_1, \dots, x_n, 1) \in \sqrt{I}$, since it vanishes on
            $X$. Therefore we have $l \in \N$ such that $h(x_1, \dots, x_n, 1)^l
            = \sum_{i = 1}^{r} \alpha_i f_i$ for some $\alpha_i \in k[x_1,
            \dots, x_n]$. There also exists $m \in \N$ such that $x^m_{n+1}
            \widetilde{h(x_1, \dots, x_n, 1)} = h$. Adding these two together we
            get $x^{ml}_{n+1} \widetilde{\sum_{i = 1}^{r} \alpha_i f_i} = h^l$.
    \end{enumerate}
\end{exercise}

\end{document}
