\newcommand{\sheet}{10}
\documentclass{article}
\usepackage[english, german]{babel}
\usepackage{amsthm,amssymb,amsmath,mathrsfs,mathtools}
\usepackage[shortlabels]{enumitem}
\usepackage{tikz}
\usepackage{tikz-cd}

% \usepackage[tmargin=1.25in,bmargin=1.25in,lmargin=1.2in,rmargin=1.2in]{geometry}


\newcommand{\C}{\mathbb{C}}
\newcommand{\R}{\mathbb{R}}
\newcommand{\N}{\mathbb{N}}
\newcommand{\Q}{\mathbb{Q}}
\newcommand{\Z}{\mathbb{Z}}

\DeclareMathOperator{\id}{id}
\DeclareMathOperator{\im}{im}
\DeclareMathOperator{\GL}{GL}
\DeclareMathOperator{\sgn}{sgn}
\DeclareMathOperator{\Tor}{Tor}
\DeclareMathOperator{\Sym}{Sym}
\DeclareMathOperator{\coker}{coker}
\DeclareMathOperator{\Quot}{Quot}
\DeclareMathOperator{\supp}{supp}
\DeclareMathOperator{\Hom}{Hom}
\DeclareMathOperator{\Spec}{Spec}
\DeclareMathOperator{\MinSpec}{MinSpec}
\DeclareMathOperator{\diag}{diag}


\newenvironment{exercise}[1] {
  \vspace{0.5cm}
  \noindent \textbf{Exercise~{#1}.}
} {
  \vspace{0.5cm}
}
\newenvironment{claim} {
  \noindent \textbf{Claim.}
} {
}

\title{Algebraic geometry 1\\Exercise sheet \sheet}
\author{Solutions by: Eric Rudolph and David Čadež}

\date{\today}


\begin{document}

\maketitle{}

\begin{exercise}{3}
    \begin{enumerate}
        \item{
                Taking global sections we obtain a long exact sequence
                \begin{equation*}
                    0 \to F(X) \to F'(X) \to F''(X) \to H^1(X, F) \to H^1(X, F')
                    \to \dots
                \end{equation*}
                We quickly see that dimensions satisfy
                \begin{equation*}
                    0 = \dim F(X) - \dim F'(X) + \dim F''(X) - \dim H^1(X, F) +
                    \dim H^1(X, F') - \dots
                \end{equation*}
                which shows what we wanted to show.

                We can prove this for example by induction. Clearly it holds for
                base cases. And the induction step:
                Suppose we have an exact sequence of $k$-vector spaces
                \begin{equation*}
                    0 \to C_0 \to \dots \to C_n \to 0
                \end{equation*}
                By induction hypothesis for
                \begin{equation*}
                    0 \to C_0 \to \dots \to C_{n-2} \to \im(d_{n-2}) \to 0
                \end{equation*}
                we get
                \begin{equation*}
                    0 = \dim C_0 - \dim C_1 + \dots \pm \dim C_{n-2} \mp \dim
                    \im(d_{n-2})
                \end{equation*}
                Then just substitute
                \begin{equation*}
                    \dim C_{n-1} = \dim \im(d_{n-2}) + \dim \coker(d_{n-1})
                    = \dim \im(d_{n-2}) + \dim C_n
                \end{equation*}
                and we get what we want.
            }
        \item{
                We started solving the exercise with $d \in \N$ in mind, so
                first solution is only valid for $d > 0$ (although we could
                probably somehow extend it).

                We do induction on the sum $n + d$.

                \emph{The base cases:}
                When $n = 0$, then $\Proj^n_k = \Spec(k)$, so $H^0(\Proj^n_k,
                \mathcal{O}_{\Proj^n_k}(d)) = k$ and all higher cohomologies
                vanish. Integer $d$ here doesn't make a difference because there
                is no nontrivial line bundles on a point.
                So $\chi(\Proj^n_k, \mathcal{O}_{\Proj^n_k}) = 1$.

                When $d = 0$ and $n > 0$ we are working with structure
                sheaf and in that case we know
                \begin{equation*}
                    H^q(\Proj^n_k, \mathcal{O}_{\Proj^n_k})
                    = \begin{cases}
                        k & q = 0\\
                        0 & q > 0
                    \end{cases}
                \end{equation*}
                So $\chi(\Proj^n_k, \mathcal{O}_{\Proj^n_k}) = 1$.

                \emph{The induction step:}
                Let $i \colon V(x_n) \cong \Proj^{n-1}_k \to \Proj^n_k$ be the closed
                immersion.
                Then we have an exact sequence of sheaves on $\Proj^n_k$
                \begin{equation*}
                    0 \to \mathcal{O}_{\Proj^n_k}(-1) \to
                    \mathcal{O}_{\Proj^n_k} \to i_* \mathcal{O}_{\Proj^{n-1}_k}
                    \to 0
                \end{equation*}
                Tensor this sequence with $\mathcal{O}_{\Proj^n_k}(d)$ to obtain
                \begin{equation*}
                    0 \to \mathcal{O}_{\Proj^n_k}(d-1) \to
                    \mathcal{O}_{\Proj^n_k}(d) \to (i_*
                    \mathcal{O}_{\Proj^{n-1}_k}) \otimes
                    \mathcal{O}_{\Proj^n_k}(d) \to 0
                \end{equation*}

                The last term
                \begin{equation*}
                    (i_* \mathcal{O}_{\Proj^n_k}) \otimes \mathcal{O}_{\Proj^n_k}(d) = 
                    i_* (\mathcal{O}_{\Proj^n_k} \otimes i^*
                    \mathcal{O}_{\Proj^n_k}(d)) = 
                    i_* (\mathcal{O}_{\Proj^n_k} \otimes \mathcal{O}_{\Proj^{n-1}_k}(d)) = 
                    i_* \mathcal{O}_{\Proj^{n-1}_k}(d)
                \end{equation*}

                Now use the previous part of the exercise to obtain
                \begin{equation*}
                    \chi(\Proj^n_k, \mathcal{O}_{\Proj^n_k}(d)) = 
                    \chi(\Proj^n_k, \mathcal{O}_{\Proj^n_k}(d-1)) +
                    \chi(\Proj^n_k, i_* \mathcal{O}_{\Proj^{n-1}_k}(d))
                \end{equation*}
                The cohomology of the pushforward along a closed immersion is
                the same as the cohomology of the original sheaf, so the last
                term above is $\chi(\Proj^{n-1}_k,
                \mathcal{O}_{\Proj^{n-1}_k}(d))$
                By induction hypothesis we obtain
                \begin{equation*}
                    \chi(\Proj^n_k, \mathcal{O}_{\Proj^n_k}(d)) = 
                    \binom{n + d - 1}{n} + \binom{n - 1 + d}{n - 1} = \binom{n +
                    d}{n}
                \end{equation*}
                which is what we needed to show.

                At the end let us treat the case when $d < 0$.
                Recall the explicit cohomology groups that we
                calculated in the lecture 17:
                \begin{equation*}
                    H^q(\Proj^n_k, \mathcal{O}_{\Proj^n_k}(d)) =
                    \begin{cases}
                        k[x_0, \dots, x_n]_d & q = 0 \\
                        (\frac{1}{x_0 \dots x_n} k[\frac{1}{x_0}, \dots,
                        \frac{1}{x_n}])_d & q = n \\
                        0 & \text{else}
                    \end{cases}
                \end{equation*}
                Immediately we see that for $d < 0$ we have
                \begin{equation*}
                    \chi(\Proj^n_k, \mathcal{O}_{\Proj^n_k}(d)) = (-1)^n
                    H^n(\Proj^n_k, \mathcal{O}_{\Proj^n_k}(d))
                \end{equation*}

                Now we just have to count the size of basis (monomials) in
                $(\frac{1}{x_0 \dots x_n} k[\frac{1}{x_0}, \dots,
                \frac{1}{x_n}])_d$.

                If $d \in \{-n, \dots, -1\}$, then there are no polynomials in
                $(\frac{1}{x_0 \dots x_n} k[\frac{1}{x_0}, \dots,
                \frac{1}{x_n}])_d$, so $\chi(\Proj^n_k,
                \mathcal{O}_{\Proj^n_k}(d)) = 0$.

                If $d < -n$, then there are $\binom{n + (-d -n -1)}{n}$
                monomials in
                $(\frac{1}{x_0 \dots x_n} k[\frac{1}{x_0}, \dots,
                \frac{1}{x_n}])_d$.
                We see that
                \begin{equation*}
                    \begin{split}
                        \binom{n + (-d -n -1)}{n} &= \Pi^n_{i=1} \frac{-d -n -1 + i}{i} \\
                        &= (-1)^n\ \Pi^n_{i=1} \frac{d + n + 1 - i}{i} \\
                        &= (-1)^n\ \Pi^n_{i=1} \frac{d + i}{i} \\
                        &= (-1)^n\ \binom{n + d}{n}
                    \end{split}
                \end{equation*}
                So 
                \begin{equation*}
                    \chi(\Proj^n_k, \mathcal{O}_{\Proj^n_k}(d)) = (-1)^n (-1)^n\
                    \binom{n + d}{n} = \binom{n + d}{n}
                \end{equation*}
                which is what we needed to show.
            }
        \item{
                We have an exact sequence
                \begin{equation*}
                    0 \to \mathcal{O}_{\Proj^2_k}(-d) \to
                    \mathcal{O}_{\Proj^2_k} \to i_* \mathcal{O}_X \to 0
                \end{equation*}
                where $i \colon X \to \Proj^2_k$ is a closed immersion.
                Now taking cohomology we get
                \begin{equation*}
                \begin{tikzcd}
                    &0 \arrow{r} &H^0(\Proj^2_k, \mathcal{O}_{\Proj^2_k}(-d))
                    \arrow{r} &H^0(\Proj^2_k, \mathcal{O}_{\Proj^2_k}) \arrow{r}
                    &H^0(\Proj^2_k, i_* \mathcal{O}_X) \arrow{dll} \\
                    & &H^1(\Proj^2_k, \mathcal{O}_{\Proj^2_k}(-d))
                    \arrow{r} &H^1(\Proj^2_k, \mathcal{O}_{\Proj^2_k}) \arrow{r}
                    &H^1(\Proj^2_k, i_* \mathcal{O}_X) \arrow{dll} \\
                    & &H^2(\Proj^2_k, \mathcal{O}_{\Proj^2_k}(-d))
                    \arrow{r} &H^2(\Proj^2_k, \mathcal{O}_{\Proj^2_k}) \arrow{r}
                    &H^2(\Proj^2_k, i_* \mathcal{O}_X) \arrow{r} & 0
                \end{tikzcd}
                \end{equation*}
                We know that $H^1(\Proj^2_k, \mathcal{O}_{\Proj^2_k})$ and
                $H^2(\Proj^2_k, \mathcal{O}_{\Proj^2_k})$ are both $0$.
                So $H^1(\Proj^2_k, i_* \mathcal{O}_X) = H^1(X, \mathcal{O}_X)$
                is isomorphic to $H^2(\Proj^2_k, \mathcal{O}_{\Proj^2_k}(-d))$.
                And for the latter one we know $\dim_k H^2(\Proj^2_k,
                \mathcal{O}_{\Proj^2_k}(-d)) = \binom{2 - d}{2} =
                \frac{(d-1)(d-2)}{2}$.

            }
    \end{enumerate}
\end{exercise}

\begin{exercise}{4}
    \begin{enumerate}
        \item{
                First thing to note:
                We know that colimits commute with stalks, so
                $\colim F_{i, x} \to (\colim F_i)_x$ is an isomorphism.
                
                Pick now $s \in F_i(U)$ that is in the kernel of the map
                $\Psi_U$. That means for every $x \in U$ the section $s$
                vanishes in $\colim F_{i, x} = (\colim F_i)_x$.
                So there exists $i_x$ such that $s$ vanishes in $F_{i_x, x}$.
                And then there also exists a neighbourhood $x \in U_x$ such that
                $s$ vanishes in $F_{i_x}(U_x)$.
                By quasi-compactness we cover $U$ with finitely many such $U_x$.
                Pick $j$ to be the maximal $i_x$ from the covering. Then $s$
                gets mapped to $0$ by the transition map $F_i(U) \to F_j(U)$.
                So $s$ is already equal to $0$ in $\colim_i F_i(U)$.
                (I omitted transition maps, hopefully its clear what was meant.
                Otherwise it becomes messy.)
            }
        \item{
                We again start with the observation that 
                $\colim F_{i, x} \to (\colim F_i)_x$ is an isomorphism.

                Take $s \in (\colim F_i)(U)$. For every stalk we find $i_x$,
                $U_x$ and $s_x \in F_{i_x}(U_x)$ such that the image of $s_x$ by
                $\Psi_U$ is equal to restriction of $s$ to $U_x$. By
                quasi-compactness there is finite subcover. Now we need to glue
                these $s_x \in F_{i_x}(U_x)$. By our choice, the difference $s_x
                |_{U_x \cap U_y} - s_y |_{U_x \cap U_y}$ (viewing both
                inside $\colim F_i(U_x \cap U_y)$) is in the kernel of
                $\Psi_{U_x \cap U_y}$. Because intersection $U_x \cap U_y$ is
                quasi-compact, we have by previous part that $s_x |_{U_x \cap
                U_y} = s_y |_{U_x \cap U_y}$ (as elements of $\colim F_i(U_x
                \cap U_y)$). So there exists some $j_{x, y} \geq i_x, i_y$ such
                that $s_x = s_y \in F_{j_{x, y}}(U_x \cap U_y)$. Now just take
                the maximum $j = \max j_{x, y}$ over all pairs $x, y$ and we get
                a section in $s' \in F_j(U)$ that gets mapped to $s$ by
                ($\Psi_U$).
                (Again, sorry for leaving out transition maps.)
            }
        \item{
                So we assume that the category of abelian sheaves on $X$ has
                enough injectives. Let $F_i \to G_i$ be an injective embedding
                of directed systems and $G_i$ injective sheaves.

                Let $H_i = \coker(F_i \to G_i)$ so we have an exact sequence
                \begin{equation*}
                    0 \to F_i \to G_i \to H_i \to 0
                \end{equation*}
                for every $i$.
                Taking finite limits commutes with filtered colimits, so we have
                an exact sequence
                \begin{equation*}
                    0 \to \colim F_i \to \colim G_i \to \colim H_i \to 0.
                \end{equation*}

                Lets now do induction. The base case was done in first two parts
                (spectral space is qcqs).

                So assume the statement holds for $n$.

                By injectivity $H^n(X, G_i) = 0$ for every $n > 0$ and $i \in
                I$. So $\colim H^n(X, G_i) = 0$ but only for $n > 0$.
                We get a diagram
                \begin{equation*}
                    \begin{tikzcd}
                        &H^n(X, \colim G_i) \arrow{r} \arrow{d} &H^n(X, \colim
                        H_i) \arrow{r} \arrow{d} &H^{n+1}(X, \colim F_i)
                        \arrow{d} \arrow{r} &H^{n+1}(X, \colim G_i) \arrow{d} \\
                        &\colim H^n(X, G_i) \arrow{r} &\colim H^n(X, H_i)
                        \arrow{r} &\colim H^{n+1}(X, F_i) \arrow{r} &\colim
                        H^{n+1}(X, G_i)
                    \end{tikzcd}
                \end{equation*}

                We have $\colim H^{n+1}(X, G_i) = 0$ by injectivness and because
                $n + 1 > 0$.

                Lets show that also $H^{n+1}(X, \colim G_i) = 0$.
                I dont know how to show this.

                Assuming we've shown $H^{n+1}(X, \colim G_i) = 0$, we conclude
                by five lemma that $H^{n+1}(X, \colim F_i) \cong \colim H^{n+1}(X,
                F_i)$.
            }
    \end{enumerate}
\end{exercise}


\end{document}
