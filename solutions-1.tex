\newcommand{\sheet}{1}
\documentclass{article}
\usepackage[english, german]{babel}
\usepackage{amsthm,amssymb,amsmath,mathrsfs,mathtools}
\usepackage[shortlabels]{enumitem}
\usepackage{hyperref}
\usepackage{biblatex}
\usepackage{tikz}
\usepackage{tikz-cd}

% \usepackage[tmargin=1.25in,bmargin=1.25in,lmargin=1.2in,rmargin=1.2in]{geometry}


\newcommand{\C}{\mathbb{C}}
\newcommand{\R}{\mathbb{R}}
\newcommand{\N}{\mathbb{N}}
\newcommand{\Q}{\mathbb{Q}}
\newcommand{\Z}{\mathbb{Z}}
\newcommand{\Proj}{\mathbb{P}}
\newcommand{\Aff}{\mathbb{A}}

\DeclareMathOperator{\id}{id}
\DeclareMathOperator{\im}{im}
\DeclareMathOperator{\GL}{GL}
\DeclareMathOperator{\sgn}{sgn}
\DeclareMathOperator{\Tor}{Tor}
\DeclareMathOperator{\Sym}{Sym}
\DeclareMathOperator{\coker}{coker}
\DeclareMathOperator{\Quot}{Quot}
\DeclareMathOperator{\supp}{supp}
\DeclareMathOperator{\Hom}{Hom}
\DeclareMathOperator{\Spec}{Spec}
\DeclareMathOperator{\MinSpec}{MinSpec}
\DeclareMathOperator{\MaxSpec}{MaxSpec}
\DeclareMathOperator{\diag}{diag}
\DeclareMathOperator{\BL}{BL}
\DeclareMathOperator{\Ouv}{Ouv}
\DeclareMathOperator{\Sh}{Sh}
\DeclareMathOperator{\PSh}{PSh}
\DeclareMathOperator{\Eq}{Eq}
\DeclareMathOperator{\colim}{colim}
\DeclareMathOperator{\Pic}{Pic}
\DeclareMathOperator{\CL}{CL}
\DeclareMathOperator{\eq}{eq}
\DeclareMathOperator{\codim}{codim}

\newenvironment{exercise}[1] {
  \vspace{0.5cm}
  \noindent \textbf{Exercise~{#1}.}
} {
  \vspace{0.5cm}
}
\newenvironment{claim} {
  \par\noindent\textbf{Claim.}
} { }

\newenvironment{proof_claim} {
  \par\noindent\textbf{Proof of claim.}
} {
    \qed (of claim)
}

\title{Algebraic geometry 1\\Exercise sheet \sheet}
\author{Solutions by: Eric Rudolph and David Čadež}

\date{\today}


\begin{document}

\maketitle

\begin{exercise}{1}
    \begin{enumerate}
        \item{} Closed subsets of $\mathbb{A}^1(k)$ are $\emptyset$,
            $\mathbb{A}^1(k)$ and all finite subsets.
        \item{} That is true because there exists no product of two closed subsets
            of $\mathbb{A}^1(k)$ that would cover closed subspace $V(x - y)
            \subseteq \mathbb{A}^2(k)$ and not cover the point $(x - 1, y)$.
            Basis for closed sets of the product topology are the whole space
            and finite union of lines parallel to one of the coordinates.
    \end{enumerate}
\end{exercise}

\begin{exercise}{2}
    \begin{enumerate}
        \item The polynomials from the definition exist.
        \item Discriminant is a polynomial function on coefficients of a
            polynomial.
        \item A matrix has pairwise different eigenvalues if and only if its
            characteristic polynomial has only simple roots. That is exactly
            when discriminant of its characteristic polynomial is non-zero. We
            can compose maps from $1)$ and $2)$ to get a map $h \colon V
            \rightarrow \mathbb{A}^1(k)$ that vanishes on a matrix $M$ if and
            only if $M$ is not diagonalizable with pairwise different
            eigenvalues. Since $\{0\}$ is a closed set in $\mathbb{A}^1(k)$ and
            $h$ continuous, we get that $\{ M \in V \mid h(M) \not= 0 \}$ is
            open in $V$.
    \end{enumerate}
\end{exercise}

\begin{exercise}{3}
    Using the hint we look at linear transormations that would simplify
    polynomial. Write $x \mapsto ux + vy$ and $y \mapsto wx + zy$. We get that
    if $a^2_2 - 4 a_1 a_3 = 0$, then we can pick $u = \sqrt{a_1}$ and $v =
    \sqrt{a_3}$ and $x^2 \mapsto a_1 x^2 + a_2 xy + a_3 y^2$. Otherwise we can
    define $u = 1$, $w = a_1$, $v$ is the solution to $v(a_2 - v a_1) = a_3$ and
    $z = a_2 - v a_1$.

    From now on $a_4$, $a_5$ and $a_6$ are not the same as in the original
    polynomial.

    So if $f(x, y) = x^2 + a_4 x + a_5 y + a_6$, then $V$ is either
    \begin{itemize}
        \item if $a_5 \not= 0$; $V$ is isomorphic to a parabola
        \item if $a_5 = 0$ and $a^2_4 - 4 a_6 \not= 0$; $V$ is isomorphic to disjoint union of two lines
        \item if $a_5 = 0$ and $a^2_4 - 4 a_6 = 0$; $V$ is isomorphic to a single line
    \end{itemize}

    If $f(x, y) = xy + a_4 x + a_5 y + a_6$, then $V$ is either isomorphic to a
    hyperbola or a union of coordinate lines. We can write $f$ in the form $(x +
    u) (y + v) - z$ for some suitable $u, v, z \in k$. Explicitly we get $u =
    a_4$, $v = a_5$ and $z = uv - a_6$. So if $a_4 a_5 = a_6$, then $z = 0$ and
    $V$ is isomorphic to the union of coordinate lines, otherwise $V$ is
    isomorphic to a hyperbola. But again note that these $a_4, a_5, a_6$ are not
    the ones in original polynomial, because we used a linear transormation
    earlier.
\end{exercise}

\begin{exercise}{4}
    Lets look at coordinate rings:
    \begin{enumerate}
        \item $\mathcal{O}_{\{y - x^2 = 0\}} = k[x, y] / (y - x^2) = k[x]$
        \item $\mathcal{O}_{\{xy - 1 = 0\}} = k[x, y] / (xy - 1) = k[x, x^{-1}]$
        \item $\mathcal{O}_{\{xy = 0\}} = k[x, y] / (xy)$
        \item $\mathcal{O}_{\{x(x - 1) = 0\}} = k[x, y] / (x(x - 1))$
        \item $\mathcal{O}_{\{x = 0\}} = k[x, y] / (x) = k[y]$
    \end{enumerate}
    The $1$st and $5$th are clearly isomorphic. Number $2$ is not isomorphic to
    any else, because it has strictly more invertible elements than just the
    field $k$. Number $3$ and $4$ are only ones with zero-divisors, so they
    could only be isomorphic to each other. But they are not, because number $4$
    has an idempotent element (other than $0$ and $1$) and $3$ doesn't.
\end{exercise}

\end{document}
