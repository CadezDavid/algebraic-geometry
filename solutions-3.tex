\newcommand{\sheet}{3}
\documentclass{article}
\usepackage[english, german]{babel}
\usepackage{amsthm,amssymb,amsmath,mathrsfs,mathtools}
\usepackage[shortlabels]{enumitem}
\usepackage{hyperref}
\usepackage{biblatex}
\usepackage{tikz}
\usepackage{tikz-cd}

% \usepackage[tmargin=1.25in,bmargin=1.25in,lmargin=1.2in,rmargin=1.2in]{geometry}


\newcommand{\C}{\mathbb{C}}
\newcommand{\R}{\mathbb{R}}
\newcommand{\N}{\mathbb{N}}
\newcommand{\Q}{\mathbb{Q}}
\newcommand{\Z}{\mathbb{Z}}
\newcommand{\Proj}{\mathbb{P}}
\newcommand{\Aff}{\mathbb{A}}

\DeclareMathOperator{\id}{id}
\DeclareMathOperator{\im}{im}
\DeclareMathOperator{\GL}{GL}
\DeclareMathOperator{\sgn}{sgn}
\DeclareMathOperator{\Tor}{Tor}
\DeclareMathOperator{\Sym}{Sym}
\DeclareMathOperator{\coker}{coker}
\DeclareMathOperator{\Quot}{Quot}
\DeclareMathOperator{\supp}{supp}
\DeclareMathOperator{\Hom}{Hom}
\DeclareMathOperator{\Spec}{Spec}
\DeclareMathOperator{\MinSpec}{MinSpec}
\DeclareMathOperator{\MaxSpec}{MaxSpec}
\DeclareMathOperator{\diag}{diag}
\DeclareMathOperator{\BL}{BL}
\DeclareMathOperator{\Ouv}{Ouv}
\DeclareMathOperator{\Sh}{Sh}
\DeclareMathOperator{\PSh}{PSh}
\DeclareMathOperator{\Eq}{Eq}
\DeclareMathOperator{\colim}{colim}
\DeclareMathOperator{\Pic}{Pic}
\DeclareMathOperator{\CL}{CL}
\DeclareMathOperator{\eq}{eq}
\DeclareMathOperator{\codim}{codim}

\newenvironment{exercise}[1] {
  \vspace{0.5cm}
  \noindent \textbf{Exercise~{#1}.}
} {
  \vspace{0.5cm}
}
\newenvironment{claim} {
  \par\noindent\textbf{Claim.}
} { }

\newenvironment{proof_claim} {
  \par\noindent\textbf{Proof of claim.}
} {
    \qed (of claim)
}

\title{Algebraic geometry 1\\Exercise sheet \sheet}
\author{Solutions by: Eric Rudolph and David Čadež}

\date{\today}


\begin{document}

\maketitle{}

\begin{exercise}{1}
    \begin{enumerate}
        \item Define 
        \begin{align*}
            \pi^{-1}:U &\longrightarrow \pi^{-1}(U) \\
            (x_1,...,x_n)  &\mapsto (x_1,...,x_n)[x_1:...:x_n].
        \end{align*}
        This is well-defined, because by definition of $U$, not all
        $x_i$ can be zero at the same time, so $[x_1:...:x_n]$ is actually a
        point in projective space. We also have 
            $(x_1,...,x_n)[x_1:...:x_n]\in Z$ for $(x_1,...,x_n)\in U$,
            because $x_ix_j=x_jx_i$ for all $1\leq i,j \leq n.$
            To see injectivity of $\pi^{-1}$, let
            $(x_1,...,x_n)\in U$ with $x_j\not = 0$. Then we have $y_j\not = 0$, because
            if we assume $x_j\not =0$ and $y_j=0$, then for some $y_i\not =0$ 
            (which exists since $[y_1:...:y_n]$ is a point in projective space) 
            we have $0\not = x_jy_i= x_iy_j=0.$
            Therefore, we can just set $y_j=1$. Then 
            \begin{align*}
                x_iy_j=x_jy_i \implies y_1=\frac{x_1y_j}{x_j}=\frac{x_1}{x_j},
            \end{align*}
            showing that all the $y_i$ are fixed up to a scalar after fixing all the $x_i$.
    \end{enumerate}
\end{exercise}

\begin{exercise}{4}
    \begin{enumerate}
        \item{
                Lets first prove that $V_{U}$ are stable under
                intersections:
                \begin{claim}
                    Take $U, W \subseteq X$ open subsets. Then $V_{U \cap W} =
                    V_U \cap V_W$.
                \end{claim}
                \begin{proof_claim}
                    Inclusion $V_{U \cap W} \subseteq V_U \cap
                    V_W$ is clear.

                    For the other inclusion take $Z \in V_U \cap V_W$. By
                    definition $Z \cap U \not= \emptyset$ and $Z \cap V \not=
                    \emptyset$. Suppose $Z \cap (U \cap V) = \emptyset$. Then
                    $(Z \cap U)^c \cup (Z \cap V)^c = X$. But since $Z$ is
                    irreducible, and is covered by $U^c \cup V^c$, we must have
                    (WLOG) $Z \subseteq U^c$. That is in contradiction with
                    $Z \cap U \not= \emptyset$. 
                \end{proof_claim}

                It also behaves well under unions:
                \begin{align*}
                    V_{U \cup W} &= \{ Z\ \text{cl. irred.} \mid Z \cap (U \cup
                    W) \not= \emptyset \} \\
                    &= \{ Z\ \text{cl. irred.} \mid (Z \cap U) \not= \emptyset
                    \ \text{or}\ (Z \cap W) \not= \emptyset \} \\
                    &= \{ Z\ \text{cl. irred.} \mid (Z \cap U) \not= \emptyset
                    \} \cup \{ Z\ \text{cl. irred.} \mid (Z \cap W) \not=
                    \emptyset \} \\
                    &= V_U \cup V_W
                \end{align*}
                and practically same argument applies to infinite unions.

                Therefore every open subset of $X^{\text{sob}}$ can be written
                as $V_U$ for some open $U \subseteq X$.

                \begin{claim}
                    Closed irreducible subsets of $X^{\text{sob}}$ are exactly
                    $V^c_U$ for open $U \subseteq X$ such
                    that (closed) subset $U^c \subseteq X$ is
                    irreducible.
                \end{claim}
                \begin{proof_claim}
                    Take $V^c_U$ such that $U^c$ is not irreducible. Then there
                    exist closed subsets $U^c_1, U^c_2 \subseteq X$ with $U^c =
                    U^c_1 \cup U^c_2$ meanwhile $U^c \not= U^c_1$ and $U^c \not=
                    U^c_2$. Then $V_U = V_{U_1 \cup U_2} = V_{U_1} \cup V_{U_2}$
                    and we can thus cover $V_U$ with $V_{U_1}$ and $V_{U_2}$ but
                    $V_U \not= V_{U_1}$ and $V_U \not= V_{U_2}$.
                \end{proof_claim}

                Let us show $X^{\text{sob}}$ is sober. Let $V_U^c$ be closed
                irreducible. Then by last claim $U^c$ is closed and irreducible.
                This $U^c$ will be the generic point with $\overline{\{U^c\}} =
                V_U^c$.
            }
    \end{enumerate}
\end{exercise}

\end{document}
