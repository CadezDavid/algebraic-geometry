\newcommand{\sheet}{1}
\documentclass{article}
\usepackage[english, german]{babel}
\usepackage{amsthm,amssymb,amsmath,mathrsfs,mathtools}
\usepackage[shortlabels]{enumitem}
\usepackage{tikz}
\usepackage{tikz-cd}

% \usepackage[tmargin=1.25in,bmargin=1.25in,lmargin=1.2in,rmargin=1.2in]{geometry}


\newcommand{\C}{\mathbb{C}}
\newcommand{\R}{\mathbb{R}}
\newcommand{\N}{\mathbb{N}}
\newcommand{\Q}{\mathbb{Q}}
\newcommand{\Z}{\mathbb{Z}}

\DeclareMathOperator{\id}{id}
\DeclareMathOperator{\im}{im}
\DeclareMathOperator{\GL}{GL}
\DeclareMathOperator{\sgn}{sgn}
\DeclareMathOperator{\Tor}{Tor}
\DeclareMathOperator{\Sym}{Sym}
\DeclareMathOperator{\coker}{coker}
\DeclareMathOperator{\Quot}{Quot}
\DeclareMathOperator{\supp}{supp}
\DeclareMathOperator{\Hom}{Hom}
\DeclareMathOperator{\Spec}{Spec}
\DeclareMathOperator{\MinSpec}{MinSpec}
\DeclareMathOperator{\diag}{diag}


\newenvironment{exercise}[1] {
  \vspace{0.5cm}
  \noindent \textbf{Exercise~{#1}.}
} {
  \vspace{0.5cm}
}
\newenvironment{claim} {
  \noindent \textbf{Claim.}
} {
}

\title{Algebraic geometry 1\\Exercise sheet \sheet}
\author{Solutions by: Eric Rudolph and David Čadež}

\date{\today}


\begin{document}

\maketitle{}

\begin{exercise}{1}
    Lets first prove that if $M$ is not torsion free it cant be flat. Take $r
    \in R$ and $m \in M$ such that $r m = 0$ and $r \not= 0 \not= m$.
    We have an exact sequence of $R$-modules
    \begin{equation*}
        0 \to (r) \to R \to R / (r) \to 0
    \end{equation*}
    but when tensoring with $M$ we get
    \begin{equation*}
        0 \to (r) \otimes_R M \to R \otimes_R M \to R / (r) \otimes_R M \to 0
    \end{equation*}
    which is not exact, because $(r) \otimes_R M \to R \otimes_R M \cong M$ is
    not injective (it maps $r \otimes m \mapsto 0$).

    For the other direction, take $m$ a maximal ideal of $R$. Since $R$ is a
    Dedeking domain, $R_m$ is also normal and thus a PID (we proved that last
    year during the lectures). We've shown a module over a PID is torsion-free
    exactly when it is flat in Algebra 1.
    We did it by showing that flatness can be checked on all finitely generated
    submodules and that a finitely generated module over a PID is flat if and
    only if it is free.
\end{exercise}

\begin{exercise}{2}
    Map $\Spec(A) \to \Spec(R)$ sends generic point to
    generic point if and only if $R \to A$ injective.
    
    And clearly $A$ is a torsion free $R$ module if and only if $R \to A$ is
    injective.

    Using first exercise we get that $R \to A$ is flat if and only if $\Spec(A)
    \to \Spec(R)$ send generic point to generic point.
\end{exercise}

\begin{exercise}{3}
    % For reference i used "V2B3: Einführung in die Komplexe Analysis" by Pavel
    % Zorin-Kranich he had in sommersemester 2020
    \begin{enumerate}[i)]
        \item{
                The derivative of $z \mapsto z g(z)$ at $z$ is $g(z) + z
                \frac{dg}{dz}(z)$, which is $g(0)$ at $0$, therefore non-zero.
                So by theorem from complex analysis there exists a holomorphic
                inverse on some neighborhood of $0$.

                Second part: from complex analysis we know that if $g$ is a
                holomorphic function on a simply connected open $\Omega$ with $g \not= 0$
                on $\Omega$, then there exists $\tilde{g}$ on $\Omega$ with
                $e^{\tilde{g}} = g$.
                So for $h$ we can take $e^{\frac{1}{n} \tilde{g}}$.
            }
        \item{
                Pick $y \in Y$ and $V \subseteq X$ a neighborhood of $f(y) \in
                X$ with $V \cong \D$ (WLOG with $f(y)$ corresponding to $0$).
                Take $U \subseteq f^{-1}(V)$ with $y \in U \cong \D$ (WLOG with
                $y$ corresponding to $0$).
                Because zero set of a non-zero holomorphic map is discrete, we
                can pick $U$ such that $y$ is the only zero of the function $U
                \to V \to \D$.
                So now we have holomorphic $h \colon \D \cong U \to V \cong \D$, for
                which $0 \mapsto 0$. Let $n_y$ be the degree of this root. 
                Therefore we can write $h(z) = z^{n_y} g(z)$ for some
                holomorphic $g \colon \D \to \D$. Observe that since $0$ is the
                only root, we have $g(z) \not= 0$ for all $z \in \D$.
                By part i) we have that there exists $n$-th root of $g$, i.e. a
                holomorphic function $p$ with $p^{n_y} = g$ on $\D$.
                Note that since $g \not= 0$ on $\D$, same is true for $p$.
                We can write $h(z) = z^{n_y} p^{n_y}(z)$.
                By part i), the function $z \mapsto z p(z)$ is biholomorphic, so
                it has a holomorphic inverse. Precomposing $h$ with this inverse
                yields a function $\tilde{h} \colon \D \to \D$ with $z \mapsto z^{n_y}$.
            }
        \item{
                Suppose we have two local descriptions with $U_1$ and $U_2$,
                which have non-empty open intersecton in $Y$. We can assume both
                map to same $V \cong \D$. We obtain a local neighborhood of $0$
                in $U_1 \cap U_2$ that is biholomorphic to its image in $U_1
                \cap U_2 \rightharpoonup U_2$. Since this change of coordinates
                is biholomorphic, degree of the root has to be $1$, so $n_1$ and
                $n_2$ corresponding with descriptions with $U_1$ and $U_2$ are
                the same as well.

                For every point $y \in Y$ we found a neighborhood $U$ on which
                it identifies with $z \to z^{n_y}$.
                From this local identification it follows that for every other
                point $z \in U$ with $z \not= y$ there exists a neighborhood
                $\tilde{U} \subseteq U$ for which $\tilde{U} \xrightarrow{\sim}
                f(\tilde{U})$.
                So for every point in $U \setminus \{y\}$ the map $f$ is locally
                biholomorphic, which means $n_z = 1$.
                Therefore the set of points $y$ with $n_y > 1$ is discrete.
                Because manifold $Y$ is compact, that set must be finite.
            }
    \end{enumerate}
\end{exercise}

\begin{exercise}{4}
    \begin{enumerate}[i)]
        \item{
                Functor $\Hom_A(-, I)$ is always left-exact, so we only have to
                check right-exactness.

                Let
                \begin{equation*}
                    0 \to M_1 \to M_2 \to M_3 \to 0
                \end{equation*}
                be exact.
                We want to show
                \begin{equation*}
                    0 \to \Hom_A(M_3, I) \to \Hom_A(M_2, I) \to \Hom_A(M_1, I)
                    \to 0
                \end{equation*}
                is exact.

                There is a natural isomorphism of abelian groups
                \begin{equation*}
                    \Hom_A(M_i, I) \cong \Hom_\Z(M_i \otimes_A A, \Q / \Z) \cong
                    \Hom_\Z(M_i, \Q / \Z).
                \end{equation*}
                (here by natural we mean functorial, i.e. that
                \begin{equation*}
                \begin{tikzcd}
                    & 0 \arrow{r} & \Hom_A(M_3, I) \arrow{d} \arrow{r} &
                    \Hom_A(M_2, I) \arrow{d} \arrow{r} & \Hom_A(M_1, I)
                    \arrow{d} \arrow{r} & 0 \\
                    & 0 \arrow{r} & \Hom_\Z(M_3, \Q / \Z) \arrow{r} &
                    \Hom_\Z(M_2, \Q / \Z) \arrow{r} & \Hom_\Z(M_1, \Q / \Z)
                    \arrow{r} & 0
                \end{tikzcd}
                \end{equation*}
                commutes. We had some confusion around the meaning of
                naturality, functoriality and something being canonical.)

                In the hint it says that $\Q / \Z$ is injective $\Z$-module, so
                we get that
                \begin{equation*}
                    0 \to \Hom_\Z(M_3, \Q / \Z) \to \Hom_\Z(M_2, \Q / \Z) \to
                    \Hom_\Z(M_1, \Q / \Z) \to 0
                \end{equation*}
                is exact.
                Therefore
                \begin{equation*}
                    0 \to \Hom_A(M_3, I) \to \Hom_A(M_2, I) \to \Hom_A(M_1, I)
                    \to 0
                \end{equation*}
                is also exact.
            }
        \item{
                There is a (forgetful) faithful functor from category of $A$-modules to
                category of abelian groups, which preserves monomorphisms and
                has a right adjoint. Then it is true that if the latter category
                has enough injectives than also former has enough injectives.
            }
    \end{enumerate}
\end{exercise}

\end{document}
