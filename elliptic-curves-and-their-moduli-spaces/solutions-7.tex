\newcommand{\sheet}{7}
\documentclass{article}
\usepackage[english, german]{babel}
\usepackage{amsthm,amssymb,amsmath,mathrsfs,mathtools}
\usepackage[shortlabels]{enumitem}
\usepackage{hyperref}
\usepackage{biblatex}
\usepackage{tikz}
\usepackage{tikz-cd}

% \usepackage[tmargin=1.25in,bmargin=1.25in,lmargin=1.2in,rmargin=1.2in]{geometry}


\newcommand{\C}{\mathbb{C}}
\newcommand{\R}{\mathbb{R}}
\newcommand{\N}{\mathbb{N}}
\newcommand{\Q}{\mathbb{Q}}
\newcommand{\Z}{\mathbb{Z}}
\newcommand{\Proj}{\mathbb{P}}
\newcommand{\Aff}{\mathbb{A}}

\DeclareMathOperator{\id}{id}
\DeclareMathOperator{\im}{im}
\DeclareMathOperator{\GL}{GL}
\DeclareMathOperator{\sgn}{sgn}
\DeclareMathOperator{\Tor}{Tor}
\DeclareMathOperator{\Sym}{Sym}
\DeclareMathOperator{\coker}{coker}
\DeclareMathOperator{\Quot}{Quot}
\DeclareMathOperator{\supp}{supp}
\DeclareMathOperator{\Hom}{Hom}
\DeclareMathOperator{\Spec}{Spec}
\DeclareMathOperator{\MinSpec}{MinSpec}
\DeclareMathOperator{\MaxSpec}{MaxSpec}
\DeclareMathOperator{\diag}{diag}
\DeclareMathOperator{\BL}{BL}
\DeclareMathOperator{\Ouv}{Ouv}
\DeclareMathOperator{\Sh}{Sh}
\DeclareMathOperator{\PSh}{PSh}
\DeclareMathOperator{\Eq}{Eq}
\DeclareMathOperator{\colim}{colim}
\DeclareMathOperator{\Pic}{Pic}
\DeclareMathOperator{\CL}{CL}
\DeclareMathOperator{\eq}{eq}
\DeclareMathOperator{\codim}{codim}

\newenvironment{exercise}[1] {
  \vspace{0.5cm}
  \noindent \textbf{Exercise~{#1}.}
} {
  \vspace{0.5cm}
}
\newenvironment{claim} {
  \par\noindent\textbf{Claim.}
} { }

\newenvironment{proof_claim} {
  \par\noindent\textbf{Proof of claim.}
} {
    \qed (of claim)
}

\title{Algebraic geometry 1\\Exercise sheet \sheet}
\author{Solutions by: Eric Rudolph and David Čadež}

\date{\today}


\begin{document}

\maketitle{}

\begin{exercise}{1}
    \begin{enumerate}
        \item{
                Cover $X$ with $D(x)$ and $D(y)$. On these the scheme is
                $\Spec(k[y, z] / z^2)$ and $\Spec(k[x, z] / z^2)$ respectively.
                
                First we show that the ideal of nilpotent elements in both of
                them is the principal ideal generated by $z$. I think we've
                shown during algebra 1 that a polynomial is nilpotent if and
                only if all of its coefficients are nilpotent. It is clear that nilpotent
                elements in $k[z] / z^2$ are $(z)$. Write
                $k[y, z] / z^2 = (k[z] / z^2)[y]$, we see that nilpotent
                elements of $(k[z] / z^2)[y]$ are polynomials where all
                coefficients are divisible by $z$. That shows $\mathcal{N}$ on
                $\Spec(k[y, z] / z^2)$ is given by $(z)$.
                The situation on $\Spec(k[x, z] / z^2)$ is exactly the same.

                Now $\mathcal{N}$ clearly has a natural structure of
                $\mathcal{O}_{X_{red}}$-module, because $\mathcal{N}^2 = 0$.
                So multiplication with $\mathcal{O}_X$ ``factors through''
                multiplication with $\mathcal{O}_{X_{red}}$.
            }
        \item{
            }
    \end{enumerate}
\end{exercise}

\begin{exercise}{2}
    \begin{enumerate}
        \item{
                For the start assume all schemes involved are affine.

                Let $X = \Spec A, S = \Spec R, T = \Spec R', Z = \Spec(A /
                I)$.

                Then we have a diagram
                \begin{equation*}
                    \begin{tikzcd}
                        &\Spec(A / I \otimes_R R') \arrow{r} \arrow{d}
                        &\Spec(A / I) \arrow{d} \\
                        &\Spec(A \otimes_R R') \arrow{r} \arrow{d} &\Spec(A)
                        \arrow{d} \\
                        &\Spec(R') \arrow{r} &\Spec(R)
                    \end{tikzcd}
                \end{equation*}

                In this case the pullback is
                \begin{equation*}
                    f^* \mathcal{I} = (I \otimes_R R')^{\widetilde{}}
                \end{equation*}
                and the base change is the kernel of surjection $A \otimes_R R'
                \to A / I \otimes_R R'$, i.e.
                \begin{equation*}
                    \mathcal{I}_{R'} = (\ker(A \otimes_R R' \to A / I
                    \otimes_R R'))^{\widetilde{}}.
                \end{equation*}

                If $R \to R'$ is flat, then
                \begin{equation*}
                    0 \to I \otimes_R R' \to A \otimes_R R' \to A / I \otimes_R
                    R' \to 0
                \end{equation*}
                is exact, which shows that in this case the map
                \begin{equation*}
                    I \otimes_R R' \to \ker(A \otimes_R R' \to A / I \otimes_R R')
                \end{equation*}
                is an isomorphism.

                Because we already have a natural map $f^* \mathcal{I} \to
                \mathcal{I}_T$, it is enough to check that is is an isomorphism
                on an affine open cover, which reduces it to affine case.
            }
        \item{
                If a non zero-divisor $a \in A$, we have an isomorphism $A \cong
                (a) = I$ as $A$-modules.

                Assume $\mathcal{I}$ is line bundle, given by $I$ on $\Spec(A)
                \subseteq X$. If there is an isomorphism $A \to I$, then $I$ is
                a principal ideal generated by the image of $1$ under
                isomorphism above.
            }
        \item{
                Let $Z$ be an effective Cartier divisor and flat over $S$.

                Pick affine local covers as in part a).

                Suppose $\mathcal{I}$ is, locally on $\Spec(A) \subseteq X$,
                defined by $a \in A$. By flatness we have an isomorphism $f^*
                \mathcal{I} \to \mathcal{I}_T$, so $\mathcal{I}_T$ is isomorphic
                to $(a) \otimes_R R'$, which is a principal ideal of $A
                \otimes_R R'$ defined by $(a) \otimes 1$.
            }
    \end{enumerate}
\end{exercise}


\end{document}
