\newcommand{\sheet}{7}
\documentclass{article}
\usepackage[english, german]{babel}
\usepackage{amsthm,amssymb,amsmath,mathrsfs,mathtools}
\usepackage[shortlabels]{enumitem}
\usepackage{tikz}
\usepackage{tikz-cd}

% \usepackage[tmargin=1.25in,bmargin=1.25in,lmargin=1.2in,rmargin=1.2in]{geometry}


\newcommand{\C}{\mathbb{C}}
\newcommand{\R}{\mathbb{R}}
\newcommand{\N}{\mathbb{N}}
\newcommand{\Q}{\mathbb{Q}}
\newcommand{\Z}{\mathbb{Z}}

\DeclareMathOperator{\id}{id}
\DeclareMathOperator{\im}{im}
\DeclareMathOperator{\GL}{GL}
\DeclareMathOperator{\sgn}{sgn}
\DeclareMathOperator{\Tor}{Tor}
\DeclareMathOperator{\Sym}{Sym}
\DeclareMathOperator{\coker}{coker}
\DeclareMathOperator{\Quot}{Quot}
\DeclareMathOperator{\supp}{supp}
\DeclareMathOperator{\Hom}{Hom}
\DeclareMathOperator{\Spec}{Spec}
\DeclareMathOperator{\MinSpec}{MinSpec}
\DeclareMathOperator{\diag}{diag}


\newenvironment{exercise}[1] {
  \vspace{0.5cm}
  \noindent \textbf{Exercise~{#1}.}
} {
  \vspace{0.5cm}
}
\newenvironment{claim} {
  \noindent \textbf{Claim.}
} {
}

\title{Algebraic geometry 1\\Exercise sheet \sheet}
\author{Solutions by: Eric Rudolph and David Čadež}

\date{\today}


\begin{document}

\maketitle{}

\begin{exercise}{1}
    \begin{enumerate}
        \item We have the following bijection
        \begin{align*}
            \Hom_{\mathcal{O}_x}(\widetilde{\mathcal{N}_{|A}},f_*\widetilde{\mathcal{N}})\cong \Hom_A (\widetilde{\mathcal{N}_{|A}}(B),f_*\widetilde{\mathcal{N}}(B))\\
            =\Hom_A(N_{|A},\widetilde{\mathcal{N}}(A))\cong \Hom_{\mathcal{O}_x}(\widetilde{\mathcal{N}_{|A}},\widetilde{\mathcal{N}_{|A}}).
        \end{align*}
        By the Yoneda lemma, this implies that $f_*\widetilde{N}\cong \tilde{N}_{|A}$.
        \item For the second part of this exercise, we extend the first part as follows, using that $f_*$ is left-adjoint to $f^*$
        \begin{align*}
            \Hom_{\mathcal{O}_y}(f^*\tilde{\mathcal{M}},\tilde{N})
            \cong \Hom_{\mathcal{O}_x}(\widetilde{\mathcal{M}},f_*\widetilde{\mathcal{N}})
            \cong \Hom_A (\widetilde{\mathcal{M}}(B),f_*\widetilde{\mathcal{N}}(B))\\
            = \Hom_A(M,\widetilde{\mathcal{N}}(A))
            \cong \Hom_B(N_{|A}\otimes_A B, \tilde{N}(A))
            \cong \Hom_{\mathcal{O}_y}(\widetilde{\mathcal{M}\otimes_A B}, \widetilde{\mathcal{N}}).
        \end{align*} 
        Now, by the Yoneda lemma we again obtain that 
        \begin{align*}
            \widetilde{\mathcal{M}\otimes_A B} 
            \cong f^*\tilde{\mathcal{M}}.
        \end{align*}
        Next, we want to show that we can extend this exercise from affine schemes to schemes.
        
        Let $S_i$ with $i\in I$ be a cover of $S$ by open affines. Then for each $i\in I$ we get that
        $g^{-1}(S_i)$ is a subscheme of $Z_i\subset Z$ (unfortunately not necessarilary affine). Now, we cover
        each of these subschemes $Z_i$ by open affines $Z_{ij}$. By construction $g$ maps $Z_{ij}$ into $S_i$.
        Hence, 
        \begin{align*}
            (g^*\mathcal{M})_{Z_{ij}}=f^*\mathcal{M}_{Z_{ij}}\cong \widetilde{M\otimes_A B},
        \end{align*}
        showing that $g^*$ preserves quasi-coherence.
    \end{enumerate}
\end{exercise}

\begin{exercise}{4}
    We don't really want to do all the explicit calculations, so we only show what we think
    is maybe the main takeaway of this exercise.

    For some polynomial $f\in \mathbb{R}[x,y]$ we have that 
    \begin{align*}
        &V(f) \times_{\Spec(\mathbb{R})} \Spec(\mathbb{C})\\
        &\cong \Spec(\mathbb{R}[x,y]/(f)) \otimes_{Spec(\mathbb{R})}\Spec(\mathbb{C})\\
        &\cong \Spec(\mathbb{R}[x,y]/(f)\times_\mathbb{R} \mathbb{C})\\
        &\cong \Spec(\mathbb{C}[x,y]/(f)).
    \end{align*}
    In the following, we take $f(x,y):=xy-1$ and $g(x,y):=x^2+y^2-1$. We know from 
    the first sheet, that 
    \begin{align*}
        \mathbb{C}[x,y]/(f)\cong \mathbb{C}[x,y]/(g),
    \end{align*}
    but one can easily check that
    \begin{align*}
        \mathbb{R}[x,y]/(f) \not \cong \mathbb{R}[x,y]/(g),
    \end{align*}
    since the left side has strictly more units than the right side.

    Therefore, this is an example showing that varieties, so in particular schemes being isomorphic 
    is not stable under base change.
\end{exercise}
\end{document}
