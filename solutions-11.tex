\newcommand{\sheet}{11}
\documentclass{article}
\usepackage[english, german]{babel}
\usepackage{amsthm,amssymb,amsmath,mathrsfs,mathtools}
\usepackage[shortlabels]{enumitem}
\usepackage{hyperref}
\usepackage{biblatex}
\usepackage{tikz}
\usepackage{tikz-cd}

% \usepackage[tmargin=1.25in,bmargin=1.25in,lmargin=1.2in,rmargin=1.2in]{geometry}


\newcommand{\C}{\mathbb{C}}
\newcommand{\R}{\mathbb{R}}
\newcommand{\N}{\mathbb{N}}
\newcommand{\Q}{\mathbb{Q}}
\newcommand{\Z}{\mathbb{Z}}
\newcommand{\Proj}{\mathbb{P}}
\newcommand{\Aff}{\mathbb{A}}

\DeclareMathOperator{\id}{id}
\DeclareMathOperator{\im}{im}
\DeclareMathOperator{\GL}{GL}
\DeclareMathOperator{\sgn}{sgn}
\DeclareMathOperator{\Tor}{Tor}
\DeclareMathOperator{\Sym}{Sym}
\DeclareMathOperator{\coker}{coker}
\DeclareMathOperator{\Quot}{Quot}
\DeclareMathOperator{\supp}{supp}
\DeclareMathOperator{\Hom}{Hom}
\DeclareMathOperator{\Spec}{Spec}
\DeclareMathOperator{\MinSpec}{MinSpec}
\DeclareMathOperator{\MaxSpec}{MaxSpec}
\DeclareMathOperator{\diag}{diag}
\DeclareMathOperator{\BL}{BL}
\DeclareMathOperator{\Ouv}{Ouv}
\DeclareMathOperator{\Sh}{Sh}
\DeclareMathOperator{\PSh}{PSh}
\DeclareMathOperator{\Eq}{Eq}
\DeclareMathOperator{\colim}{colim}
\DeclareMathOperator{\Pic}{Pic}
\DeclareMathOperator{\CL}{CL}
\DeclareMathOperator{\eq}{eq}
\DeclareMathOperator{\codim}{codim}

\newenvironment{exercise}[1] {
  \vspace{0.5cm}
  \noindent \textbf{Exercise~{#1}.}
} {
  \vspace{0.5cm}
}
\newenvironment{claim} {
  \par\noindent\textbf{Claim.}
} { }

\newenvironment{proof_claim} {
  \par\noindent\textbf{Proof of claim.}
} {
    \qed (of claim)
}

\title{Algebraic geometry 1\\Exercise sheet \sheet}
\author{Solutions by: Eric Rudolph and David Čadež}

\date{\today}


\begin{document}

\maketitle{}

\begin{exercise}{3}
    \begin{enumerate}
        \item Since $k$ is algebraically closed, the only irreducible polynomials $f\in k[x,y]$ are of degree $1$.
        
        Hence, we can write 
        \begin{align*}
            f_r=l_1  \dots  l_r,
        \end{align*}
        where $l_i\in k[x,y]$ is of degree $1$. From the assumption that $f_r$ is homogenous it follows that the $l_i$ are homogenous.

        Therefore, we can write
        \begin{align*}
            Z=V(f_r)=V(l_1 \dots  l_r)=\cup_i V(l_i)
        \end{align*}
        and since $V(l_i)$ is a line through the origin, $Z$ can be written as the finite union of lines through the origin.

        \item Assume that $r=1$, i.e. $X=V(f_1).$
        We want to show that for $x=\lbrace 0 \rbrace$ we have that$\mathcal{O}_{X,x}$ is a field, which would imply that it is regular.

        We calculate 
        \begin{align*}
            \mathcal{O}_{X,x}=\varinjlim_{x\in D(g)} k[x,y]/f_1 [g^{-1}]=\varinjlim_{x\in D(g)} k[u] [g^{-1}]=K(A),
        \end{align*}
        using that $f_1$ is homogenous of degree $1$ and where $K(A)$ is the field of fractions of $A$.

        \item By part two of this exercise, all the schemes have a singular point at the origin. I don't know why they do not have singular
        points anywhere else.
    \end{enumerate}

\end{exercise}

\end{document}
