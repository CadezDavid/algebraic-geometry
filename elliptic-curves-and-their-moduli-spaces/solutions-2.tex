\newcommand{\sheet}{2}
\documentclass{article}
\usepackage[english, german]{babel}
\usepackage{amsthm,amssymb,amsmath,mathrsfs,mathtools}
\usepackage[shortlabels]{enumitem}
\usepackage{tikz}
\usepackage{tikz-cd}

% \usepackage[tmargin=1.25in,bmargin=1.25in,lmargin=1.2in,rmargin=1.2in]{geometry}


\newcommand{\C}{\mathbb{C}}
\newcommand{\R}{\mathbb{R}}
\newcommand{\N}{\mathbb{N}}
\newcommand{\Q}{\mathbb{Q}}
\newcommand{\Z}{\mathbb{Z}}

\DeclareMathOperator{\id}{id}
\DeclareMathOperator{\im}{im}
\DeclareMathOperator{\GL}{GL}
\DeclareMathOperator{\sgn}{sgn}
\DeclareMathOperator{\Tor}{Tor}
\DeclareMathOperator{\Sym}{Sym}
\DeclareMathOperator{\coker}{coker}
\DeclareMathOperator{\Quot}{Quot}
\DeclareMathOperator{\supp}{supp}
\DeclareMathOperator{\Hom}{Hom}
\DeclareMathOperator{\Spec}{Spec}
\DeclareMathOperator{\MinSpec}{MinSpec}
\DeclareMathOperator{\diag}{diag}


\newenvironment{exercise}[1] {
  \vspace{0.5cm}
  \noindent \textbf{Exercise~{#1}.}
} {
  \vspace{0.5cm}
}
\newenvironment{claim} {
  \noindent \textbf{Claim.}
} {
}

\title{Algebraic geometry 1\\Exercise sheet \sheet}
\author{Solutions by: Eric Rudolph and David Čadež}

\date{\today}


\begin{document}

\maketitle{}

\begin{exercise}{1}
    \begin{enumerate}[a)]
        \item{
                % We first prove that $\underline{A}$ is a Zariski sheaf.

                % So to prove it is representable, it is enough to check on affine
                % schemes.

                % We show that $\Hom_k(k[x_1, \dots, x_n], B)$ is naturally
                % isomorphic to $B \otimes_k A$.

                % We can embed $A$ into $k$-algebra of endomorphisms on $k^n$.
                % Then we can 
                From algebraic geometry 1 we know that for every scheme$/k$ $T$,
                we have an isomorphism
                \begin{equation*}
                    \Hom_{Sch/k}(T, \Aff^n_k) \to \Hom_k(k[x_1, \dots, x_n],
                    \mathcal{O}_T(T)).
                \end{equation*}
                which is natural in $T$.
                After picking the basis for $A$, we have a simple identification
                \begin{equation*}
                    \begin{split}
                        \Hom_k(k[x_1, \dots, x_n], \mathcal{O}_T(T)) &\to \mathcal{O}_T(T) \otimes_k A \\
                        \varphi &\mapsto \sum_i \varphi(x_i) \otimes x_i,
                    \end{split}
                \end{equation*}
                where we use $x_i$ to denote basis as well.
                This identification is simply on the level of sets, because
                although $\mathcal{O}_T(T) \otimes_k A$ has the structure of a
                (maybe non-commutative) $k$-algebra, $\Hom_k(k[x_1, \dots, x_n], \mathcal{O}_T(T))$
                doesn't seem to have any structure (is there any? Set of maps of
                $k$-algebras is nothing more than a set, right? We're assuming it
                has to map $1$ to $1$ and be $k$-linear).

                For second part, we want some sort of criterion for when is
                $\sum_i f_i \otimes x_i$ invertible. We can represent
                $\mathcal{O}_T(T) \otimes_k A$ by embedding it into algebra of
                endomorphisms $\End(\mathcal{O}_T(T)^n$ which is isomorphic to
                matrix algebra $M_{n \times n}(\mathcal{O}_T(T))$. 
                Then we can simply calculate if the element is invertible by
                evaluating its determinant.
                In our case determinant is a polynomial in $n = \dim A$
                variables (matrix depends on $n$ coefficients, for example first
                column, all others are implicitly given by that). 
                So for the ring we take $k[x_1, \dots, x_n, \det(x_1, \dots,
                x_n)^{-1}]$.


                Maybe it would be easier to define ring from scratch and not
                look at open subschemes of $\Aff^n_k$. In which case we take
                $k[t_{i j}, i, j \in \{1, \dots, n\}]$, invert $\det(t_{ij})$
                and then quotient by relations $t_{i j} = \sum_k a^k_{i j} t_{k
                1}$ where $a^k_{i j}$ represents $k$-th coordinate in expansion
                of the product $(t_{1 1} x_1 + \dots + t_{n 1} x_n) x_j$ as $i
                \in \{1, \dots, n\}$ and $j \in \{2, \dots, n\}$.
                In example of $\R$-algebra $\C$ that would mean
                \begin{equation*}
                    \R[x_{1 1}, x_{2 1}, x_{1 2}, x_{2 2}, (x_{1 1} x_{2 2} -
                    x_{1 2} x_{2 1})^{-1}] / (x_{1 2} = - x_{2 1}, x_{2 2} =
                    x_{1 1})
                \end{equation*}
                }
        \item{
                
            }
        \item{
                Let $k = \R$, $A = \C$ and $G = \underline{A}^\times$.
                In part a) of problem 1 we argued that $\underline{A}^\times$ is
                the affine open subscheme of $\Aff^n_k$ we get by inverting the
                determinant of embedding $A \hookrightarrow \End(k^n) \cong M_{n
                \times n}(k)$. In this case (one possible) embedding is $a + b i \mapsto
                \begin{bmatrix}
                    a & -b \\
                    b & a
                \end{bmatrix}.
                $
                So we have to invert $a^2 + b^2$.
                So now we need to define a scheme morphism. Since both $G$ and
                $\mathbb{G}_{m, \R}$ are affine, we can give map of rings:
                \begin{equation*}
                    \begin{split}
                        \R[t, t^{-1}] &\to \R[x, y, (x^2 + y^2)^{-1}] \\
                        t &\mapsto x^2 + y^2
                    \end{split}
                \end{equation*}
                So if we have a map $\Spec(\R) \to G$ (i.e. a $\R$-valued point
                $z$) and postcompose it with $G \to \mathbb{G}_{m, \R}$, we get
                $\Spec(\R) \to \mathbb{G}_{m, \R}$ corresponding to $\R$-valued
                point $z \overline{z}$.
            }
    \end{enumerate}
\end{exercise}

\begin{exercise}{2}
    Let $k$ be a field and $\G_{a, k} = \Spec(k[t])$ with $a^* \colon t
    \mapsto t \otimes 1 + 1 \otimes t$.
    \begin{enumerate}[a)]
        \item{
                We want to find maps $\G_{a, k} \to \G_{a, k}$ that will respect
                operation $a$. Since $\G_{a, k}$ is affine, this is equivalent to
                finding all maps $f^* \colon k[t] \to k[t]$ such that

                \begin{tikzcd}
                    & k[t] \otimes_k k[t] & k[t] \arrow{l}{a^*} \\
                    & k[t] \otimes_k k[t] \arrow{u}{f^* \otimes f^*} & k[t]
                    \arrow{l}{a*} \arrow{u}{f^*}
                \end{tikzcd}

                commutes.
                Since we are working over schemes over $k$, these are maps of
                $k$-algebras, which means that $f^*$ is uniquely defined by its
                value at $t$.
                From commutativity we get the condition
                \begin{equation*}
                    f^*(t) \otimes 1 + 1 \otimes f^*(t) = f^*(t \otimes 1 + 1
                    \otimes t)
                \end{equation*}
                Writing $f^*(t) = \sum_i a_i t^i$ we get
                \begin{equation*}
                    \sum_i a_i t^i \otimes 1 + 1 \otimes \sum_i a_i t^i = \sum_i a_i \sum_j
                    \binom{i}{j} (t^{i-j} \otimes t^j)
                \end{equation*}
                Since $\characteristic(k) = 0$, none of the elements on right
                vanish. So by comparing terms we obtain $a_i = 0$ for $i > 1$. 
                So $f^*(t) = a_0 + a_1 t$.
                \begin{equation*}
                    (a_0 + a_1 t) \otimes 1 + 1 \otimes (a_0 + a_1 t) = a_0 +
                    a_1 (t \otimes 1 + 1 \otimes t)
                \end{equation*}
                Compare again and get that $2 a_0 = 0$, so $a_0 = 0$ and $a_1$
                can be anything. Therefore endomorphisms $\End(\G_{a, k})$ are
                parametrized by $a_1 \in k$.
            }
        \item{
                Let now $\characteristic(k) = p$.
                Same as before, but at the step when we have 
                \begin{equation*}
                    \sum_i a_i t^i \otimes 1 + 1 \otimes \sum_i a_i t^i = \sum_i a_i \sum_j
                    \binom{i}{j} (t^{i-j} \otimes t^j)
                \end{equation*}
                when $i$ is a power of $p$, all terms $\binom{i}{j}$ vanish in
                $k$, but if $i$ is not a power of $p$, then there exists $j$
                such that $\binom{i}{j}$ does not vanish, meaning $a_i$ has to
                be $0$ for $i$ not a power of $p$.
                We also get $a_0 = 0$ by comparing terms.
                So $f^*(t) = \sum_k a_{p^k} t^{p^k}$.
                We see that all endomorphisms of $\G_{a, k}$ are of this form
                and clearly all morphisms of above form in fact define
                endomorphisms of $\G_{a, k}$
            }
    \end{enumerate}
\end{exercise}

\end{document}
