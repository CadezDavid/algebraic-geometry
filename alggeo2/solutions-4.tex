\newcommand{\sheet}{4}
\documentclass{article}
\usepackage[english, german]{babel}
\usepackage{amsthm,amssymb,amsmath,mathrsfs,mathtools}
\usepackage[shortlabels]{enumitem}
\usepackage{hyperref}
\usepackage{biblatex}
\usepackage{tikz}
\usepackage{tikz-cd}

% \usepackage[tmargin=1.25in,bmargin=1.25in,lmargin=1.2in,rmargin=1.2in]{geometry}


\newcommand{\C}{\mathbb{C}}
\newcommand{\R}{\mathbb{R}}
\newcommand{\N}{\mathbb{N}}
\newcommand{\Q}{\mathbb{Q}}
\newcommand{\Z}{\mathbb{Z}}
\newcommand{\Proj}{\mathbb{P}}
\newcommand{\Aff}{\mathbb{A}}

\DeclareMathOperator{\id}{id}
\DeclareMathOperator{\im}{im}
\DeclareMathOperator{\GL}{GL}
\DeclareMathOperator{\sgn}{sgn}
\DeclareMathOperator{\Tor}{Tor}
\DeclareMathOperator{\Sym}{Sym}
\DeclareMathOperator{\coker}{coker}
\DeclareMathOperator{\Quot}{Quot}
\DeclareMathOperator{\supp}{supp}
\DeclareMathOperator{\Hom}{Hom}
\DeclareMathOperator{\Spec}{Spec}
\DeclareMathOperator{\MinSpec}{MinSpec}
\DeclareMathOperator{\MaxSpec}{MaxSpec}
\DeclareMathOperator{\diag}{diag}
\DeclareMathOperator{\BL}{BL}
\DeclareMathOperator{\Ouv}{Ouv}
\DeclareMathOperator{\Sh}{Sh}
\DeclareMathOperator{\PSh}{PSh}
\DeclareMathOperator{\Eq}{Eq}
\DeclareMathOperator{\colim}{colim}
\DeclareMathOperator{\Pic}{Pic}
\DeclareMathOperator{\CL}{CL}
\DeclareMathOperator{\eq}{eq}
\DeclareMathOperator{\codim}{codim}

\newenvironment{exercise}[1] {
  \vspace{0.5cm}
  \noindent \textbf{Exercise~{#1}.}
} {
  \vspace{0.5cm}
}
\newenvironment{claim} {
  \par\noindent\textbf{Claim.}
} { }

\newenvironment{proof_claim} {
  \par\noindent\textbf{Proof of claim.}
} {
    \qed (of claim)
}

\title{Algebraic geometry 1\\Exercise sheet \sheet}
\author{Solutions by: Eric Rudolph and David Čadež}

\date{\today}


\begin{document}

\maketitle{}

\begin{exercise}{1}
    First of all we can use that finite projective modules are locally finite
    free. So since we are searching for an open neighbourhood of a point,
    we can first localize to some neighbourhood where $N$ is finite free. (So
    assume $N = A^n$ is finite free.)

    Since $M \otimes_A k(x)$ is finite dimensional $k(x)$-vsp, we can pick a
    basis $\{b_i \otimes 1\}_{i = 1, \dots, m}$. Let $g \colon F := A^n \to M$ be defined
    by $e_i \mapsto b_i$. At $x$ we obtain an isomorphism $F \otimes_A k(x)
    \xrightarrow{\sim} M \otimes_A k(x)$.

    The composition $F \to M \to N$ is a map of free $A$-modules, so it can be
    represented by a matrix $J \in M_{n \times m}(A)$.
    At $x$ this matrix has rank $m = \dim_{k(x)} (M \otimes_A k(x))$. So there
    is a neighbourhood $U$ on which it has rank at least $m$ (here we use
    argument from the previous sheet: $U$ is taken to be the non-vanishing locus
    of determinant of some appropriate minor). On $U$, the composition $F
    \xrightarrow{J} N$ has left inverse $N \xrightarrow{I} F$ (i.e. it is
    injective).
    
    On $U$, the section of the map $M \to N$ is given by composition $N
    \xrightarrow{I} F \xrightarrow{g} M$, which is what we wanted to show.
\end{exercise}

\begin{exercise}{3}
    By the definition of formally étale, the exercise reduces to show that in a
    diagram
    \begin{equation*}
        \begin{tikzcd}
            \F_p \arrow{d} \arrow{r} & R \arrow{d} \\
            A \arrow{r}{g} & R / I,
        \end{tikzcd}
    \end{equation*}
    where $I^2 = 0$, there exists a unique lift $A \to R$.

    We can define a lift very explicitly:

    Define $(-)^p \colon R \to R$ with $x \mapsto x^p$. Since $R$ has
    characteristic $p$, this is a homomorphism. Ideal $I$ is clearly contained
    in the kernel, so it factors through the quotient: $R \to R/I \to R$. Denote
    $u \colon R/I \to R$.

    By assumption $A$ is a perfect $\F_p$-algebra, so Frobenius endomorphism is
    an automorphism.
    We claim that a composition
    \begin{equation*}
        A \xrightarrow{\text{Fr}^{-1}_A} A \xrightarrow{g} R / I \xrightarrow{u} R
    \end{equation*}
    lifts $g$.
    Indeed, for any $x = y^p \in A$, we have $(u \circ g \circ \text{Fr}_A^{-1})
    (x) = g(y)^p = g(x)$.

    Now we prove uniqueness:
    Let $\varphi, \psi$ be two lifts. Take any $x = y^p \in A$.
    Since they are lifts, we have $\varphi(y) - \psi(y) \in I$. But then
    $(\varphi(y) - \psi(y))^p = 0$ and thus also
    $\varphi(y^p) - \psi(y^p) = \varphi(x) - \psi(x) = 0$, so $\varphi = \psi$.
\end{exercise}



\end{document}
