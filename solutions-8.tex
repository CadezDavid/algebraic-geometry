\newcommand{\sheet}{8}
\documentclass{article}
\usepackage[english, german]{babel}
\usepackage{amsthm,amssymb,amsmath,mathrsfs,mathtools}
\usepackage[shortlabels]{enumitem}
\usepackage{hyperref}
\usepackage{biblatex}
\usepackage{tikz}
\usepackage{tikz-cd}

% \usepackage[tmargin=1.25in,bmargin=1.25in,lmargin=1.2in,rmargin=1.2in]{geometry}


\newcommand{\C}{\mathbb{C}}
\newcommand{\R}{\mathbb{R}}
\newcommand{\N}{\mathbb{N}}
\newcommand{\Q}{\mathbb{Q}}
\newcommand{\Z}{\mathbb{Z}}
\newcommand{\Proj}{\mathbb{P}}
\newcommand{\Aff}{\mathbb{A}}

\DeclareMathOperator{\id}{id}
\DeclareMathOperator{\im}{im}
\DeclareMathOperator{\GL}{GL}
\DeclareMathOperator{\sgn}{sgn}
\DeclareMathOperator{\Tor}{Tor}
\DeclareMathOperator{\Sym}{Sym}
\DeclareMathOperator{\coker}{coker}
\DeclareMathOperator{\Quot}{Quot}
\DeclareMathOperator{\supp}{supp}
\DeclareMathOperator{\Hom}{Hom}
\DeclareMathOperator{\Spec}{Spec}
\DeclareMathOperator{\MinSpec}{MinSpec}
\DeclareMathOperator{\MaxSpec}{MaxSpec}
\DeclareMathOperator{\diag}{diag}
\DeclareMathOperator{\BL}{BL}
\DeclareMathOperator{\Ouv}{Ouv}
\DeclareMathOperator{\Sh}{Sh}
\DeclareMathOperator{\PSh}{PSh}
\DeclareMathOperator{\Eq}{Eq}
\DeclareMathOperator{\colim}{colim}
\DeclareMathOperator{\Pic}{Pic}
\DeclareMathOperator{\CL}{CL}
\DeclareMathOperator{\eq}{eq}
\DeclareMathOperator{\codim}{codim}

\newenvironment{exercise}[1] {
  \vspace{0.5cm}
  \noindent \textbf{Exercise~{#1}.}
} {
  \vspace{0.5cm}
}
\newenvironment{claim} {
  \par\noindent\textbf{Claim.}
} { }

\newenvironment{proof_claim} {
  \par\noindent\textbf{Proof of claim.}
} {
    \qed (of claim)
}

\title{Algebraic geometry 1\\Exercise sheet \sheet}
\author{Solutions by: Eric Rudolph and David Čadež}

\date{\today}


\begin{document}

\maketitle{}

\begin{exercise}{1}
    \begin{enumerate}
        \item  Let $0\not = f \in I$ be a non-zero element. Since $A$ is a unique factorization domain,
        we can write 
        \begin{align*}
            f=u{p}_1^{a_1}\dots {p}_r^{a_r},
        \end{align*}
        where ${p}_i$ are pairwise nosn-associated primes.
        Now, 
        \begin{align*}
            I_({p}_i)=I[(I\backslash (p_i)^{-1})]=(p_i^{k_i})
        \end{align*}
        for some $k_i\leq a_i.$
        Since $I$ is finite locally free, 
        \begin{align*}
            I=(\prod_i p_i^{k_i}).
        \end{align*}
        \item 
    \end{enumerate}
\end{exercise}

\begin{exercise}{2}
    Note that for a unique factorization domain $A$ we get by Gauss that also $A[x_1,\dots, x_n]$ is a unique factorization domain.
    This means that by construction of $\mathbb{P}_A^n$ its local rings are UFD's. Using \href{https://stacks.math.columbia.edu/tag/0BE9}{stacks project},
    we infer that $\Pic(\mathbb{P}_A^n)\cong \CL(\mathbb{P}_A^n)=\mathbb{Z}$.

    We now want to give a concrete argument using the given map.

    Note that by definion $\mathcal{O}^n_A(0)$ is just the structure sheaf and since maps of groups send $1$ to $1$, we found
    the neutral element of this group. One can also check locally that 
    \begin{align*}
       O_{\mathcal{P}^n_A}(m)\otimes_{O_{\mathcal{P}^n_A}} O_{\mathcal{P}^n_A}(n) = O_{\mathcal{P}^n_A}(m+n).
    \end{align*}
    This also proves that the given map maps to $\Pic(\mathcal{P}^n_A)$.

    It is also quite clear by definition that for $m\not=n$ we have
    \begin{align}
        O_{\mathcal{P}^n_A}(m)\not \cong O_{\mathcal{P}^n_A}(n).
    \end{align}
    It remains to show surjectivity of this map.
\end{exercise}

\begin{exercise}{3}
    \begin{enumerate}
        \item{In exercise $2$ we showed that all invertible quasicoherent
            sheaves on $\Proj^n_k$ are isomorphic to
            $\mathcal{O}_{\Proj^n_k}(d)$ for some $d \geq 0$. So we have to show
            $f^* \mathcal{O}_{\Proj^m_k}(1)$ is an invertible sheaf.

            Since invertible $\mathcal{O}_{\Proj^n_k}$-modules are same as line
            bundles, we have to show that locally $f^*
            \mathcal{O}_{\Proj^m_k}(1)$ is isomorphic to the structure sheaf
            $\mathcal{O}_{\Proj^m_k}$.

            By definition $f^* \mathcal{O}_{\Proj^m_k}(1) = f^{-1}
            \mathcal{O}_{\Proj^m_k}(1) \otimes_{f^{-1} \mathcal{O}_{\Proj^m_k}}
            \mathcal{O}_{\Proj^n_k}$. Pick some $x \in \Proj^n_k$. Pick small
            enough affine neighborhood $f(x) \in U \subseteq \Proj^m_k$ such that
            $\mathcal{O}_{\Proj^m_k} (1)$ is isomorphic to the structure sheaf
            $\mathcal{O}_{\Proj^m_k}$ on $U$. Now pick neighborhood $x \in W \subseteq \Proj^m_k$
            such that $f(W) \subseteq U$.

            Then 
            \begin{align*}
                f^{-1} \mathcal{O}_{\Proj^m_k}(1) (W) &= \colim_{f(W) \subseteq
                V} \mathcal{O}_{\Proj^m_k} (1) (V) \\
                &= \colim_{f(W) \subseteq V \subseteq U} \mathcal{O}_{\Proj^m_k}
                (1) (V) \\
                &\cong \colim_{f(W) \subseteq V \subseteq U} \mathcal{O}_{\Proj^m_k}
                (V) \\
                &\cong f^{-1} \mathcal{O}_{\Proj^m_k} (W).
            \end{align*}
            So locally $f^{-1} \mathcal{O}_{\Proj^m_k}(1)$ is isomorphic to
            $f^{-1} \mathcal{O}_{\Proj^m_k}$, so $f^{-1}
            \mathcal{O}_{\Proj^m_k}(1) \otimes_{f^{-1} \mathcal{O}_{\Proj^m_k}}
            \mathcal{O}_{\Proj^n_k}$ is locally isomorphic to
            $\mathcal{O}_{\Proj^n_k}$, which proves that $f^*
            \mathcal{O}_{\Proj^m_k}(1)$ is an invertible
            $\mathcal{O}_{\Proj^n_k}$-module and thus isomorphic to
            $\mathcal{O}_{\Proj^n_k}(d)$ for some $d \geq 0$.
            }
        \item{The polynomials $y_0, \dots, y_n$ generate the module of
            homogenous polynomials of degree $1$.
            }
    \end{enumerate}
\end{exercise}

\begin{exercise}{4}
    \begin{enumerate}
        \item{
                Let $U_i = \Spec(A_i)$.

                Take a point $x \in U_1 \cap U_2$.

                Take a principal open $x \in D(f) \subseteq U_1$ ($f \in U_1$).
                Then find a smaller principal open $x \in D(g) \subseteq U
                \subseteq U_2$ ($g \in U_2$).

                Now we show that $D(g)$ is also a principal open in $U_1$.

                Since $D(f) \subseteq U_2$, we have a map $\mathcal{O}(U_2) \to
                \mathcal{O}(D(f))$, which induces $A_2 \to (A_1)_f$. Denote by
                $g' = g |_{\Spec((A_1)_f)}$ the image of $g$ under this map.
                Since $g' \in (A_1)_f$, we can write it as $g' = \frac{h}{f^n}$.
                Then $D(g) = D(g) \cap D(f) = D(g') \cap D(f) = D(h) \cap D(f) =
                D(hf)$, where $h, f \in A_1$. This shows that $D(g)$ is also
                principal open in $U_1$.
            }
        \item{
                We have to show that the property of being of finite
                presentation is a local property and that $f$ as defined above
                is locally of finite presentation.

                Let $\Spec(B) \subseteq X$ and $\Spec(A) \subseteq S$ open
                affines.
                Pick a point $x \in \Spec(B)$. Then $x \in \Spec(B) \cap
                \Spec(B_i)$ for some $i$. Pick some neighborhood $x \in U \subseteq \Spec(B) \cap
                \Spec(B_i)$ such that $U$ is principal open in $\Spec(B)$
                and in $\Spec(B_i)$.

                Now take a neighborhood $f(x) \in V \subseteq f(U)$ so that $V$
                is principal open in $\Spec(A)$ and in $\Spec(A_i)$.
                % (Note that then $V \subseteq \Spec(A) \cap \Spec(A_i)$ follows).
                Now take another smaller neighborhood $x \in U' \subseteq
                f^{-1}(V)$ such that $U'$ is principal open in $\Spec(B)$ and in
                $\Spec(B_i)$.

                So we have $U' \to V$, where both $U'$ and $V$ are principal
                opens of $\Spec(B_i)$ and $\Spec(A_i)$ respectively. Since $A_i
                \to B_i$ is of finite presentation, then localizations $(A_i)_f
                \to (B_i)_g$ (for some $f \in A_i$ and $g \in B_i$) are as well.
                %TODO describe this more clearly

                So for every point $x \in \Spec(B)$ we can find a principal open
                neighborhood in $x \in D(f_x)$ and a principal open
                neighborhood $f(x) \in D(g_x)$ such that $A_{g_x} \to B_{f_x}$.

                Since $\Spec(B)$ is quasi-compact, we have $\Spec(B) = D(f_1)
                \cup \dots \cup D(f_n)$. Denote $g_1, \dots, g_n \in A$ be the
                respective elements in $A$.

                We have composition $\Spec(B_{f_i}) \to \Spec(A_{g_i}) \hookrightarrow
                \Spec(A)$, which induces a map of rings $A \to A_{g_i} \to B_{f_i}$.
                Since $A_{g_i} \cong A[X]/(X g_i - 1)$ and $A_{g_i} \to B_{f_i}$
                are of finite presentation by assumption, and being of
                finite presentation is stable under compositions, we have that
                $A \to B_{f_i}$ are of finite presentation for every $i$.

                Now its just commutative algebra to show that $A \to B$ is of
                finite presentation as well, so I hope its okay to assume this
                part. Otherwise we could just rewrite something like
                \href{https://stacks.math.columbia.edu/tag/00EP}{Lemma 00EP}.

                % Since $\Spec(B) = D(f_1) \cup \dots \cup D(f_n)$, we have that
                % $(f_1, \dots, f_n) = B$. Therefore there exist $\alpha_i$ such
                % that $\sum_i \alpha_i f_i = 1$.
            }
    \end{enumerate}
\end{exercise}

\end{document}
