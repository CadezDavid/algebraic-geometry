\newcommand{\sheet}{4}
\documentclass{article}
\usepackage[english, german]{babel}
\usepackage{amsthm,amssymb,amsmath,mathrsfs,mathtools}
\usepackage[shortlabels]{enumitem}
\usepackage{hyperref}
\usepackage{biblatex}
\usepackage{tikz}
\usepackage{tikz-cd}

% \usepackage[tmargin=1.25in,bmargin=1.25in,lmargin=1.2in,rmargin=1.2in]{geometry}


\newcommand{\C}{\mathbb{C}}
\newcommand{\R}{\mathbb{R}}
\newcommand{\N}{\mathbb{N}}
\newcommand{\Q}{\mathbb{Q}}
\newcommand{\Z}{\mathbb{Z}}
\newcommand{\Proj}{\mathbb{P}}
\newcommand{\Aff}{\mathbb{A}}

\DeclareMathOperator{\id}{id}
\DeclareMathOperator{\im}{im}
\DeclareMathOperator{\GL}{GL}
\DeclareMathOperator{\sgn}{sgn}
\DeclareMathOperator{\Tor}{Tor}
\DeclareMathOperator{\Sym}{Sym}
\DeclareMathOperator{\coker}{coker}
\DeclareMathOperator{\Quot}{Quot}
\DeclareMathOperator{\supp}{supp}
\DeclareMathOperator{\Hom}{Hom}
\DeclareMathOperator{\Spec}{Spec}
\DeclareMathOperator{\MinSpec}{MinSpec}
\DeclareMathOperator{\MaxSpec}{MaxSpec}
\DeclareMathOperator{\diag}{diag}
\DeclareMathOperator{\BL}{BL}
\DeclareMathOperator{\Ouv}{Ouv}
\DeclareMathOperator{\Sh}{Sh}
\DeclareMathOperator{\PSh}{PSh}
\DeclareMathOperator{\Eq}{Eq}
\DeclareMathOperator{\colim}{colim}
\DeclareMathOperator{\Pic}{Pic}
\DeclareMathOperator{\CL}{CL}
\DeclareMathOperator{\eq}{eq}
\DeclareMathOperator{\codim}{codim}

\newenvironment{exercise}[1] {
  \vspace{0.5cm}
  \noindent \textbf{Exercise~{#1}.}
} {
  \vspace{0.5cm}
}
\newenvironment{claim} {
  \par\noindent\textbf{Claim.}
} { }

\newenvironment{proof_claim} {
  \par\noindent\textbf{Proof of claim.}
} {
    \qed (of claim)
}

\title{Algebraic geometry 1\\Exercise sheet \sheet}
\author{Solutions by: Eric Rudolph and David Čadež}

\date{\today}


\begin{document}

\maketitle{}

\begin{exercise}{1}
    % \begin{enumerate}
    %     \item Let $X$ be a finite set that is irreducible with 
    %     respect to some topology $\mathcal{F}$ on $X$. Then we get
    %     $\mid \mathcal{F} \mid<\infty$ and since finite unions of
    %     closed sets are closed again, we get that
    %     \begin{align*}
    %         X^{\prime}:=\bigcup_{U \subsetneq X \text{closed}}{U}
    %     \end{align*}
    %     is closed in X. Since $X$ is by assumption irreducible,
    %     $X\neq X^{\prime}$, so we can pick $x_0\in X \backslash X^{\prime}$,
    %     which is by construction generic. For the second part of the exercise we use 
    %     part 2 of Hochster's Theorem. As a finite set, $X$ is 
    %     quasicompact and as a basis $\mathcal{B}$ consisting of quasicompact 
    %     open sets stable under finite intersections take all of the open sets. \\
    %     It remains to show that $X$ is sober. We need to check that every irreducible
    %     subset of $X$ has a unique generic point. The existence of a generic point comes
    %     from part of of this exercise. \\
    %     Uniqueness of this point is due to the fact
    %     that generic points in $T_0$ spaces are unique if they exist, which follows directly
    %     from the definition of $T_0$. 
    % \end{enumerate}
    
    \begin{enumerate}
        \item{Let $X$ be a finite irreducible topological space. Since
            \begin{equation*}
                X = \bigcup_{x \in X} \overline{\{x\}}
            \end{equation*}
            is a finite decomposition in closed sets, we must have
            $\overline{\{x\}} = X$ for some $x \in X$. This $x$ is a generic
            point of $X$.

            If we additionally assumed $X$ is $T_0$, then this point $x$ would
            be unique, since in a $T_0$ space we have $\overline{\{x\}} \not=
            \overline{\{y\}}$ for $x \not= y$. Other conditions are also
            fulfilled, namely a finite space is always quasicompact. For the
            basis we can take $B$ to contain all open sets, it is clearly closed
            under finite intersections and every element is quasicompact. So
            finite $T_0$ spaces are spectral.
            }
        \item{
                % Let us first describe what $\Spec(\Z)$ looks like. It is a PID
                % with prime ideals being those $(a)$ for which $a \in \Z$ is a
                % prime number or $a = 0$. So
                % \begin{equation*}
                %     \Spec(\Z) = \{ (p) \mid p\ \text{prime} \} \cup \{ (0) \}.
                % \end{equation*}
                % In this space $(0)$ is a unique generic point.

                % Closed sets in $\Spec(\Z)$ are by definiton
                % \begin{equation*}
                %     V((a)) = \{ (p) \in \Spec(\Z) \mid (a) \subseteq (p) \} = 
                %     \{ (p) \in \Spec(\Z) \mid p\ \text{divides}\ a \}.
                % \end{equation*}
                % So if $a = \Pi_{i} p^{k_i}_i$, then $V((a)) = {\{ (p_i) \}}_i
                % \subseteq \Spec(\Z)$. Since any $a \in \Z$ is only divisible by
                % finitely many prime numbers, we get the finite completement
                % topology on $\Spec(\Z) \setminus \{(0)\}$.

                % Adding generic point $(0)$ to $\Spec(\Z) \setminus \{(0)\}$ is
                % actually the construction $X \rightarrow X^{\text{sob}}$ which
                % we did last week.

                % Let $\mathbb{P} = \{ p_1, p_2, \dots \} \subseteq \N$ be an
                % ``ordered set'' of prime numbers.

                % Construct an inverse limit
                % \begin{equation*}
                %     F_i := \{ (p_1), \dots, (p_i) \} \cup \{ (0) \}
                % \end{equation*}
                % for $i \in \N$ and
                % \begin{equation*}
                %     f_i \colon F_i \xhookrightarrow{} F_{i+1}
                % \end{equation*}
                % being subset inclusions. We still have to say what topologies
                % $F_i$ have and then show that $f_i$ are continuous. Define
                % closed sets on $F_i$ to be $A \subseteq F_i$ with $(0) \notin
                % A$, and the whole $F_i$. This basis for closed sets is clearly
                % stable under finite unions.

                % Now show that $f_i$ are continuous. Take $A \subseteq F_{i+1}$
                % closed. Then $f^{-1}_i(A) = A \cap F_i$. If $(0) \in A$, then
                % $(0) \in A \cap F_i$ and therefore $A = F_{i+1}$. Thus
                % $f^{-1}_i(A) = F_i$ is closed. If $(0) \notin A$, then $(0)
                % \notin A \cap F_i$ and thus $f^{-1}_i((0)) = A \cap F_i
                % \subseteq F_i$ also closed.

                % \begin{tikzcd}
                %     &F_1 \arrow{drr}{g_1} \arrow{r}{f_1} &F_2 \arrow{dr}{g_2}
                %     \arrow{r}{f_2} &F_3 \arrow{d}{g_3} \arrow{r}{f_3} &F_4
                %     \arrow{dl}{g_4} \arrow{r}{f_4} &\dots \arrow{dll}{g_i}\\
                %     && &F
                % \end{tikzcd}

                % Now we claim that $\Spec(\Z)$ is the inverse limit of this
                % diagram $(F_i, f_i)$.

                % Define $g_i \colon F_i \rightarrow \Spec(\Z)$ to be the
                % inclusions. Commutativity of the diagram ($g_i = g_{i+1} \circ
                % f_i$) is clear, since all maps are inclusions.

                % Take some other topological space $Y$ with $h_i \colon F_i
                % \rightarrow Y$. Define the map $h \colon \Spec(\Z) \rightarrow
                % Y$ by mapping $(p_i) \mapsto h_i((p_i))$, i.e. we identify
                % $(p_i)$ with $(p_i) \in F_i \subseteq F_{i+1} \subseteq \dots$
                % and then map it by $h_i \colon F_i \rightarrow Y$.

                % To check continuity take a closed set $A \subseteq Y$. By
                % definition $h^{-1}(A) = \cap $
            }
    \end{enumerate}
\end{exercise}

\begin{exercise}{2}
    Denote $A = \lim A_i$, $B = \lim B_i$ and $C = \lim C_i$. Also denote maps
    $A_i \rightarrow A$ with $f_i$, $B_i \rightarrow B$ with $g_i$ and $C_i
    \rightarrow C$ with $h_i$.

    By composing $\alpha_i$ and $g_i$ we get $A_i \rightarrow B$ defined as $g_i
    \circ \alpha_i$. Then by the definition of a colimit we have a unique map $\alpha:A
    \rightarrow B$, such that $g_i \circ \alpha_i = \alpha
    \circ f_i$. In the same way we obtain $\beta \colon B \rightarrow C$. With
    these definitions the whole diagram commutes.

    First look at maps $A_i \rightarrow B_i \rightarrow C_i \rightarrow C$.
    Since bottom rows are all exact, they are $0$ for all $i$. Since $\beta
    \circ \alpha$ makes this commute and $0$ does well, by uniqueness of the map
    $A \rightarrow C$, we get $\beta \circ \alpha = 0$. In this argument we did
    not use that the colimit is filtered in our case, because it is true in
    general that colimits are right-exact.

    For left exactness we need to use that we have a filtered colimit.
    For this part we used Proposition 2 from page 212 of Mac Lane's ``Categories For
    the Working Mathematician,'' stating ``The forgetful functor $Grp \rightarrow
    Set$ creates filtered colimits.''
    Now take $b \in B$ with $\beta(b) = 0$. Using the proposition we know there
    exists an $i$ and $\overline{b} \in B_i$ for which $g_i(\overline{b}) = b$.
    Since diagram commutes we have $(h_i \circ \beta_i)(b) = 0$. But since $h_i$
    maps an element to $0$ if and only if there exists a $C_j$ and $C_i
    \rightarrow C_j$ (in a diagram) that maps it to $0$, we must have $C_j$ and
    $\gamma \colon C_i \rightarrow C_j$ (in a diagram) with $\gamma(\beta_i(b))
    = 0$. Take $B_j$ and the map $\delta \colon B_i \rightarrow B_j$ with
    $(\beta_j \circ \delta)(\overline{b}) = 0$. Since bottom rows are exact, we
    have $\overline{a} \in A_j$ with $\alpha_j(\overline{a}) =
    \delta(\overline{b})$. Then we have $(g_j \circ \alpha_j)(\overline{a}) =
    b$ and using commutativity of the diagram we get $\alpha(f_j(\overline{a}))
    = b$, so $b \in \im(\alpha)$.

    This completes the proof of exactness. In it we a little implicitly used the
    facts that follow from the construction of a colimit in abelian category by
    first computing the colimit in sets and then defining a structure on that
    set. But we could not find a way to do this more elegantly without chasing
    an explicit element around the diagram.
\end{exercise}

\begin{exercise}{3}
    \begin{enumerate}
        \item{Let $\mathcal{F}, \mathcal{G} \in \Sh_{\text{Ab}}(X)$ and $\phi,
            \psi \colon \mathcal{F} \rightarrow \mathcal{G}$ morphisms of
            sheaves.

            We define $\phi + \psi$ to be a morphism of sheaves defined by
            \begin{equation*}
                (\phi+\psi)(U):=\phi(U) + \psi(U) \colon \mathcal{F}(U)
                \rightarrow \mathcal{G}(U).
            \end{equation*}
            for every $U \in \Ouv^{\text{op}}_X$. So its just a sum of
            homomorphisms of abelian groups. It is clear that this still defines
            a natural transformation (i.e. a morphism) in $\Sh_{\text{Ab}}(X)$.

            The zero object in this category is the sheaf that sends every open
            set $U \in \Ouv^{\text{op}}_X$ to the trivial group $(0,+)$. It is
            inital and terminal, because the trivial group is a zero object in
            the category of abelian groups. For this reason the category of
            sheaves is an additive category.

            Instead of checking for kernels and cokernels, we simply check for
            all limits and colimits, of which they are a special case.

            Let us check that it has limits and colimits. Let $F_i$ be some
            diagram in $\Sh_{\text{Ab}}(X)$. Define a limit of this diagram to
            be a sheaf $F$ which maps $U \in \Ouv^{\text{op}}_X$ to $\lim_i
            F_i(U)$. So for each $U$ we let $F(U)$ be the limit of a diagram
            $F_i(U)$ (which we know exist because its in the category of abelian
            groups).

            Check that $F$ this is actually a limit in $\Sh_{\text{Ab}}(X)$. Let
            $G$ be some other presheaf with maps $G \rightarrow F_i$. Then for
            every $U \in \Ouv^{\text{op}}_X$ we have maps $G(U) \rightarrow
            F_i(U)$, so this cone factors uniquely through $F(U)$. This gives
            rise to a morphism $G \rightarrow F$. So $F$ is indeed a limit.

            Now we check that $F$ is a presheaf. Let $U \rightarrow V$. Then we
            have maps $F_i(U) \rightarrow F_i(V)$. Composing them with $F(U)
            \rightarrow F_i(U)$ gives maps $F(U) \rightarrow F_i(V)$. By the
            definition of a limit there exists a unique map $F(U) \rightarrow
            F(V)$. This proves that $F$ is indeed a presheaf.

            Now sheafify $F$ to get a sheaf $\overline{F} \in
            \Sh_{\text{Ab}}(X)$. Since $F_i$ are sheaves, the maps $F
            \rightarrow F_i$ uniquely factors through $\overline{F}$. For any
            sheaf $G$ and a cone $G \rightarrow F_i$, we can look at this inside
            the category of presheaves and find a unique map $F \rightarrow G$.
            But since $G$ is a sheaf, it factors through $\overline{F}$ and we
            get a unique map $\overline{F} \rightarrow G$. This proves that
            $\overline{F}$ is a limit of $F_i \in \Sh_{\text{Ab}}(X)$.

            For colimits everything works exactly the same, except all arrows
            are reversed.

            Maybe some parts of this proof follow from some more elegant
            argument using adjoints and forgetful functors.
            }

        \item{Assume $f \colon \mathcal{F} \rightarrow \mathcal{G}$ is
            surjective on each stalk. Then we can show that $f(U) \colon
            \mathcal{F}(U) \rightarrow \mathcal{G}(U)$ is surjective for each $U
            \in \Ouv^{\text{op}}_X$. Pick $s \in \mathcal{G}(U)$. Then $s_x \in
            \mathcal{G}_x$ the germ of $s$ at $x$. Since $f_x \colon
            \mathcal{F}_x \rightarrow \mathcal{G}_x$ is surjective we have a
            function $t_x \in \mathcal{F}(U)$ such that $f(U)(t_x)$ has the same
            germ at $x$ as $s$. So $t_x \in \mathcal{F}(U)$ is mapped to a
            function that is locally at $x$ same as $s$. Denote $V_x$ the open
            subset with $x \in V_x$ for which $f(U)(t_x) \mid_{V_x} = s
            \mid_{V_x}$. So we have a cover $X = \cup_x V_x$. Using that
            $\mathcal{F}$ is a sheaf we can find an element $t$ such that
            $f(U)(t) = s$. That proves one implication.

            We were inspired by
            \href{https://math.stackexchange.com/questions/205658/an-easy-way-to-prove-that-epimorphism-of-sheaves-implies-surjectivity-on-stalk?rq=1}{some
            stack exchange site} and only worked out some of the details. Let
            $\mathcal{F}, \mathcal{G}$ be sheaves on $X$ and
            $\phi:\mathcal{F}\to \mathcal{G}$ an epimorphism on sheaves. We want
            to show that $\phi$ is surjective on each stalk.\\ Let $\mathcal{F}$
            be the skyscraper sheaf at $x$ with value $\mathcal{G}_x/\im(\phi_x)$,
            i.e. for $U\subset X$ open, we have
            \begin{equation*}
                \mathcal{F}(U)=
                \begin{cases*}
                    \mathcal{G}_x/\im(\phi_x) & \text{if} $x\in U$\\
                    (0,+) & else.
                \end{cases*}
            \end{equation*}
            Also, define
            \begin{align*}
                \psi&:\mathcal{G}\to \mathcal{H}\\
                    &f\mapsto \bar{f}_x
            \end{align*}
            if f is a section over a set containig x. \\
            Now assume, that $\phi_x$ is not surjective for all $x\in X$.
            Then there exists $x\in X$ such that $\phi_x$ is not surjective,
            meaning $\mathcal{G}_x/\im(\phi_x)$ is not the trivial group. 
            In this case, the two maps 
            \begin{align*}
                (\psi,0),(0,\psi):\mathcal{G}\to \mathcal{H} \oplus \mathcal{H}
            \end{align*}
            are not the same. However, by
            construction they are the same if precomposed with $\phi$ showing that
            $\phi$ is not an epimorphism of sheaves.
            }\label{ex3.2}

        \item{First prove that $\mathcal{F}$ and $\mathcal{G}$ are sheaves.

            Define a morphism $f \colon \mathcal{F} \rightarrow \mathcal{G}$
            with $f(U) \colon \mathcal{F}(U) \rightarrow \mathcal{G}(U)$ being
            the map $s \oplus t \mapsto s \mid_{D_- \cap D_+}$. Note that if $U
            \cap D_-$ is connected, the map $s \mid_{D_- \cap D_+}$ must assume
            same value on whole $U \cap D_- \cap D_+$.

            Using part~\ref{ex3.2} it suffices to check that $f_x$ is surjective
            at every $x \in X$. For $x \notin D_- \cap D_+$, then
            $\mathcal{G}_x$ is a null group in $AbGrp$, so $f_x$ is surjective.
            Let now $x = (-1, 0)$.

            Define $U_\epsilon = B_\epsilon(-1, 0) \cap X$. Every neighborhood
            contains some $U_\epsilon$ for small enough $\epsilon$. So every
            $\mathcal{F}(U) \rightarrow F_x$ factors through
            $\mathcal{F}(U_\epsilon)$ for small enough $\epsilon$. We have
            $\mathcal{F}(U_\epsilon) = \Z \oplus \{0\}$ for small $\epsilon >
            0$, so $\mathcal{F}_x = \Z$ with morphisms
            \begin{equation*}
                \mathcal{F}(U_\epsilon) \rightarrow \mathcal{F}_x
            \end{equation*}
            being $a \oplus 0 \mapsto a$. We do exactly the same for
            $\mathcal{G}$ and we get $\mathcal{G}_x = \Z$. The identity $\id$ is
            the unique map that makes

            \begin{tikzcd}
                & \mathcal{F}(U_\epsilon) \arrow{r}{f(U_\epsilon)} \arrow{d} &
                \mathcal{G}(U_\epsilon) \arrow{d} \\
                & \mathcal{F}_x \arrow{r}{\id} & \mathcal{G}_x
            \end{tikzcd}

            commute, so $f_x$ is surjective for $x = (-1, 0)$. We do exactly the
            same for $x = (1, 0)$. So $f_x$ is surjective for every $x \in X$. By
            part~\ref{ex3.2}, $\mathcal{F}$ is an epimorphism.

            We have $\mathcal{F}(X) \cong \Z \oplus \Z$ and $\mathcal{G}(X)
            \cong \Z \oplus \Z$ and the map $f(X)$ as defined above. The image
            $\im f(X)$ is the diagonal $\triangle (\Z \oplus \Z)$, so $f(X)$ is
            not surjective.
            }
    \end{enumerate}
\end{exercise}

\begin{exercise}{4}
    Denote the maps
    \begin{equation*}
        \varphi \colon \Sh(X) \rightarrow \Sh_B(X)
    \end{equation*}
    and
    \begin{equation*}
        \mu \colon \Sh_B(X) \rightarrow \Sh(X).
    \end{equation*}
    The composite $\varphi \circ \mu$ is the identity on $\Sh_B(X)$.

    We have to prove that $\mu \circ \varphi \colon \Sh(X) \rightarrow \Sh(X)$
    mapping a functor $F$ to a functor $(U \mapsto \lim_{B \subseteq U} F(B))$
    is equivalent to the identity.

    Define a natural transformation
    \begin{equation*}
        \zeta \colon \id \rightarrow \mu \circ \varphi
    \end{equation*}
    which for every $F \in \Sh(X)$ defines a map
    \begin{equation*}
        \id(F) = F \longrightarrow (U \mapsto \lim_{B \subseteq U} F(B))
    \end{equation*}
    being the unique natural transformation $F(U) \rightarrow \lim_{B \subseteq
    U} F(B)$ (which exists $F(U) \rightarrow F(B)$ exist for every $B \subseteq
    U$).
\end{exercise}

\end{document}
