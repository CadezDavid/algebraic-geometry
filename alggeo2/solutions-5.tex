\newcommand{\sheet}{5}
\documentclass{article}
\usepackage[english, german]{babel}
\usepackage{amsthm,amssymb,amsmath,mathrsfs,mathtools}
\usepackage[shortlabels]{enumitem}
\usepackage{tikz}
\usepackage{tikz-cd}

% \usepackage[tmargin=1.25in,bmargin=1.25in,lmargin=1.2in,rmargin=1.2in]{geometry}


\newcommand{\C}{\mathbb{C}}
\newcommand{\R}{\mathbb{R}}
\newcommand{\N}{\mathbb{N}}
\newcommand{\Q}{\mathbb{Q}}
\newcommand{\Z}{\mathbb{Z}}

\DeclareMathOperator{\id}{id}
\DeclareMathOperator{\im}{im}
\DeclareMathOperator{\GL}{GL}
\DeclareMathOperator{\sgn}{sgn}
\DeclareMathOperator{\Tor}{Tor}
\DeclareMathOperator{\Sym}{Sym}
\DeclareMathOperator{\coker}{coker}
\DeclareMathOperator{\Quot}{Quot}
\DeclareMathOperator{\supp}{supp}
\DeclareMathOperator{\Hom}{Hom}
\DeclareMathOperator{\Spec}{Spec}
\DeclareMathOperator{\MinSpec}{MinSpec}
\DeclareMathOperator{\diag}{diag}


\newenvironment{exercise}[1] {
  \vspace{0.5cm}
  \noindent \textbf{Exercise~{#1}.}
} {
  \vspace{0.5cm}
}
\newenvironment{claim} {
  \noindent \textbf{Claim.}
} {
}

\title{Algebraic geometry 1\\Exercise sheet \sheet}
\author{Solutions by: Eric Rudolph and David Čadež}

\date{\today}


\begin{document}

\maketitle{}

\begin{exercise}{1}
    \begin{enumerate}
        \item{
                Flatness is local on the target, so we can check it on affine
                cover of $\Proj^1_A$.

                Lets first look at $D(z) \to D(z)$.
                We have a map of rings $A[y] \to A[x, y]/(y^2 - g(x))$ mapping
                $y \mapsto y$. We have to check the target is a flat
                $A[y]$-module. We can let $B := A[y]$, then $B \to B[x]/(b -
                g(x))$ (where $b = y^2 \in B$). Writing it this way makes it
                clear that the target is isomorphic to $B^d$ as a $B$-module.

                And now observe $D(y) \to D(y)$.
                We similarly get a ring map $A[z] \to A[x, z]/(z^{d-2} - z^d
                g(x/z))$ with $z \mapsto z$. Again, setting $B := A[z]$ makes it
                clear that $B \to B[x]/(f)$ ($f$ some polynomial in $B[x]$ of
                degree $d$) makes the target a finite free $B$-module.

                So $X \to \Proj^1_A$ is flat.

                Morphism $\Proj^1_A \to \Spec(A)$ is flat, because maps $A \to
                A[t]$ are flat.

                Composition of flat morphisms is flat, so $X \to \Proj^1_A \to
                S = \Spec(A)$ is flat.
            }
        \item{
                On discord somebody wrote an example that they think is a
                counterexample to this statement.
                And to me it seems like its valid.
            }
        \item{
                In the first case the discriminant is $a^2 - 4b$ and in the
                second it is $-4a^3 - 27b^2$.
            }
    \end{enumerate}
\end{exercise}

\begin{exercise}{3}
    Because curves $X$ and $Y$ are smooth, local rings $\mathcal{O}_{X, x}$
    and $\mathcal{O}_{Y, y}$ are geometrically regular. In our case $k$ is
    algebraically closed, so they are already regular without any base
    change necessary.
    And regular $1$-dimensional local rings (and integral) are DVRs, so we can
    pick $t_x$ and $t_y$ to be their respective uniformizers.

    Then, using commutative algebra facts, we get that $\widehat{\mathcal{O}_{X,
    x}} = k[[t_x]]$ and $\widehat{\mathcal{O}_{Y, y}} = k[[t_y]]$.

    By our assumptions we have $f^*_y(t_x) = s t^e_y$ for some $s \in
    \mathcal{O}^\times_{Y, y}$. By previous exercise we know $s$ admits an
    $e$-th root. Then we can go back to the start and pick $t_y$ to be
    $\sqrt{e}{y} t_y$, which is still a uniformizer as $\sqrt{e}{y} \in
    \mathcal{O}^\times_{Y, y}$. So we can assume $f^*_y(t_x) = t^e_y$. Map
    $f^*_y$ induces a map between diagrams $\{ \mathcal{O}_{X, x}/m^i_{X, x}
    \}_i \to \{ \mathcal{O}_{Y, y}/m^i_{Y, y} \}_i$, which in turn gives a map
    between limits of these diagrams, i.e. the completions. This induces map
    between completions is clearly given by the same rule $t_x \mapsto t^e_y$.
\end{exercise}

\end{document}
