\newcommand{\sheet}{8}
\documentclass{article}
\usepackage[english, german]{babel}
\usepackage{amsthm,amssymb,amsmath,mathrsfs,mathtools}
\usepackage[shortlabels]{enumitem}
\usepackage{hyperref}
\usepackage{biblatex}
\usepackage{tikz}
\usepackage{tikz-cd}

% \usepackage[tmargin=1.25in,bmargin=1.25in,lmargin=1.2in,rmargin=1.2in]{geometry}


\newcommand{\C}{\mathbb{C}}
\newcommand{\R}{\mathbb{R}}
\newcommand{\N}{\mathbb{N}}
\newcommand{\Q}{\mathbb{Q}}
\newcommand{\Z}{\mathbb{Z}}
\newcommand{\Proj}{\mathbb{P}}
\newcommand{\Aff}{\mathbb{A}}

\DeclareMathOperator{\id}{id}
\DeclareMathOperator{\im}{im}
\DeclareMathOperator{\GL}{GL}
\DeclareMathOperator{\sgn}{sgn}
\DeclareMathOperator{\Tor}{Tor}
\DeclareMathOperator{\Sym}{Sym}
\DeclareMathOperator{\coker}{coker}
\DeclareMathOperator{\Quot}{Quot}
\DeclareMathOperator{\supp}{supp}
\DeclareMathOperator{\Hom}{Hom}
\DeclareMathOperator{\Spec}{Spec}
\DeclareMathOperator{\MinSpec}{MinSpec}
\DeclareMathOperator{\MaxSpec}{MaxSpec}
\DeclareMathOperator{\diag}{diag}
\DeclareMathOperator{\BL}{BL}
\DeclareMathOperator{\Ouv}{Ouv}
\DeclareMathOperator{\Sh}{Sh}
\DeclareMathOperator{\PSh}{PSh}
\DeclareMathOperator{\Eq}{Eq}
\DeclareMathOperator{\colim}{colim}
\DeclareMathOperator{\Pic}{Pic}
\DeclareMathOperator{\CL}{CL}
\DeclareMathOperator{\eq}{eq}
\DeclareMathOperator{\codim}{codim}

\newenvironment{exercise}[1] {
  \vspace{0.5cm}
  \noindent \textbf{Exercise~{#1}.}
} {
  \vspace{0.5cm}
}
\newenvironment{claim} {
  \par\noindent\textbf{Claim.}
} { }

\newenvironment{proof_claim} {
  \par\noindent\textbf{Proof of claim.}
} {
    \qed (of claim)
}

\title{Algebraic geometry 1\\Exercise sheet \sheet}
\author{Solutions by: Eric Rudolph and David Čadež}

\date{\today}


\begin{document}

\maketitle{}

\begin{exercise}{2}
    
    \href{http://math.uchicago.edu/~may/REU2021/REUPapers/Davidovsky.pdf}{source}
    \begin{enumerate}
        \item On the right side, we are given transition maps. 
        
        We have that
        \begin{align*}
            \alpha_{| U_0\cap U_1}^{-1} \circ \beta_{| U_0\cap U_1}
        \end{align*}
        is invertible, because by assumption $\alpha$ and $\beta$ are isomorphisms. 
        To see injectivity, remember that given sheaves on a cover and transition maps,
        we can uniquely (up to isomorphism) glue them to get a sheaf on the whole space.

        Well definedness of this map comes from the fact that if two vector bundles 
        $V_1\cong V_2$ are ismorphic, then the transition map is the same.

        It remains to show surjectivity.
        
        \item dots
        \item 
        Claim: Every line bundle on $\Proj^1$ can be written as 
        \begin{align*}
            \mathcal{O}_{\Proj^1}(d_n)\oplus \dots \oplus \mathcal{O}_{\Proj ^1}(d_n).
        \end{align*}
        
        Proof of claim:

        By part 1 of this exercise, we can characterize the isomorphism classes of rank n vector bundles by looking at the transition functions. 

        In the second part of this exercise, we showed that (for $n=2$, but actually inductively for all $n$) these 
        transition functions can be writte as $T^d$. The claim now follows from observing that the transition matrix of 
        \begin{align*}
            \mathcal{O}_{\Proj^1}(d_n)\oplus \dots \oplus \mathcal{O}_{\Proj ^1}(d_n)
        \end{align*}
        is given by
        \begin{align*}
            T^{(d_1,\dots, d_n)}.
        \end{align*}
    \end{enumerate}
\end{exercise}

\end{document}
