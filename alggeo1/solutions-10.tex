\newcommand{\sheet}{10}
\documentclass{article}
\usepackage[english, german]{babel}
\usepackage{amsthm,amssymb,amsmath,mathrsfs,mathtools}
\usepackage[shortlabels]{enumitem}
\usepackage{hyperref}
\usepackage{biblatex}
\usepackage{tikz}
\usepackage{tikz-cd}

% \usepackage[tmargin=1.25in,bmargin=1.25in,lmargin=1.2in,rmargin=1.2in]{geometry}


\newcommand{\C}{\mathbb{C}}
\newcommand{\R}{\mathbb{R}}
\newcommand{\N}{\mathbb{N}}
\newcommand{\Q}{\mathbb{Q}}
\newcommand{\Z}{\mathbb{Z}}
\newcommand{\Proj}{\mathbb{P}}
\newcommand{\Aff}{\mathbb{A}}

\DeclareMathOperator{\id}{id}
\DeclareMathOperator{\im}{im}
\DeclareMathOperator{\GL}{GL}
\DeclareMathOperator{\sgn}{sgn}
\DeclareMathOperator{\Tor}{Tor}
\DeclareMathOperator{\Sym}{Sym}
\DeclareMathOperator{\coker}{coker}
\DeclareMathOperator{\Quot}{Quot}
\DeclareMathOperator{\supp}{supp}
\DeclareMathOperator{\Hom}{Hom}
\DeclareMathOperator{\Spec}{Spec}
\DeclareMathOperator{\MinSpec}{MinSpec}
\DeclareMathOperator{\MaxSpec}{MaxSpec}
\DeclareMathOperator{\diag}{diag}
\DeclareMathOperator{\BL}{BL}
\DeclareMathOperator{\Ouv}{Ouv}
\DeclareMathOperator{\Sh}{Sh}
\DeclareMathOperator{\PSh}{PSh}
\DeclareMathOperator{\Eq}{Eq}
\DeclareMathOperator{\colim}{colim}
\DeclareMathOperator{\Pic}{Pic}
\DeclareMathOperator{\CL}{CL}
\DeclareMathOperator{\eq}{eq}
\DeclareMathOperator{\codim}{codim}

\newenvironment{exercise}[1] {
  \vspace{0.5cm}
  \noindent \textbf{Exercise~{#1}.}
} {
  \vspace{0.5cm}
}
\newenvironment{claim} {
  \par\noindent\textbf{Claim.}
} { }

\newenvironment{proof_claim} {
  \par\noindent\textbf{Proof of claim.}
} {
    \qed (of claim)
}

\title{Algebraic geometry 1\\Exercise sheet \sheet}
\author{Solutions by: Eric Rudolph and David Čadež}

\date{\today}


\begin{document}

\maketitle{}

\begin{exercise}{1}
    \begin{enumerate}
        \item{
                Since $X$ is closed and irreducible, it is of the form $X =
                \overline{\{p_0\}}$ for some (Eric thinks unique) $p_0 \in \Aff^n_k$.
                That means $X \cong \Spec(k[x_1, \dots x_n] / p_0)$.
                Denote $A = k[x_1, \dots x_n] / p_0$.

                By assumption there is a chain of specializations $p_0 \subset
                \dots \subset p_d$ inside $X$.

                Let $Z \subseteq X \cap V(f_1, \dots, f_r)$ be a irreducible
                component. Thus it is the closure of a minimal prime ideal in $A
                / (f_1, \dots, f_r)$.

                By Krull's principal ideal theorem we have $\dim(A / (f_1,
                \dots, f_r)) \geq d - r$.

                Denote minimal prime ideals in $A / (f_1, \dots, f_r)$ with
                $q_1, \dots q_l$.

                (Eric thinks that there is a unique generic point here again, since $X$ is sober, so 
                there should only be one of these prime ideals, right?)
                
                We argue that
                \begin{equation*}
                    \dim(A / (f_1, \dots, f_r) / q_j) = \dim(A / (f_1, \dots, f_r)).
                \end{equation*}
                for any $j$.

                That follows from $A$ being catenary.
                If there existed a maximal chain in $A / (f_1, \dots, f_r)$ that
                starts at $q_j$ we could simply extend it below to get a
                maximal chain in $A$. Since all maximal chains in $A$ are of the
                same length, we get that all maximal chains in $A / (f_1, \dots,
                f_r)$ are also of the same length.

                Since $Z$ is an irreducible component, we have $Z =
                \overline{\{q_i\}} \subseteq \Spec(A / (f_1, \dots, f_r))$.

                Therefore any maximal chain in $Z$ is exactly as long as the
                longest chain in $A / (f_1, \dots, f_r)$. And the longest chain
                in $A / (f_1, \dots, f_r)$ is at least of length $d - r$.
            }
        \item{
                The diagonal $\triangle \subseteq \Aff^n_k \times \Aff^n_k$ can
                be defined as $V(x_i \otimes x_i \mid i = 1, \dots n) \subseteq
                \Spec(k[x_1, \dots, x_n] \otimes_\Z k[x_1, \dots, x_n])$.

                (Should there not be a minus instead of $\otimes$ in the above expression?)

                Using exercise above we get that any irreducible component of
                $X \cap Y \cong (X \times Y) \cap V(x_i \otimes x_i \mid i = 1,
                \dots n)$ has dimension at least $d + e - n$.
            }
        \item{
                Let $\tilde{X} = \overline{f^{-1}(X)}$ and $\tilde{Y} = \overline{f^{-1}(Y)}$ as in
                the hint.

                We have $\dim(\tilde{X}) = d + 1$ and $\dim(\tilde{Y}) = e + 1$.
                By the previous exercise we have $\dim(\tilde{X} \cap \tilde{Y})
                \geq d + 1 + e + 1 - (n + 1) = (d + e - n) + 1 \geq 1$.

                Therefore there there exists $0 \not= x \in \tilde{X} \cap
                \tilde{Y}$.

                Questions from Eric:
                
                Why is $\tilde{X}$ irreducible (to be able to use part 2) and why does the dimension 
                increase by 1 when we go to affine space?

            }
    \end{enumerate}
\end{exercise}

\begin{exercise}{2}
    \begin{enumerate}
        \item{It is enough to show that there exists a cover $X=\cup_i \Spec(A_i)$ of $X$ by open affines 
        such that $f^{-1}(\Spec(A_i))$ is affine for all $i$. Therefore it is enough to show that the hint holds,
        since $X=\cup_{x\in X} U_x$, where each $U_x$ is an open affine with $x\in U_x$.

        Take some $x\in X$. If $x \not \in f(Y)$ then by continuity and since $f(X)$ is closed, 
        there exists an affine $U_x$ that is disjoint from $f(Y)$. In this case $f^{-1}(U_x)=\emptyset=D(1)$ the preimage under 
        $f$ is of affine.
        
        If $x\in f(Y)$, then $x\in \Spec(A_k)$ for some $k$. 
        Now choose some principal open $D(g)\subset Z$ with $f^{-1}(x)\in D(g)$. 

        We can now take a principal open $D(g^\prime) \subset f(D(g))$ such that $D(g^\prime) \subset U_k$ and $x\in D(g^\prime)$.
        Then we can show similarly to exercise 4.1 on sheet 8 that 
        \begin{align*}
            f^{-1}(D(g^\prime))
        \end{align*}
        is a principal open again, so in particular affine.
                

        v2

        Take $x \in |X|$. If $x \notin f(|Y|)$, we can find an open $U_x$ such
        that $f^{-1} (U_x) = \emptyset$. So assume $x \in f(|Y|)$. Then look
        at $f^{-1}(x)$. Take an open affine $V_x \subseteq |Y|$ with $f^{-1}
        (x) \in V_x$. Since $f$ is homeomorphism on its image, we have can
        take $U_x = f(V_x)$ an affine neighborhood of $x$ such that
        $f^{-1}(U_x) = V_x$ is affine.
        }
        \item{Assume that $f$ is universally closed. We want to show that $f$ is
            integral, surjective and universally injective. By the first part of
            this exercise we get that $f$ is affine. It says on
            \href{https://stacks.math.columbia.edu/tag/01WM}{The Stacks project}
            that maps that are affine and unversally closed are also integral.
            The other two properties follow immediately from the assumptions.

            On the other hand, assume that $f$ is integral, surjective and
            universally injective. We now from algebra 1 that integral maps are
            closed and we learned in this course that the property of a morphism
            being integral is stable under base change. If you put these two
            facts together you get that integral maps are universally closed.

            We also know from \href{https://stacks.math.columbia.edu/tag/01S1}{The
            Stacks project} that the property of a map of schemes being
            surjective is stable under base change, so $f$ surjective already
            implies $f$ unviversally surjective. All in all, we get that $f$ is
            unviversally bijective.
            }
    \end{enumerate}
\end{exercise}


    % https://math.stackexchange.com/questions/3774859/global-sections-of-a-proper-variety-over-an-arbitrary-field
\begin{exercise}{3}
    \begin{enumerate}
        \item{
                We have a map $k \rightarrow \Gamma(X, \mathcal{O}_X)$.
                For any $f \in \Gamma(X, \mathcal{O}_X)$ we can define $k[x]
                \rightarrow \Gamma(X, \mathcal{O}_X)$ by $x \mapsto f$.

                Pick some $f \in \Gamma(X, \mathcal{O}_X)$ and define the
                induced $g \colon X \to \Aff^1_k$.

                First observe that:
                $g(X)$ does not contain the generic point of $\Aff^1_k$ if and
                only if $k[x] \to \Gamma(X, \mathcal{O}_X)$ with $x \mapsto f$ is not injective.

                We have a composition $k \to k[x] \to \Gamma(X, \mathcal{O}_X)$.
                So also $X \to \Aff^1_k \to \Spec(k)$.

                Map $X \to \Spec(k)$ is proper.

                Map $\Aff^1_k \to \Spec(k)$ is separated, since it is a map of
                affine schemes. (Follows from the fact that $k[x] \otimes_k k[x]
                \to k[x]$ is surjective, and thus $\Aff^1_k \to \Aff^1_k
                \times_k \Aff^1_k$ a closed immersion.)

                Thus by the proposition from the lectures, the map $g \colon X \to
                \Aff^1_k$ is proper. In particular it is closed. Since $X$ is
                connected, the image $g(X)$ is connected as well.

                Using the hint, we can postcompose with open inclusion $\Aff^1_k
                \subseteq \Proj^1_k$ to obtain $X \to \Aff^1_k \to \Proj^1_k$.
                Now the conclusion should be that the image of $X$ in
                $\Proj^1_k$ is also closed. Since $\Aff^1_k \subseteq \Proj^1_k$
                is not closed, the image of $X$ cannot be whole $\Aff^1_k$.
                Therefore it must be a single point.

                Since we did not exactly understand why should composition $X
                \to \Proj^1_k$ be closed, we decided to rather show that $X \to
                \Aff^1_k$ cannot be surjective, as that would imply $\Aff^1_k$
                being universally closed over $\Spec(k)$ (which we've shown
                during the lectures to be false). 

                Instead of doing it abstractly, we can show that $X \to
                \Aff^1_k$ being surjective would imply $\Aff^2_k \to \Aff^1_k$
                being closed.

                By the universal property of $\Aff^2_k$ we get a map $X \to
                \Aff^2_k$, induced by $X \to \Aff^1_k$.
                So we have a map $X \to \Aff^2_k \to \Aff^1_k$.
                Denote $\alpha \colon X \to \Aff^2_k$ and $\beta \colon \Aff^2_k
                \to \Aff^1_k$.
                If $\alpha$ would be surjective, then for any $U \subseteq
                \Aff^2_k$ we would have $(\beta \circ \alpha)(\alpha^{-1}(U)) =
                \beta(U)$. Since $\beta \circ \alpha$ is closed by assumption,
                this would prove that $\beta$ is closed. That is not true, so
                $g = \beta \circ \alpha$ is not surjective.

                We've shown that the image of $X \to \Aff^1_k$ is a single
                point. Since this point is closed, it is not the generic point.
                This shows that $k[x] \to \Gamma(X, \mathcal{O}_X)$ induced by
                $f \in \Gamma(X, \mathcal{O}_X)$ is not injective.
            }
        \item{
                We have a map $k \to \Gamma(X, \mathcal{O}_X)$. It cannot be
                $0$, since $X$ is locally finite type over $\Spec(k)$. So it is
                injective.

                It is also surjective, since for any $f \in \Gamma(X,
                \mathcal{O}_X)$ the map $k[x] \to \Gamma(X, \mathcal{O}_X)$
                defined by $x \mapsto f$ is not injective. Therefore $k \cong
                \Gamma(X, \mathcal{O}_X)$.
            }
    \end{enumerate}
\end{exercise}
\begin{exercise}{4}
    % https://stacks.math.columbia.edu/tag/01R5
    % https://stacks.math.columbia.edu/tag/01HP
    % https://stacks.math.columbia.edu/tag/01QQ
    \begin{enumerate}
        \item{
            Let $\{Z_i\}_{i \in I}$ be all closed subschemes $Z_i$ such that $f
            \colon X \to S$ factors through $Z_i$. Since equalizers exist in
            category of schemes, we take $\im(f)$ to be the equalizer $\eq(Z_i
            \rightrightarrows S)$.

            Let $\im(f)$ be the schematic image of $f \colon X \to S$. We have a
            factorization $f = i \circ f'$, where $f' \colon X \to \im(f)$ and
            $i \colon \im(f) \to S$ closed immersion.

            Then $\mathcal{O}_S \to f_* (\mathcal{O}_X)$ factors as
            $\mathcal{O}_S \to i_* \mathcal{O}_{\im(f)} \to f_*
            (\mathcal{O}_X)$. This shows that ideal sheaf of the image is indeed
            contained in the kernel of $f^\#$.

            Show that ideal sheaf of a closed immersion is indeed a
            quasi-coherent sheaf.

            Let $i \colon \im(f) \to S$ be a closed immersion. We have a
            surjection $\mathcal{O}_S \to i_* \mathcal{O}_Z$.

            Pick a point $x \in S$ and an affine open neighborhood $x \in U =
            \Spec(A) \subseteq S$. We obtain a map $A \to i_* \mathcal{O}_Z
            (U)$.

            Denote the kernel $I = \ker(A \to i_* \mathcal{O}_Z (U))$.

            Since $\mathcal{O}_S \to i_* \mathcal{O}_Z$ surjective implies $A
            \to i_* \mathcal{O}_Z (U)$ surjective, we have $i_* \mathcal{O}_Z
            (U) \cong A / I$.

            We want to show the ideal sheaf is equal to $\tilde{I}$. Pick any $f
            \in A$.

            Since $i_* \mathcal{O}_Z$ is a quasi-coherent
            $\mathcal{O}_S$-module, we have $i_* \mathcal{O}_Z(D(f)) = i_*
            \mathcal{O}_Z(U)[f^{-1}] = (A/I)[f^{-1}]$.

            Kernel of $A[f^{-1}] \to (A/I)[f^{-1}]$ is then $I[f^{-1}]$. This
            shows that ideal sheaf isomorphic to $\tilde{I}$ on $U$ and thus a
            quasi-coherent sheaf.

            Now we have to show it is in fact maximal such.

            Take any quasi-coherent ideal $M$ that factors through $\ker(f^\#)$.
            Sheaf $M$ induces a closed subscheme, locally on $\Spec(A) \subseteq
            S$ defined as a closed subscheme $V(M(U)) \subseteq \Spec(A)$.

            Sheaf $M$ induces a closed subscheme $Z$, through which $f$ factors,
            so we have $X \to Z \to S$. Therefore we obtain a closed inlusion $\im(f) \to
            Z$ which implies that on each affine ideal $Z$ is contained in
            $\im(f)$.
            }
        \item{
            If $f_* \mathcal{O}_X$ would be quasi-coherent, then the statement
            would hold, since kernels of quasi-coherent sheaves are
            quasi-coherent.

            However, let us assume now only that $f$ is quasi-compact.
            Since quasi-coherentness is a local property, we can assume that $S$ is affine.
            Using that $f$ is quasi-compact we have that
            \begin{align*}
                X=f^{-1}(Y)
            \end{align*}
            is compact, so we can write 
            \begin{align*}
                X=\bigcup_{i=1}^n U_i
            \end{align*}
            as a finite union of open affines.

            This gives a map 
            \begin{align*}
                f^{\prime}:\bigsqcup U_i \to X \to S.
            \end{align*}
            Now $f_*\mathcal{O}_X$ is a subsheaf of $f^{\prime}_*\mathcal{O}_{X^\prime}$,
            so 
            \begin{align*}
                \mathcal{I}=\ker(\mathcal{O}_S \to \mathcal{O}_{X^\prime}).
            \end{align*}
            Therefore, by \href{https://stacks.math.columbia.edu/tag/01LC}{stacks project}
            the sheaf of ideals is quasi-coherent in this case.

            Now the scheme-theoretic image is just the closed subscheme determined by $\mathcal{I}$.
            }
        \item{
            Denote $X = \bigsqcup_{n \geq 0} \Spec(\Z / p^n)$ and $f \colon X
            \to \Spec(Z)$.
            For every $n \geq 0$ we have $\Z/p^n$ which has a unique prime
            ideal, namely $(0)$ if $0 \leq n \leq 1$ and $(p)$ if $n \geq 2$.

            Every $\Spec(\Z/p^n)$ thus has one point.
            By looking at preimages of $\Z \to \Z / p^n$ we see that
            all of them are mapped to $(p) \in \Spec(\Z)$.
            So topologically the image should be $\{(p)\} \subseteq \Spec(\Z)$
            (we thought naively at the start).

            We use previous two parts to compute ideal sheaf, from which we can
            infer closed subscheme $\im(f)$.
            Ideal sheaf is quasi-coherent, so corresponds to some ideal $I
            \subseteq Z$.
            We also know ideal sheaf $\tilde{I}$ is contained in the kernel
            $\ker(\mathcal{O}_{\Spec(\Z)} \to f_* \mathcal{O}_X)$
            Applying this to global sections we get that $I$ must be
            contained in the kernel of $\ker(\Z \to \prod_{n \geq 0} \Z / p^n)$.
            But this map is injective, since every non-zero $a \in \Z$ will be non-zero
            in some $\Z / p^n$ for big enough $n$. Therefore $I$ is zero.

            Ideal $I$ vanishes everywhere, so $\im(f)$ is topologically
            homeomorphism on $\Spec(\Z)$ and $\mathcal{O}_{\Spec(\Z)} \to
            \mathcal{O}_{\im(f)}$ is surjective map with trivial kernel, so an
            isomorphism. Therefore $\im(f) = \Spec(\Z)$ (which is very
            surprising).
            }
    \end{enumerate}

\end{exercise}

\end{document}
