\newcommand{\sheet}{5}
\documentclass{article}
\usepackage[english, german]{babel}
\usepackage{amsthm,amssymb,amsmath,mathrsfs,mathtools}
\usepackage[shortlabels]{enumitem}
\usepackage{hyperref}
\usepackage{biblatex}
\usepackage{tikz}
\usepackage{tikz-cd}

% \usepackage[tmargin=1.25in,bmargin=1.25in,lmargin=1.2in,rmargin=1.2in]{geometry}


\newcommand{\C}{\mathbb{C}}
\newcommand{\R}{\mathbb{R}}
\newcommand{\N}{\mathbb{N}}
\newcommand{\Q}{\mathbb{Q}}
\newcommand{\Z}{\mathbb{Z}}
\newcommand{\Proj}{\mathbb{P}}
\newcommand{\Aff}{\mathbb{A}}

\DeclareMathOperator{\id}{id}
\DeclareMathOperator{\im}{im}
\DeclareMathOperator{\GL}{GL}
\DeclareMathOperator{\sgn}{sgn}
\DeclareMathOperator{\Tor}{Tor}
\DeclareMathOperator{\Sym}{Sym}
\DeclareMathOperator{\coker}{coker}
\DeclareMathOperator{\Quot}{Quot}
\DeclareMathOperator{\supp}{supp}
\DeclareMathOperator{\Hom}{Hom}
\DeclareMathOperator{\Spec}{Spec}
\DeclareMathOperator{\MinSpec}{MinSpec}
\DeclareMathOperator{\MaxSpec}{MaxSpec}
\DeclareMathOperator{\diag}{diag}
\DeclareMathOperator{\BL}{BL}
\DeclareMathOperator{\Ouv}{Ouv}
\DeclareMathOperator{\Sh}{Sh}
\DeclareMathOperator{\PSh}{PSh}
\DeclareMathOperator{\Eq}{Eq}
\DeclareMathOperator{\colim}{colim}
\DeclareMathOperator{\Pic}{Pic}
\DeclareMathOperator{\CL}{CL}
\DeclareMathOperator{\eq}{eq}
\DeclareMathOperator{\codim}{codim}

\newenvironment{exercise}[1] {
  \vspace{0.5cm}
  \noindent \textbf{Exercise~{#1}.}
} {
  \vspace{0.5cm}
}
\newenvironment{claim} {
  \par\noindent\textbf{Claim.}
} { }

\newenvironment{proof_claim} {
  \par\noindent\textbf{Proof of claim.}
} {
    \qed (of claim)
}

\title{Algebraic geometry 1\\Exercise sheet \sheet}
\author{Solutions by: Eric Rudolph and David Čadež}

\date{\today}


\begin{document}

\maketitle{}

\begin{exercise}{1}
    % https://stacks.math.columbia.edu/tag/0BX7
    \begin{enumerate}[label=(\roman*)]
        \item{
                Pick $f \in k(\eta) \setminus k$ ($\eta$ is the generic point of
                the curve $C$).
                We will show that $f$ can be viewed as a non-constant function
                $C \to \Proj^1_k$.

                First we will define $U \to \Aff^1_k$ on some open cofinite
                subscheme $U \in C$. Then we will postcompose this morphism with
                $\Aff^1_k \to \Proj^1_k$. At we will extend map $U \to
                \Proj^1_k$ to $\varphi \colon C \to \Proj^1_k$. By
                construction it will clear that the pullback of $t$ is $f$.
        
                Take some open affine $\Spec(A) \subseteq C$.
                % By definition $A$ is a $1$-dimensional domain.
                By integrality of $C$, $\Spec(A)$ contains the generic point, so
                we have $\Quot(A) = k(\eta)$.
                Write $f = \frac{f_1}{f_2} \in \Quot(A)$.
                Then $f$ is a section on $D(f_2)$.
                Because $A$ is $1$-dimensional, the complement $V(f_2)$ is finite.
                We use the bijection $\Gamma(\Spec(A), \mathcal{O}_X) \cong \Mor_k(\Spec(A),
                \Aff^1_k)$ to realize $f$ as a morphism $D(f_2) \to \Aff^1_k$.
                We do this on an finite open affine cover to obtain $f \colon U \to \Aff^1_k$, where $U
                \subseteq C$ is open, cofinite and dense. We can pick finite
                cover by quasicompactness and density of $U$ is by integrality.
                Note here that this definition on affine cover clearly agrees on
                intersections because we always had the same $f \in k(\eta)$.

                Now note that local rings at closed points on the curves are
                valuation rings. At this point we could recall a proposition
                from alg geo 1 that said that every dominant rational map
                between proper normal curves is represented by a morphism of
                schemes. By our definition curves are proper and smooth, so that
                would work. (And any morphism representing $U \to \Aff^1_k$
                would pullback $t$ to $f$.)
        
                Postcompose with $\Aff^1_k \to \Proj^1_k$ to obtain $f \colon U \to
                \Proj^1_k$.

                Now all we need to do is extend this $f$ to the whole $C$.

                We use valuative criteria to do it.
                \begin{equation*}
                    \begin{tikzcd}
                        & \Spec(k(\eta)) \arrow{r} \arrow{d} &
                        \Spec(\mathcal{O}_{C, x}) \arrow[d, dashrightarrow] \arrow{dr} \\
                        & U \arrow{r} & \Proj^1_k \arrow{r} & \Spec(k) \\
                    \end{tikzcd}
                \end{equation*}
                We obtain dashed arrow by valuative criteria for properness of
                $\Proj^1_k$ over $k$.
                Since $\Proj^1_k$ is of finite presentation over $k$, we can use
                the spreading argument to obtain an open neighbourhood $x \in V \subset C$
                and a morphism $V \to \Proj^1_k$ that extends
                $\Spec(\mathcal{O}_{C, x}) \to \Proj^1_k$.
                Morphisms $V \to \Proj^1_k$ and $U \to \Proj^1_k$ match on generic
                point and, since $C$ is separated, the equalizer is closed
                subscheme, so they must match on $U \cup V$ (here we use
                integrality of $C$).

                There are only finitely many points in $C \setminus U$, so we
                can do this process finitely many times and obtain $C \to
                \Proj^1_k$.

                Let us now argue why the pullback of $t$ is $f$. When defining
                $U \to \Aff^1_k$ we defined it using identification
                \begin{equation*}
                    \Hom_k(k[t], \Gamma(\Spec(A), \mathcal{O}_X)) \cong
                    \Mor_k(\Spec(A), \Aff^1_k).
                \end{equation*}
                So we defined the morphism by saying that the pullback of $t$
                should be $f$.
                Also notice that if $f$ is not in $k$, then $k[t] \to
                \Gamma(\Spec(A), \mathcal{O}_X))$ is injective, so the induced
                morphism maps generic point to generic point and is thus
                not constant.
            }
        \item{
                We simply follow the definitions
                \begin{equation*}
                    \operatorname{div} (\varphi^*(f)) = \sum_{x \in C_1}
                    \operatorname{ord}_x(\varphi^*(f)) [x]
                \end{equation*}
                and 
                \begin{equation*}
                    \begin{split}
                    \varphi^* \operatorname{div} (f)
                        &= \varphi^* \left( \sum_{y \in C_2}
                        \operatorname{ord}_y(f) [y] \right) \\
                        &= \sum_{y \in C_2} \operatorname{ord}_y(f) \left( \sum_{x \in
                        \varphi^{-1}(y)} e_x [x] \right) \\
                        &= \sum_{x \in C_1} \operatorname{ord}_{\varphi(x)}(f) 
                        e_x [x].
                    \end{split}
                \end{equation*}
                So we have to show that for any $x \in C_1$ we have
                $\operatorname{ord}_x(\varphi^* f) =
                \operatorname{ord}_{\varphi(x)}(f) e_x$.

                Denote $n = \operatorname{ord}_{\varphi(x)}(f)$.

                By normality we can pick a uniformizers
                \begin{equation*}
                        m_{C_2, \varphi(x)} = (t_{C_2, \varphi(x)})\quad
                        \text{and}\quad m_{C_1, x} = (t_{C_1, x}).
                \end{equation*}
                % By definition $f \in \mathcal{O}_{C_2, \varphi(x)}^\times
                % t^m_{C_2, \varphi(x)}$.

                The map $\varphi^\#_x \colon \mathcal{O}_{C_2, \varphi(x)} \to
                \mathcal{O}_{C_1, x}$ is a restriction of $\varphi^* \colon
                k(\eta_2) \to k(\eta_1)$, so we can use $\varphi^*$ for both.

                By the definition of ramification index $\varphi^*(m_{C_2,
                \varphi(x)}) \mathcal{O}_{C_1, x} = m_{C_1, x}^{e_x}$.
                So $\varphi^*(t_{C_2, \varphi(x)}) = \beta t_{C_1, x}^{e_x}$ for
                some $\beta \in \mathcal{O}^\times_{C_1, x}$.

                Write $f = \alpha t^n_{C_2, \varphi(x)}$ with $\alpha \in
                \mathcal{O}_{C_2, \varphi(x)}^\times$.
                Then
                \begin{equation*}
                    \varphi^*(f) = \varphi^*(\alpha) \varphi^*(t_{C_2,
                    \varphi(x)})^n = \varphi^*(\alpha) \beta^n t^{n
                    e_x}_{C_1, x} \in \mathcal{O}_{C_1, x}^\times t^{n
                    e_x}_{C_1, x}.
                \end{equation*}
                So $\operatorname{ord}_x(\varphi^*(f)) = n e_x$ which is what
                we needed to show.
            }
    \end{enumerate}
\end{exercise}


\end{document}
