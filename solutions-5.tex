\newcommand{\sheet}{5}
\documentclass{article}
\usepackage[english, german]{babel}
\usepackage{amsthm,amssymb,amsmath,mathrsfs,mathtools}
\usepackage[shortlabels]{enumitem}
\usepackage{hyperref}
\usepackage{biblatex}
\usepackage{tikz}
\usepackage{tikz-cd}

% \usepackage[tmargin=1.25in,bmargin=1.25in,lmargin=1.2in,rmargin=1.2in]{geometry}


\newcommand{\C}{\mathbb{C}}
\newcommand{\R}{\mathbb{R}}
\newcommand{\N}{\mathbb{N}}
\newcommand{\Q}{\mathbb{Q}}
\newcommand{\Z}{\mathbb{Z}}
\newcommand{\Proj}{\mathbb{P}}
\newcommand{\Aff}{\mathbb{A}}

\DeclareMathOperator{\id}{id}
\DeclareMathOperator{\im}{im}
\DeclareMathOperator{\GL}{GL}
\DeclareMathOperator{\sgn}{sgn}
\DeclareMathOperator{\Tor}{Tor}
\DeclareMathOperator{\Sym}{Sym}
\DeclareMathOperator{\coker}{coker}
\DeclareMathOperator{\Quot}{Quot}
\DeclareMathOperator{\supp}{supp}
\DeclareMathOperator{\Hom}{Hom}
\DeclareMathOperator{\Spec}{Spec}
\DeclareMathOperator{\MinSpec}{MinSpec}
\DeclareMathOperator{\MaxSpec}{MaxSpec}
\DeclareMathOperator{\diag}{diag}
\DeclareMathOperator{\BL}{BL}
\DeclareMathOperator{\Ouv}{Ouv}
\DeclareMathOperator{\Sh}{Sh}
\DeclareMathOperator{\PSh}{PSh}
\DeclareMathOperator{\Eq}{Eq}
\DeclareMathOperator{\colim}{colim}
\DeclareMathOperator{\Pic}{Pic}
\DeclareMathOperator{\CL}{CL}
\DeclareMathOperator{\eq}{eq}
\DeclareMathOperator{\codim}{codim}

\newenvironment{exercise}[1] {
  \vspace{0.5cm}
  \noindent \textbf{Exercise~{#1}.}
} {
  \vspace{0.5cm}
}
\newenvironment{claim} {
  \par\noindent\textbf{Claim.}
} { }

\newenvironment{proof_claim} {
  \par\noindent\textbf{Proof of claim.}
} {
    \qed (of claim)
}

\title{Algebraic geometry 1\\Exercise sheet \sheet}
\author{Solutions by: Eric Rudolph and David Čadež}

\date{\today}


\begin{document}

\maketitle{}

\begin{exercise}{1}
    \begin{enumerate}
        \item Define 
        \begin{align*}
            X:=U_1\coprod U_2/\!{\sim},
        \end{align*}
        where $x\sim y$ if $x=\varphi(y)$ and for $i\in \lbrace
        1,2\rbrace$
        \begin{align*}
            \pi_i&:U_i\to X\\
            x&\mapsto \bar{x}.
        \end{align*}
        We can now give $X$ the structure of a topological 
        space by defining a subset $U\subset X$ to be open
        if $\pi^{-1}(U)\in$ are open in $U_1$.\\
        Notice, that $\pi_i$ are homeomorphic onto open subsets
        of X. This will become important later.
        Next we want to define a structure sheaf on $X$ that
        behaves well with restricting to $U_i.$\\
        For $U\subset X$ open, let 
        \begin{align*}
            \mathcal{O}_X(U):=ker(\mathcal{O}_{U_1}(\pi^{-1}(U))\oplus \mathcal{O}_{U_2}(\pi^{-1}(U)) \to \mathcal{O}_{U_1}(\pi^{-1}(U) \cap U_1)\\
            (x,y) \mapsto x_{\mid \pi^{-1}(U)\cap V_1}-\varphi^\sharp (\pi_2^{-1}(U)\cap V_2)(y_{\mid \pi_2^{-1}(U)\cap V_2})),
        \end{align*}
            where the substraction in the above term comes from the group
            structure of $\mathcal{O}_{U_1}(\pi^{-1}(U)\cap V_1)$. This is of course
            a group again, as the kernel of a ring map.\\
            We conclude, that $(X,\mathcal{O}_X)$ is a scheme, because 
            $X=\pi_1(U_1)\cup \pi _2(U_2)$  can be covered by affine
            schemes using the cover from $U_1$ and $U_2$ and since by construction
            of the structure sheaf $\mathcal{O}_{X \mid U_1}=\mathcal{O}_{x_i}.$
            Here we finally used, as promised, that $\pi_i$ are homeomorphisms
            onto open subsets of $X$.
    \end{enumerate}
\end{exercise}
\begin{exercise}{3}
    \begin{enumerate}
        \item Remember, that the 
        contravariant functor
         $A\mapsto (Spec(A),\mathcal{O}_{Spec(A)})$
         is an equivalence of categories, meaning that 
         $\mathcal{O}^{op}$ is equivalent to $Rings$. Then
         the affine case follows from the Yoneda embedding.\\
         How to show general case?
    \end{enumerate}
\end{exercise}

\end{document}
