\newcommand{\sheet}{2}
\documentclass{article}
\usepackage[english, german]{babel}
\usepackage{amsthm,amssymb,amsmath,mathrsfs,mathtools}
\usepackage[shortlabels]{enumitem}
\usepackage{tikz}
\usepackage{tikz-cd}

% \usepackage[tmargin=1.25in,bmargin=1.25in,lmargin=1.2in,rmargin=1.2in]{geometry}


\newcommand{\C}{\mathbb{C}}
\newcommand{\R}{\mathbb{R}}
\newcommand{\N}{\mathbb{N}}
\newcommand{\Q}{\mathbb{Q}}
\newcommand{\Z}{\mathbb{Z}}

\DeclareMathOperator{\id}{id}
\DeclareMathOperator{\im}{im}
\DeclareMathOperator{\GL}{GL}
\DeclareMathOperator{\sgn}{sgn}
\DeclareMathOperator{\Tor}{Tor}
\DeclareMathOperator{\Sym}{Sym}
\DeclareMathOperator{\coker}{coker}
\DeclareMathOperator{\Quot}{Quot}
\DeclareMathOperator{\supp}{supp}
\DeclareMathOperator{\Hom}{Hom}
\DeclareMathOperator{\Spec}{Spec}
\DeclareMathOperator{\MinSpec}{MinSpec}
\DeclareMathOperator{\diag}{diag}


\newenvironment{exercise}[1] {
  \vspace{0.5cm}
  \noindent \textbf{Exercise~{#1}.}
} {
  \vspace{0.5cm}
}
\newenvironment{claim} {
  \noindent \textbf{Claim.}
} {
}

\title{Algebraic geometry 1\\Exercise sheet \sheet}
\author{Solutions by: Eric Rudolph and David Čadež}

\date{\today}


\begin{document}

\maketitle

\begin{exercise}{1}
    Let $I = (f_1, \dots, f_r) \subseteq k[x_1, \dots, x_n]$ be an ideal and $X
    = V(I) \subseteq \mathbb{A}^n(k)$ be its vanishing locus.
    \begin{enumerate}
        \item{} We have to show
            \[
                \overline{X} = \bigcap \{V^+(h) \mid h\ \text{homogenous}, h(X) = 0\} =
                V(\{ \tilde{g} \mid g \in I \}).
            \]
            Pick any homogenous $h$ that vanishes on $X \subseteq \mathbb{A}^n \subseteq \mathbb{P}^n$. By letting $x_{n+1} =
            1$ we see that $h(x_1, \dots, x_n, 1) \in \sqrt{I}$, since it vanishes on
            $X$. Therefore we have $l \in \N$ such that $h(x_1, \dots, x_n, 1)^l
            = \sum_{i = 1}^{r} \alpha_i f_i$ for some $\alpha_i \in k[x_1,
            \dots, x_n]$.

            { \color{gray} Homogenization is almost a bijection between
            homogenous polynomials in $n+1$ variables and all polynomials in $n$
            variables. It is bijective between homogenous polynomials in $n+1$
            variables 
            that are not divisible by $x_{n+1}$ and polynomials in $n$
            variables. The ``inverse'' to homogenization would be the letting
            $x_{n+1} = 1$. }

            Adding these two together we get $x^{ml}_{n+1} \widetilde{\sum_{i =
            1}^{r} \alpha_i f_i} = h^l$. Therefore
            \[
                h = \sum^r_{i = 1} \beta_i \widetilde{f_i}
            \]
            and thus 
            $\bigcap \{V^+(h) \mid h\ \text{homogenous}, h(X) = 0\} \supseteq
            V(\{ \tilde{g} \mid g \in I \})$.

        \item{} We calculate $\overline{X}$ using the first part of the exercise and take the numbering of the
        varieties from the first exercise sheet:\\
        \begin{align*}
            \overline{X_1}\cap \mathbb{P}^2_k \backslash \mathbb{A}^2_k&=V(x^2-xz)\cap V(z)=(0:1:0),\\
            \overline{X_2}\cap \mathbb{P}^2_k \backslash \mathbb{A}^2_k&=V(z^2-xy)\cap V(z)=(1:0:0)\cup (0:1:0),\\
            \overline{X_3}\cap \mathbb{P}^2_k \backslash \mathbb{A}^2_k&=V(xy)\cap V(z)=(1:0:0)\cup (0:1:0),\\
            \overline{X_4}\cap \mathbb{P}^2_k \backslash \mathbb{A}^2_k&=V(x^2-xy)\cap V(z)=(0:1:0),\\
            \overline{X_5}\cap \mathbb{P}^2_k \backslash \mathbb{A}^2_k&=V(x)\cap V(z)=(0:1:0).
        \end{align*}

        \item{} 
    \end{enumerate}Define $f(x,y):=xy-1$ and $g(x,y):=xy$. Then we have 
    $$X=V(f,g)=\emptyset,$$
    and since $\emptyset$ is closed in projective space as the vanishing 
    set of a non-zero constant, we have $\overline{X}=\emptyset$.
    However, 
    $$
    (0:1:0)\in V^+(xy-z,xy)=V^+(\tilde{f}, \tilde{g}),
    $$
    proving that in this case $\overline{X}\neq V(\tilde{f},\tilde{g})$.
\end{exercise}

\begin{exercise}{3}
    \begin{enumerate}
        \item{} We define a closed subset $V(f - g) \subseteq X$, which contains
            the open set $\mathcal{U} \subseteq V(f - g)$. By definition the
            completent $\mathcal{U}^C$ is closed and $V(f - g) \cup
            \mathcal{U}^C = X$. Since $X$ is irreducible and $\mathcal{U}$ is
            non-empty, we have $f = g$.

        \item{} Lets show first that $\chi \colon A \mapsto \chi_A(A)$ vanishes
            on diagonalizable matrices with pairwise different eigenvalues: For
            $A = T D T^{-1}$ we have
            \begin{equation*}
                \chi_A(A) = \sum^n_{i = 0} \alpha_i A^i = \sum^n_{i = 0}
                \alpha_i (T D T^{-1})^i = T \left( \sum^n_{i = 0} \alpha_i  D^i
                \right) T^{-1}.
            \end{equation*}
            Denote $D = \diag(d_1, \dots, d_n)$ and notice that
            \begin{equation*}
                \sum^n_{i = 0} \alpha_i  D^i = 
                \sum^n_{i = 0} \diag(\alpha_i d^i_1, \dots, \alpha_i d^i_n) = 
                \diag(\chi_A(d_1), \dots, \chi_A(d_n)) = 0
            \end{equation*}
            because characteristic polynomial vanishes on eigenvalues $d_i$.

            Since $\mathbb{A}^{n \times n}(L)$ is irreducible
            and $\chi$ vanishes on open subset of it, namely on diagonalizable
            matrices with pairwise different eigenvalues, $\chi$ must be $0$ on
            whole $\mathbb{A}^{n \times n}(L)$.
    \end{enumerate}
\end{exercise}

\begin{exercise}{4}
    \begin{enumerate}
        \item First, we note that points in projective space are closed.
        To see this, take some point $a=(a_0:...:a_n)\in \mathbb{P}^n_k$.
        Then $a=V(\lbrace f\in k[x_0,...,x_n]: f=x_ia_j-x_ja_i \rbrace).$
        Now, we let $X\subset \mathbb{P}^n_k$ be some quasi-projective variety and
        $x,y\in X$ with $x\neq y$. Define $U:=\lbrace x \rbrace^c$, which is open as
        the complement of a closed set and by definition fulfills the properties
        $x\not \in U$ and $y \in U$.
     \end{enumerate}
\end{exercise}
\end{document}
