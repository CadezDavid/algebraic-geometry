\newcommand{\sheet}{8}
\documentclass{article}
\usepackage[english, german]{babel}
\usepackage{amsthm,amssymb,amsmath,mathrsfs,mathtools}
\usepackage[shortlabels]{enumitem}
\usepackage{tikz}
\usepackage{tikz-cd}

% \usepackage[tmargin=1.25in,bmargin=1.25in,lmargin=1.2in,rmargin=1.2in]{geometry}


\newcommand{\C}{\mathbb{C}}
\newcommand{\R}{\mathbb{R}}
\newcommand{\N}{\mathbb{N}}
\newcommand{\Q}{\mathbb{Q}}
\newcommand{\Z}{\mathbb{Z}}

\DeclareMathOperator{\id}{id}
\DeclareMathOperator{\im}{im}
\DeclareMathOperator{\GL}{GL}
\DeclareMathOperator{\sgn}{sgn}
\DeclareMathOperator{\Tor}{Tor}
\DeclareMathOperator{\Sym}{Sym}
\DeclareMathOperator{\coker}{coker}
\DeclareMathOperator{\Quot}{Quot}
\DeclareMathOperator{\supp}{supp}
\DeclareMathOperator{\Hom}{Hom}
\DeclareMathOperator{\Spec}{Spec}
\DeclareMathOperator{\MinSpec}{MinSpec}
\DeclareMathOperator{\diag}{diag}


\newenvironment{exercise}[1] {
  \vspace{0.5cm}
  \noindent \textbf{Exercise~{#1}.}
} {
  \vspace{0.5cm}
}
\newenvironment{claim} {
  \noindent \textbf{Claim.}
} {
}

\title{Algebraic geometry 1\\Exercise sheet \sheet}
\author{Solutions by: Eric Rudolph and David Čadež}

\date{\today}


\begin{document}

\maketitle{}

\begin{exercise}{3}
    \begin{enumerate}
        \item{
            }
        \item{
                We can show that $j_!$ is exact. We know it preserves
                epimorphisms because it has a right adjoint. But monomorphisms
                are preserved because they can be checked on stalks.

                Let $0 \to A \to B \to C \to 0$ be exact. We want to show $0 \to
                j_! A \to j_! B \to j_! C \to 0$ is exact. For $x \in V$ we
                clearly have $(j_! A)_x = A_x$ and for $x \not V$ we have $(j_!
                A)_x = 0$ (for this we use explicit definition of sheafification
                from alggeo1). So $j_! A \to j_! B$ is a monomorphism.

                Now we show that $j^*$ preserves injectives.

                Let $F$ be injective $\mathcal{O}_X$-module.
                Let $0 \to A \to B \to C \to 0$ be exact sequence of
                $\mathcal{O}_V$-modules.
                Then $0 \to j_! A \to j_! B \to j_! C \to 0$ is exact.
                By injectiveness of $F$ we have exact sequence $0 \to \Hom(j_!
                A, F) \to \Hom(j_! B, F) \to \Hom(j_! C, F) \to 0$ is exact.
                Then we use that $j_!$ and $j^*$ are adjoint pair and we get
                that $0 \to \Hom(A, j^* F) \to \Hom(B, j^* F) \to \Hom(C, j^* F)
                \to 0$ is exact. So $\Hom(_, j^* F)$ preserves exact sequences,
                which is what we wanted to show.
            }
        \item{
                We can take $X = \Spec(\Z)$ and $V = X \setminus \{ p \}$ for
                prime $p$. We will show that $j_! \mathcal{O}_V$ has no
                global sections.

                Denote the presheaf defined in the exercise by $F$.

                At alggeo1 we constructed sheafification of a
                presheaf $F$ explicitly as the sheaf
                \begin{equation*}
                        \tilde{F} \colon U \mapsto \{ (s_x)_{x \in U} \in \Pi_{x
                        \in U} F_x \mid 
                        (s_x)_{x \in U}\ \text{satisfies condition below} \} \\
                \end{equation*}
                for all $x \in U$ there exists a neighbourhood $V_x$ and $t \in
                F(V_x)$ such that for all $y \in V_x$ we have $s_y = t_y$.

                So global sections of $j_! \mathcal{O}_V$ are $(s_x)_{x \in X}$
                (that satisfy the condition). But for any neighbourhood $V_p$ of
                $p$ we have $F(V_p) = 0$, so there has to be an open set $V_p$
                such that $s_y = 0$ for all $y \in V_p$. But then $s_y = 0$ for
                all $y \in X$, since any non-zero section of $(j_!
                \mathcal{O}_V) \mid_V$ is zero only on finitely many points.
                
                The map $0 = (j_! \mathcal{O}_V)(X)[p^{-1}] \to \Z[p^{-1}]$ is
                is therefore not an isomorphism, and $j_! \mathcal{O}_V$ not a
                quasi-coherent sheaf.
            }
    \end{enumerate}
\end{exercise}

\end{document}
