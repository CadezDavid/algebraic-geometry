\newcommand{\sheet}{2}
\documentclass{article}
\usepackage[english, german]{babel}
\usepackage{amsthm,amssymb,amsmath,mathrsfs,mathtools}
\usepackage[shortlabels]{enumitem}
\usepackage{hyperref}
\usepackage{biblatex}
\usepackage{tikz}
\usepackage{tikz-cd}

% \usepackage[tmargin=1.25in,bmargin=1.25in,lmargin=1.2in,rmargin=1.2in]{geometry}


\newcommand{\C}{\mathbb{C}}
\newcommand{\R}{\mathbb{R}}
\newcommand{\N}{\mathbb{N}}
\newcommand{\Q}{\mathbb{Q}}
\newcommand{\Z}{\mathbb{Z}}
\newcommand{\Proj}{\mathbb{P}}
\newcommand{\Aff}{\mathbb{A}}

\DeclareMathOperator{\id}{id}
\DeclareMathOperator{\im}{im}
\DeclareMathOperator{\GL}{GL}
\DeclareMathOperator{\sgn}{sgn}
\DeclareMathOperator{\Tor}{Tor}
\DeclareMathOperator{\Sym}{Sym}
\DeclareMathOperator{\coker}{coker}
\DeclareMathOperator{\Quot}{Quot}
\DeclareMathOperator{\supp}{supp}
\DeclareMathOperator{\Hom}{Hom}
\DeclareMathOperator{\Spec}{Spec}
\DeclareMathOperator{\MinSpec}{MinSpec}
\DeclareMathOperator{\MaxSpec}{MaxSpec}
\DeclareMathOperator{\diag}{diag}
\DeclareMathOperator{\BL}{BL}
\DeclareMathOperator{\Ouv}{Ouv}
\DeclareMathOperator{\Sh}{Sh}
\DeclareMathOperator{\PSh}{PSh}
\DeclareMathOperator{\Eq}{Eq}
\DeclareMathOperator{\colim}{colim}
\DeclareMathOperator{\Pic}{Pic}
\DeclareMathOperator{\CL}{CL}
\DeclareMathOperator{\eq}{eq}
\DeclareMathOperator{\codim}{codim}

\newenvironment{exercise}[1] {
  \vspace{0.5cm}
  \noindent \textbf{Exercise~{#1}.}
} {
  \vspace{0.5cm}
}
\newenvironment{claim} {
  \par\noindent\textbf{Claim.}
} { }

\newenvironment{proof_claim} {
  \par\noindent\textbf{Proof of claim.}
} {
    \qed (of claim)
}

\title{Algebraic geometry 1\\Exercise sheet \sheet}
\author{Solutions by: Eric Rudolph and David Čadež}

\date{\today}


\begin{document}

\maketitle

\begin{exercise}{1}
    Let $I = (f_1, \dots, f_r) \subseteq k[x_1, \dots, x_n]$ be an ideal and $X
    = V(I) \subseteq \mathbb{A}^n(k)$ be its vanishing locus.
    \begin{enumerate}
        \item{} We have to show
            \[
                \overline{X} = \bigcap \{V^+(h) \mid h\ \text{homogenous}, h(X) = 0\} =
                V(\{ \tilde{g} \mid g \in I \}).
            \]
            Pick any homogenous $h$ that vanishes on $X \subseteq \mathbb{A}^n
            \subseteq \mathbb{P}^n$. By letting $x_{n+1} = 1$ we see that
            $h(x_1, \dots, x_n, 1) \in \sqrt{I}$, since it vanishes on $X$.
            Therefore we have $l \in \N$ such that $h(x_1, \dots, x_n, 1)^l =
            \sum_{i = 1}^{r} \alpha_i f_i$ for some $\alpha_i \in k[x_1, \dots,
            x_n]$.

            { \color{gray} Homogenization is almost a bijection between
            homogenous polynomials in $n+1$ variables and all polynomials in $n$
            variables. It is bijective between homogenous polynomials in $n+1$
            variables 
            that are not divisible by $x_{n+1}$ and polynomials in $n$
            variables. The ``inverse'' to homogenization would be the letting
            $x_{n+1} = 1$. }

            Adding these two together we get $x^{ml}_{n+1} \widetilde{\sum_{i =
            1}^{r} \alpha_i f_i} = h^l$. Therefore
            \[
                h = \sum^r_{i = 1} \beta_i \widetilde{f_i}
            \]
            and thus $\bigcap \{V^+(h) \mid h\ \text{homogenous}, h(X) = 0\}
            \supseteq V(\{ \tilde{g} \mid g \in I \})$.

        \item{} 

        \item{} 
    \end{enumerate}
\end{exercise}

\begin{exercise}{2}
    Let $F \in k[x, y, z]$ be a non-zero homogenous polynomial of degree 2.
    \begin{enumerate}
        \item{} Let us first prove the statement in the hint. So let
            \begin{equation*}
                T (x, y) = (a x + b y + u, c x + d y + v)
                % T \begin{bmatrix}
                %     x \\ y
                % \end{bmatrix} =
                % \begin{bmatrix}
                %     a & b \\
                %     c & d
                % \end{bmatrix}
                % \begin{bmatrix}
                %     x \\ y
                % \end{bmatrix} +
                % \begin{bmatrix}
                %     u \\ v
                % \end{bmatrix},
            \end{equation*}
            with $ad - cb \not= 0$, be an affine linear transformation of
            $\mathbb{A}^2(k)$.
            We can extend it to an automorphism of $\mathbb{P}^2(k)$ by defining
            \begin{equation*}
                \widetilde{T} (x : y : z) = (a x + b y + u z : c x + d y + v z : z)
                % \widetilde{T} \colon{} \mathbb{P}^2(k) \rightarrow
                % \mathbb{P}^2(k),
                % \begin{bmatrix}
                %     x \\ y \\ z
                % \end{bmatrix}
                % \mapsto
                % \begin{bmatrix}
                %     a & b & 0 \\
                %     c & d & 0 \\
                %     0 & 0 & 1
                % \end{bmatrix}
                % \begin{bmatrix}
                %     x \\ y \\ z
                % \end{bmatrix}
                % + z \begin{bmatrix}
                %     u \\ v \\ 0
                % \end{bmatrix}
            \end{equation*}
            It is an extension of $T$, because we have $\widetilde{T} \circ i =
            i \circ T$ where $i \colon \mathbb{A}^2(k) \rightarrow
            \mathbb{P}^2(k)$ is given by $(x, y) \mapsto (x : y : 1)$.
            Mapping $\widetilde{T}$ is bijective with inverse
            \[
                \widetilde{T}^{-1}(x : y : z) = ( dx - by - (du - bv) z : - cx +
                ay + (cu - av)z : (ad - bc)z).
            \]
            This proves that $\widetilde{T}$ is an automorphism of
            $\mathbb{P}^2(k)$.

            Lets consider the polynomial $F$ now. We looks at its zero-set on
            $\mathbb{A}^2(k)$, which is exactly $V(F) \cap V(z - 1) = V(F, z - 1) = V(F(x, y, 1))$.

            This set could be empty, in which case $F(x, y, 1)$ must be a
            non-zero constant (because $k$ is algebraically closed). A general
            form of $F$ would be $F(x, y, z) = a_1 x^2 + a_2 y^2 + a_3 z^2 + a_4
            xy + a_5 xz + a_6 yz$. If we set $z = 1$ and $F$ becomes a non-zero
            constant, we get that $a_3$ is a non-zero constant and other $a_i$
            are $0$. So $F(x, y, z) = a z^2$, which means $V(F)$ is isomorphic
            to the projective line $V(x) = V(x^2)$.

            Suppose now $V := V(F(x, y, 1)) \subseteq \mathbb{A}^2(k)$ is not
            empty. Note that $F(x, y, 1)$ is of the form we treated in exercise
            $3$ of sheet $1$. Then we can use an affine linear transformation
            $T$ on $\mathbb{A}^2(k)$ so that $V \subseteq \mathbb{A}^2(k)$
            assumes the forms of one of the $5$ curves we treated in exercise
            $3$ sheet $1$, namely $V(x), V(y - x^2), V(xy), V(xy - 1), V(x(x -
            1))$. We showed earlier that his transformation can be extended to
            an isomorphism $\widetilde{T}$ of $\mathbb{P}^2(k)$.

            Loosely speaking, we might have ``lost some information'' when we
            focused on $\mathbb{A}^2(k) \cap V(F)$, but it turns out that we
            didnt, if we take into account that the curve we started with was
            non-zero homogenous polynomial of degree $2$. We are able to
            ``retrieve all of it back'' by taking the closure and remembering
            that original curve was of degree $2$.

            Since the curve $V(F)$ is by definition of topology closed, the
            curve $\widetilde{T}(V(F)) \subseteq \mathbb{P}^2(k)$ will also be
            closed. By construction we have on of the following cases
            \begin{itemize}
                \item{} $V(x) \cap \mathbb{A}^2(k) = V(x^2) \cap \mathbb{A}^2(k)
                    \subseteq \widetilde{T}(V(F))$
                \item{} $V(xy) \cap \mathbb{A}^2(k) \subseteq
                    \widetilde{T}(V(F))$
                \item{} $V(xy - 1) \cap \mathbb{A}^2(k) \subseteq
                    \widetilde{T}(V(F))$
                \item{} $V(y - x^2) \cap \mathbb{A}^2(k) \subseteq
                    \widetilde{T}(V(F))$
                \item{} $V(x(x - 1)) \cap \mathbb{A}^2(k) \subseteq
                    \widetilde{T}(V(F))$
            \end{itemize}
            In each case we can take the closure of the left side and, since
            right side is closed, the inclusion will still hold. We use $1$st
            exercise of this sheet to calculate the closure by homogenizing the
            polynomial. Namely we get one of the following cases
            \begin{itemize}
                \item{} $V(x) = V(x^2) \subseteq \widetilde{T}(V(F))$
                \item{} $V(xy) \subseteq \widetilde{T}(V(F))$
                \item{} $V(xy - z^2) \subseteq \widetilde{T}(V(F))$
                \item{} $V(yz - x^2) \subseteq \widetilde{T}(V(F))$
                \item{} $V(x^2 - xz) \subseteq \widetilde{T}(V(F))$
            \end{itemize}
            These inclusions are in fact equalities,
    \end{enumerate}
\end{exercise}

\begin{exercise}{3}
    \begin{enumerate}
        \item{} We define a closed subset $V(f - g) \subseteq X$, which contains
            the open set $\mathcal{U} \subseteq V(f - g)$. By definition the
            completent $\mathcal{U}^C$ is closed and $V(f - g) \cup
            \mathcal{U}^C = X$. Since $X$ is irreducible and $\mathcal{U}$ is
            non-empty, we have $f = g$.

        \item{} Lets show first that $\chi \colon A \mapsto \chi_A(A)$ vanishes
            on diagonalizable matrices with pairwise different eigenvalues: For
            $A = T D T^{-1}$ we have
            \begin{equation*}
                \chi_A(A) = \sum^n_{i = 0} \alpha_i A^i = \sum^n_{i = 0}
                \alpha_i (T D T^{-1})^i = T \left( \sum^n_{i = 0} \alpha_i  D^i
                \right) T^{-1}.
            \end{equation*}
            Denote $D = \diag(d_1, \dots, d_n)$ and notice that
            \begin{equation*}
                \sum^n_{i = 0} \alpha_i  D^i = 
                \sum^n_{i = 0} \diag(\alpha_i d^i_1, \dots, \alpha_i d^i_n) = 
                \diag(\chi_A(d_1), \dots, \chi_A(d_n)) = 0
            \end{equation*}
            because characteristic polynomial vanishes on eigenvalues $d_i$.

            Since $\mathbb{A}^{n \times n}(L)$ is irreducible
            and $\chi$ vanishes on open subset of it, namely on diagonalizable
            matrices with pairwise different eigenvalues, $\chi$ must be $0$ on
            whole $\mathbb{A}^{n \times n}(L)$.
    \end{enumerate}
\end{exercise}

\end{document}
