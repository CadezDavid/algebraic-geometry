\newcommand{\sheet}{10}
\documentclass{article}
\usepackage[english, german]{babel}
\usepackage{amsthm,amssymb,amsmath,mathrsfs,mathtools}
\usepackage[shortlabels]{enumitem}
\usepackage{hyperref}
\usepackage{biblatex}
\usepackage{tikz}
\usepackage{tikz-cd}

% \usepackage[tmargin=1.25in,bmargin=1.25in,lmargin=1.2in,rmargin=1.2in]{geometry}


\newcommand{\C}{\mathbb{C}}
\newcommand{\R}{\mathbb{R}}
\newcommand{\N}{\mathbb{N}}
\newcommand{\Q}{\mathbb{Q}}
\newcommand{\Z}{\mathbb{Z}}
\newcommand{\Proj}{\mathbb{P}}
\newcommand{\Aff}{\mathbb{A}}

\DeclareMathOperator{\id}{id}
\DeclareMathOperator{\im}{im}
\DeclareMathOperator{\GL}{GL}
\DeclareMathOperator{\sgn}{sgn}
\DeclareMathOperator{\Tor}{Tor}
\DeclareMathOperator{\Sym}{Sym}
\DeclareMathOperator{\coker}{coker}
\DeclareMathOperator{\Quot}{Quot}
\DeclareMathOperator{\supp}{supp}
\DeclareMathOperator{\Hom}{Hom}
\DeclareMathOperator{\Spec}{Spec}
\DeclareMathOperator{\MinSpec}{MinSpec}
\DeclareMathOperator{\MaxSpec}{MaxSpec}
\DeclareMathOperator{\diag}{diag}
\DeclareMathOperator{\BL}{BL}
\DeclareMathOperator{\Ouv}{Ouv}
\DeclareMathOperator{\Sh}{Sh}
\DeclareMathOperator{\PSh}{PSh}
\DeclareMathOperator{\Eq}{Eq}
\DeclareMathOperator{\colim}{colim}
\DeclareMathOperator{\Pic}{Pic}
\DeclareMathOperator{\CL}{CL}
\DeclareMathOperator{\eq}{eq}
\DeclareMathOperator{\codim}{codim}

\newenvironment{exercise}[1] {
  \vspace{0.5cm}
  \noindent \textbf{Exercise~{#1}.}
} {
  \vspace{0.5cm}
}
\newenvironment{claim} {
  \par\noindent\textbf{Claim.}
} { }

\newenvironment{proof_claim} {
  \par\noindent\textbf{Proof of claim.}
} {
    \qed (of claim)
}

\title{Algebraic geometry 1\\Exercise sheet \sheet}
\author{Solutions by: Eric Rudolph and David Čadež}

\date{\today}


\begin{document}

\maketitle{}

\begin{exercise}{1}
    \begin{enumerate}
        \item{
                Since $X$ is closed and irreducible, it is of the form $X =
                \overline{\{p_0\}}$ for some (Eric thinks unique) $p_0 \in \Aff^n_k$.
                That means $X \cong \Spec(k[x_1, \dots x_n] / p_0)$.
                Denote $A = k[x_1, \dots x_n] / p_0$.

                By assumption there is a chain of specializations $p_0 \subset
                \dots \subset p_d$ inside $X$.

                Let $Z \subseteq X \cap V(f_1, \dots, f_r)$ be a irreducible
                component. Thus it is the closure of a minimal prime ideal in $A
                / (f_1, \dots, f_r)$.

                By Krull's principal ideal theorem we have $\dim(A / (f_1,
                \dots, f_r)) \geq d - r$.

                Denote minimal prime ideals in $A / (f_1, \dots, f_r)$ with
                $q_1, \dots q_l$.

                (Eric thinks that there is a unique generic point here again, since $X$ is sober, so 
                there should only be one of these prime ideals, right?)
                
                We argue that
                \begin{equation*}
                    \dim(A / (f_1, \dots, f_r) / q_j) = \dim(A / (f_1, \dots, f_r)).
                \end{equation*}
                for any $j$.

                That follows from $A$ being catenary.
                If there existed a maximal chain in $A / (f_1, \dots, f_r)$ that
                starts at $q_j$ we could simply extend it below to get a
                maximal chain in $A$. Since all maximal chains in $A$ are of the
                same length, we get that all maximal chains in $A / (f_1, \dots,
                f_r)$ are also of the same length.

                Since $Z$ is an irreducible component, we have $Z =
                \overline{\{q_i\}} \subseteq \Spec(A / (f_1, \dots, f_r))$.

                Therefore any maximal chain in $Z$ is exactly as long as the
                longest chain in $A / (f_1, \dots, f_r)$. And the longest chain
                in $A / (f_1, \dots, f_r)$ is at least of length $d - r$.
            }
        \item{
                The diagonal $\triangle \subseteq \Aff^n_k \times \Aff^n_k$ can
                be defined as $V(x_i \otimes x_i \mid i = 1, \dots n) \subseteq
                \Spec(k[x_1, \dots, x_n] \otimes_\Z k[x_1, \dots, x_n])$.

                (Should there not be a minus instead of $\otimes$ in the above expression?)

                Using exercise above we get that any irreducible component of
                $X \cap Y \cong (X \times Y) \cap V(x_i \otimes x_i \mid i = 1,
                \dots n)$ has dimension at least $d + e - n$.
            }
        \item{
                Let $\tilde{X} = \overline{f^{-1}(X)}$ and $\tilde{Y} = \overline{f^{-1}(Y)}$ as in
                the hint.

                We have $\dim(\tilde{X}) = d + 1$ and $\dim(\tilde{Y}) = e + 1$.
                By the previous exercise we have $\dim(\tilde{X} \cap \tilde{Y})
                \geq d + 1 + e + 1 - (n + 1) = (d + e - n) + 1 \geq 1$.

                Therefore there there exists $0 \not= x \in \tilde{X} \cap
                \tilde{Y}$.

                Questions from Eric:
                
                Why is $\tilde{X}$ irreducible (to be able to use part 2) and why does the dimension 
                increase by 1 when we go to affine space?
            }
    \end{enumerate}
\end{exercise}

\begin{exercise}{2}
    \begin{enumerate}
        \item{
                Take $x \in |X|$. If $x \notin f(|Y|)$, we can find an open
                $U_x$ such that $f^{-1} (U_x) = \emptyset$. So assume $x \in
                f(|Y|)$. Then look at $f^{-1}(x)$. Take an open affine $V_x
                \subseteq |Y|$ with $f^{-1} (x) \in V_x$. Since $f$ is
                homeomorphism on its image, we have can take $U_x = f(V_x)$ an
                affine neighborhood of $x$ such that $f^{-1}(U_x) = V_x$ is
                affine.
            }
        \item{
            }
    \end{enumerate}
\end{exercise}


    % https://math.stackexchange.com/questions/3774859/global-sections-of-a-proper-variety-over-an-arbitrary-field
\begin{exercise}{3}
    \begin{enumerate}
        \item{
                We have a map $k \rightarrow \Gamma(X, \mathcal{O}_X)$.
                For any $f \in \Gamma(X, \mathcal{O}_X)$ we can define $k[x]
                \rightarrow \Gamma(X, \mathcal{O}_X)$ by $x \mapsto f$.

                Showing that $g(X)$ does not contain the generic point of
                $\Aff^1_k$ is equivalent to showing that $k[x] \to
                \Gamma(X, \mathcal{O}_X)$ is not injective.

                We have a composition $k \to k[x] \to \Gamma(X, \mathcal{O}_X)$.
                So also $X \to \Aff^1_k \to \Spec(k)$.

                Map $X \to \Spec(k)$ is proper.

                Map $\Aff^1_k \to \Spec(k)$ is separated, since it is a map of
                affine schemes. (Follows from the fact that $k[x] \otimes_k k[x]
                \to k[x]$ is surjective, and thus $\Aff^1_k \to \Aff^1_k
                \times_k \Aff^1_k$ a closed immersion.)

                Thus by the proposition from the lectures, the map $g \colon X \to
                \Aff^1_k$ is proper. In particular it is closed. Since $X$ is
                connected, the image $g(X)$ must be connected as well.

                Using the hint, we can postcompose to obtain $X \to \Aff^1_k \to
                \Proj^1_k$.
                Now the conclusion should be that the image of $X$ in
                $\Proj^1_k$ is also closed. Since $\Aff^1_k \subseteq \Proj^1_k$
                is not closed, the image of $X$ in $\Aff^1_k$ can also not be
                closed. Therefore it must be a single point.

                Since we did not exactly understand why should $X \to \Proj^1_k$
                be closed, we decided to rather show that $X \to \Aff^1_k$
                cannot be surjective, as that would imply $\Aff^1_k$ being
                universally closed over $\Spec(k)$ (which we've shown during the
                lectures to be false). 

                Instead of doing it abstractly, we can show that $X \to
                \Aff^1_k$ being surjective would imply $\Aff^2_k \to \Aff^1_k$
                being closed.

                By the universal property of $\Aff^2_k$ we get a map $X \to
                \Aff^2_k$, induced by $X \to \Aff^1_k$.
                So we have a map $X \to \Aff^2_k \to \Aff^1_k$.
                Denote $\alpha \colon X \to \Aff^2_k$ and $\beta \colon \Aff^2_k
                \to \Aff^1_k$.
                If $\alpha$ would be surjective, then for any $U \subseteq
                \Aff^2_k$ we would have $(\beta \circ \alpha)(\alpha^{-1}(U)) =
                \beta(U)$. Since $\beta \circ \alpha$ is closed by assumption,
                this would prove that $\beta$ is closed. That is not true, so
                $\beta \circ \alpha$ is not surjective.

                We've shown that the image of $X \to \Aff^1_k$ is a single
                point. Since this point is closed, it is not the generic point.
                This shows that $k[x] \to \Gamma(X, \mathcal{O}_X)$ induced by
                $f \in \Gamma(X, \mathcal{O}_X)$ is not injective.
            }
        \item{
                We have a map $k \to \Gamma(X, \mathcal{O}_X)$. It cannot be
                $0$, since $X$ is locally finite type over $\Spec(k)$. So it is
                injective.

                It is also surjective, since for any $f \in \Gamma(X,
                \mathcal{O}_X)$ the map $k[x] \to \Gamma(X, \mathcal{O}_X)$
                defined by $x \mapsto f$ is not injective. Therefore $k \cong
                \Gamma(X, \mathcal{O}_X)$.
            }
    \end{enumerate}
\end{exercise}

\end{document}
