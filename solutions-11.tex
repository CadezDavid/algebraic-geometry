\newcommand{\sheet}{11}
\documentclass{article}
\usepackage[english, german]{babel}
\usepackage{amsthm,amssymb,amsmath,mathrsfs,mathtools}
\usepackage[shortlabels]{enumitem}
\usepackage{hyperref}
\usepackage{biblatex}
\usepackage{tikz}
\usepackage{tikz-cd}

% \usepackage[tmargin=1.25in,bmargin=1.25in,lmargin=1.2in,rmargin=1.2in]{geometry}


\newcommand{\C}{\mathbb{C}}
\newcommand{\R}{\mathbb{R}}
\newcommand{\N}{\mathbb{N}}
\newcommand{\Q}{\mathbb{Q}}
\newcommand{\Z}{\mathbb{Z}}
\newcommand{\Proj}{\mathbb{P}}
\newcommand{\Aff}{\mathbb{A}}

\DeclareMathOperator{\id}{id}
\DeclareMathOperator{\im}{im}
\DeclareMathOperator{\GL}{GL}
\DeclareMathOperator{\sgn}{sgn}
\DeclareMathOperator{\Tor}{Tor}
\DeclareMathOperator{\Sym}{Sym}
\DeclareMathOperator{\coker}{coker}
\DeclareMathOperator{\Quot}{Quot}
\DeclareMathOperator{\supp}{supp}
\DeclareMathOperator{\Hom}{Hom}
\DeclareMathOperator{\Spec}{Spec}
\DeclareMathOperator{\MinSpec}{MinSpec}
\DeclareMathOperator{\MaxSpec}{MaxSpec}
\DeclareMathOperator{\diag}{diag}
\DeclareMathOperator{\BL}{BL}
\DeclareMathOperator{\Ouv}{Ouv}
\DeclareMathOperator{\Sh}{Sh}
\DeclareMathOperator{\PSh}{PSh}
\DeclareMathOperator{\Eq}{Eq}
\DeclareMathOperator{\colim}{colim}
\DeclareMathOperator{\Pic}{Pic}
\DeclareMathOperator{\CL}{CL}
\DeclareMathOperator{\eq}{eq}
\DeclareMathOperator{\codim}{codim}

\newenvironment{exercise}[1] {
  \vspace{0.5cm}
  \noindent \textbf{Exercise~{#1}.}
} {
  \vspace{0.5cm}
}
\newenvironment{claim} {
  \par\noindent\textbf{Claim.}
} { }

\newenvironment{proof_claim} {
  \par\noindent\textbf{Proof of claim.}
} {
    \qed (of claim)
}

\title{Algebraic geometry 1\\Exercise sheet \sheet}
\author{Solutions by: Eric Rudolph and David Čadež}

\date{\today}


\begin{document}

\maketitle{}

\begin{exercise}{1}
    We claim that $\mathcal{O}_{X,x}$ is a normal, local noetherian domain of dimension at most one. 
    Normality is by definition of normality of $X$. Stalks are of course local rings by definition of locally ringed space.
    Noetherian comes from the assumption that $X$ is of finite type over $k$. Also, 
    \begin{align*}
        \dim(X)=\sup_{x\in X}\dim(\mathcal{O}_{X,x}).
    \end{align*}
    Hence, $\dim(\mathcal{O}_{X,x})\le 1$.

    If $\dim(\mathcal{O}_{X,x})= 0$, then $\mathcal{O}_{X,x}$ is of course regular as the only prime ideal is $(0)$.

    If $\dim(\mathcal{O}_{X,x})= 1$, then we know from the lecture that the maximal ideal $m \subset \mathcal{O}_{X,x}$
    is principal (and not zero) and hence $\mathcal{O}_{X,x}$ is regular.
\end{exercise}


\begin{exercise}{3}
    \begin{enumerate}
        \item Since $k$ is algebraically closed, the only irreducible polynomials $f\in k[x,y]$ are of degree $1$.
        
        Hence, we can write 
        \begin{align*}
            f_r=l_1  \dots  l_r,
        \end{align*}
        where $l_i\in k[x,y]$ is of degree $1$. From the assumption that $f_r$ is homogenous it follows that the $l_i$ are homogenous.

        Therefore, we can write
        \begin{align*}
            Z=V(f_r)=V(l_1 \dots  l_r)=\cup_i V(l_i)
        \end{align*}
        and since $V(l_i)$ is a line through the origin, $Z$ can be written as the finite union of lines through the origin.

        \item 
        We first want to prove that $\dim(\mathcal{O}_{X,(x,y)})=1$ for all $r$. The prime ideals $p$ in this ring 
        fulfil $(f)\subset p \subset (x,y)$. Remember that we can write down these prime ideals explicitely as in 
        \href{https://math.stackexchange.com/questions/56916/what-do-prime-ideals-in-kx-y-look-like}{"What do primes of k[x,y] look like"}.
        From this the claim follows.

        We know that $\dim_k(m_{\mathcal{O}_{X,(x,y)}}/m_{\mathcal{O}_{X,(x,y)}}^2)$ is the number of generators 
        of $m_{\mathcal{O}_{X,(x,y)}}$.

        Now if $r=1$, then we can write $f=g(x,y)x+h(x,y)y$ and w.l.o.g. we have $g(0,0)=1$, meaning that it is invertible (after localizing
        ). Therefore $f=x+h(x,y)y$, so $y\mid x$ meaning $(x,y)=(y)$
        On the other hand, if $r>1$, then $x\nmid y$ and $y\nmid x$ meaning that $m$ is no principal ideal showing that $X$ is 
        singular at zero in this case. (This can be seen by writing $f$ as $f=x^2h_1(x,y)+xyh_2(x,y)+y^2h_3(x,y))$.

        \item By part two of this exercise, all the schemes have a singular point at the origin. I don't know why they do not have singular
        points anywhere else.
    \end{enumerate}

\end{exercise}

\begin{exercise}{4}
    Intuitively, since
    \begin{align*}
        \Gamma(X,\mathcal{O}_X)\subset \Gamma(U,\mathcal{O}_X)
     \end{align*}
     one can always restrict sections on $X$ to sections on $U$. In this exercise we basically show that under these
     special conditions we can also uniquely extend a section on $U$ to one on the whole $X$.


    We will show that the restriction map on global sections is an isomorphism, i.e. that
    \begin{align*}
        \Gamma(X,\mathcal{O}_X)\cong \Gamma(U,\mathcal{O}_X).
    \end{align*}
    This immediately implies the claim of the exercise by definition of vector bundle (if rings are isomorphic
    then so is their finite sum).

    By definition, $\codim(Z)\geq 2.$ (where $\codim(Z)=\inf_{z\in Z}(\mathcal{O}_{X,z}))$ as defined in Görz Wedhorn.

    Take $Y\subset X$ an irreducible component of codimension $1$. By construction, $Y$ and $U$ intersect nontrivially
     (either $Z\cap Y=\emptyset$ or $Y\subsetneq Z$). 
     In particular, $U$ contains the generic point $\mu$ of $Y$. 
     This means that $\Gamma(U,\mathcal{O}_X)\subset \mathcal{O}_{X,\mu}$.

     The hint tells us that $A$ is the intersection of all localizations of $A$ at prime ideals $p$ of height $1$.
     Those prime ideals of height $1$ correspond to irreducible closed subsets of codimension $1$. Hence, we have just shown that
     \begin{align*}
        \Gamma(U,\mathcal{O}_X)\subset A = \Gamma(X,\mathcal{O}_X).
     \end{align*}
     The other inclusion is immediate.
\end{exercise}

\end{document}
