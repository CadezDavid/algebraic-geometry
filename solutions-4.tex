\newcommand{\sheet}{3}
\documentclass{article}
\usepackage[english, german]{babel}
\usepackage{amsthm,amssymb,amsmath,mathrsfs,mathtools}
\usepackage[shortlabels]{enumitem}
\usepackage{hyperref}
\usepackage{biblatex}
\usepackage{tikz}
\usepackage{tikz-cd}

% \usepackage[tmargin=1.25in,bmargin=1.25in,lmargin=1.2in,rmargin=1.2in]{geometry}


\newcommand{\C}{\mathbb{C}}
\newcommand{\R}{\mathbb{R}}
\newcommand{\N}{\mathbb{N}}
\newcommand{\Q}{\mathbb{Q}}
\newcommand{\Z}{\mathbb{Z}}
\newcommand{\Proj}{\mathbb{P}}
\newcommand{\Aff}{\mathbb{A}}

\DeclareMathOperator{\id}{id}
\DeclareMathOperator{\im}{im}
\DeclareMathOperator{\GL}{GL}
\DeclareMathOperator{\sgn}{sgn}
\DeclareMathOperator{\Tor}{Tor}
\DeclareMathOperator{\Sym}{Sym}
\DeclareMathOperator{\coker}{coker}
\DeclareMathOperator{\Quot}{Quot}
\DeclareMathOperator{\supp}{supp}
\DeclareMathOperator{\Hom}{Hom}
\DeclareMathOperator{\Spec}{Spec}
\DeclareMathOperator{\MinSpec}{MinSpec}
\DeclareMathOperator{\MaxSpec}{MaxSpec}
\DeclareMathOperator{\diag}{diag}
\DeclareMathOperator{\BL}{BL}
\DeclareMathOperator{\Ouv}{Ouv}
\DeclareMathOperator{\Sh}{Sh}
\DeclareMathOperator{\PSh}{PSh}
\DeclareMathOperator{\Eq}{Eq}
\DeclareMathOperator{\colim}{colim}
\DeclareMathOperator{\Pic}{Pic}
\DeclareMathOperator{\CL}{CL}
\DeclareMathOperator{\eq}{eq}
\DeclareMathOperator{\codim}{codim}

\newenvironment{exercise}[1] {
  \vspace{0.5cm}
  \noindent \textbf{Exercise~{#1}.}
} {
  \vspace{0.5cm}
}
\newenvironment{claim} {
  \par\noindent\textbf{Claim.}
} { }

\newenvironment{proof_claim} {
  \par\noindent\textbf{Proof of claim.}
} {
    \qed (of claim)
}

\title{Algebraic geometry 1\\Exercise sheet \sheet}
\author{Solutions by: Eric Rudolph and David Čadež}

\date{\today}


\begin{document}

\maketitle{}

\begin{exercise}{1}
    \begin{enumerate}
        \item Let $X$ be a finite set that is irreducible with 
        respect to some topology $\mathcal{F}$ on $X$. Then we get
        $\mid \mathcal{F} \mid<\infty$ and since finite unions of
        closed sets are closed again, we get that
        \begin{align*}
            X^{\prime}:=\bigcup_{U \subsetneq X \text{closed}}{U}
        \end{align*}
        is closed in X. Since $X$ is by assumption irreducible,
        $X\neq X^{\prime}$, so we can pick $x_0\in X \backslash X^{\prime}$,
        which is by construction generic. For the second part of the exercise we use 
        part 2 of Hochster's Theorem. As a finite set, $X$ is 
        quasicompact and as a basis $\mathcal{B}$ consisting of quasicompact 
        open sets stable under finite intersections take all of the open sets. \\
        It remains to show that $X$ is sober. We need to check that every irreducible
        subset of $X$ has a unique generic point. The existence of a generic point comes
        from part of of this exercise. \\
        Uniqueness of this point is due to the fact
        that generic points in $T_0$ spaces are unique if they exist, which follows directly
        from the definition of $T_0$. 
    \end{enumerate}
    
    \begin{enumerate}
        \item{Let $X$ be a finite irreducible topological space. Since
            \begin{equation*}
                X = \bigcup_{x \in X} \overline{\{x\}}
            \end{equation*}
            is a finite decomposition in closed sets, we must have
            $\overline{\{x\}} = X$ for some $x \in X$. This $x$ is a generic
            point of $X$.

            If we additionally assumed $X$ is $T_0$, then this point $x$ would
            be unique, since in a $T_0$ space we have $\overline{\{x\}} \not=
            \overline{\{y\}}$ for $x \not= y$. Also in a finite space the
            conditions of quasicompactness and the basis being stable under
            finite intersections are clearly fulfilled. So finite $T_0$ spaces
            are spectral.
            }
        \item{
                Let us first describe what $\Spec(\Z)$ looks like. It is a PID
                with prime ideals being those $(a)$ for which $a \in \Z$ is a
                prime number or $a = 0$. So
                \begin{equation*}
                    \Spec(\Z) = \{ (p) \mid p\ \text{prime} \} \cup \{ (0) \}.
                \end{equation*}
                In this space $(0)$ is a unique generic point.

                Closed sets in $\Spec(\Z)$ are by definiton
                \begin{equation*}
                    V((a)) = \{ (p) \in \Spec(\Z) \mid (a) \subseteq (p) \} = 
                    \{ (p) \in \Spec(\Z) \mid p\ \text{divides}\ a \}.
                \end{equation*}
                So if $a = \Pi_{i} p^{k_i}_i$, then $V((a)) = {\{ (p_i) \}}_i
                \subseteq \Spec(\Z)$. Since any $a \in \Z$ is only divisible by
                finitely many prime numbers, we get the finite completement
                topology on $\Spec(\Z) \setminus \{(0)\}$.

                Adding generic point $(0)$ to $\Spec(\Z) \setminus \{(0)\}$ is
                actually the construction $X \rightarrow X^{\text{sob}}$ which
                we did last week.


                % Let $\mathbb{P} = \{ p_1, p_2, \dots \} \subseteq \N$ be an
                % ``ordered set'' of prime numbers.

                % Construct an inverse limit
                % \begin{equation*}
                %     F_i := \{ (p_1), \dots, (p_i) \} \cup \{ (0) \}
                % \end{equation*}
                % for $i \in \N$ and
                % \begin{equation*}
                %     f_i \colon F_i \xhookrightarrow{} F_{i+1}
                % \end{equation*}
                % being subset inclusions. We still have to say what topologies
                % $F_i$ have and then show that $f_i$ are continuous. Define
                % closed sets on $F_i$ to be $A \subseteq F_i$ with $(0) \notin
                % A$, and the whole $F_i$. This basis for closed sets is clearly
                % stable under finite unions.

                % Now show that $f_i$ are continuous. Take $A \subseteq F_{i+1}$
                % closed. Then $f^{-1}_i(A) = A \cap F_i$. If $(0) \in A$, then
                % $(0) \in A \cap F_i$ and therefore $A = F_{i+1}$. Thus
                % $f^{-1}_i(A) = F_i$ is closed. If $(0) \notin A$, then $(0)
                % \notin A \cap F_i$ and thus $f^{-1}_i((0)) = A \cap F_i
                % \subseteq F_i$ also closed.

                % \begin{tikzcd}
                %     &F_1 \arrow{drr}{g_1} \arrow{r}{f_1} &F_2 \arrow{dr}{g_2}
                %     \arrow{r}{f_2} &F_3 \arrow{d}{g_3} \arrow{r}{f_3} &F_4
                %     \arrow{dl}{g_4} \arrow{r}{f_4} &\dots \arrow{dll}{g_i}\\
                %     && &F
                % \end{tikzcd}

                % Now we claim that $\Spec(\Z)$ is the inverse limit of this
                % diagram $(F_i, f_i)$.

                % Define $g_i \colon F_i \rightarrow \Spec(\Z)$ to be the
                % inclusions. Commutativity of the diagram ($g_i = g_{i+1} \circ
                % f_i$) is clear, since all maps are inclusions.

                % Take some other topological space $Y$ with $h_i \colon F_i
                % \rightarrow Y$. Define the map $h \colon \Spec(\Z) \rightarrow
                % Y$ by mapping $(p_i) \mapsto h_i((p_i))$, i.e. we identify
                % $(p_i)$ with $(p_i) \in F_i \subseteq F_{i+1} \subseteq \dots$
                % and then map it by $h_i \colon F_i \rightarrow Y$.

                % To check continuity take a closed set $A \subseteq Y$. By
                % definition $h^{-1}(A) = \cap $
            }
    \end{enumerate}
\end{exercise}

\begin{exercise}{2}
    Denote $A = \lim A_i$, $B = \lim B_i$ and $C = \lim C_i$. Also denote maps
    $A_i \rightarrow A$ with $f_i$, $B_i \rightarrow B$ with $g_i$ and $C_i
    \rightarrow C$ with $h_i$.

    By composing $\alpha_i$ and $g_i$ we get $A_i \rightarrow B$ defined as $g_i
    \circ \alpha_i$. Then by the definition of a colimit we have a unique map $\alpha:A
    \rightarrow B$, such that $g_i \circ \alpha_i = \alpha
    \circ f_i$. In the same way we obtain $\beta \colon B \rightarrow C$. With
    these definitions the whole diagram commutes.
\end{exercise}

\begin{exercise}{3}
    \begin{enumerate}
        \item Let $\mathcal{F}, \mathcal{G}$ be presheaves on $X$ and $(\phi_U)_{U\in Ouv(X)}, 
        (\psi_U)_{U\in Ouv(X)}$ morphism of presheaves from $\mathcal{F}$ to $\mathcal{G}$.\\
        We define for $U\in Ouv(X)$
        \begin{align*}
            (\phi+\psi)(U):=\phi(U)+\psi(U).
        \end{align*}
        This is indeed a map of presheaves again, because we can
        restrict $\phi$ and $\psi$ independently.\\
        The zero object in this category is the presheaf that sends 
        every open set $U\in Ouv(X)$ to the trivial group $(0,+)$. 
        It is inital and terminal, because group maps send $0$ to $0$.
        For this reason the category of presheaves is an additive category
        and since morphism of sheaves are defined using their underlying
        presheaves, also the category of sheaves form an additive category.\\
        We still need to check that kernels and cokernels exist in the 
        category of Sheaves, so from now on $\mathcal{F},\mathcal{G}$
        and maps between them we be of sheaves.\\
        Define 
        \begin{align*}
            (ker\phi)(U):=ker\phi(U).
        \end{align*}
        We check that this is indeed a presheaf. To do that one has 
        to check that restrictions behave well.\\
        Next, one checks that this is indeed a sheaf.
    \end{enumerate}
\end{exercise}
\end{document}
