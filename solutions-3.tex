\newcommand{\sheet}{3}
\documentclass{article}
\usepackage[english, german]{babel}
\usepackage{amsthm,amssymb,amsmath,mathrsfs,mathtools}
\usepackage[shortlabels]{enumitem}
\usepackage{hyperref}
\usepackage{biblatex}
\usepackage{tikz}
\usepackage{tikz-cd}

% \usepackage[tmargin=1.25in,bmargin=1.25in,lmargin=1.2in,rmargin=1.2in]{geometry}


\newcommand{\C}{\mathbb{C}}
\newcommand{\R}{\mathbb{R}}
\newcommand{\N}{\mathbb{N}}
\newcommand{\Q}{\mathbb{Q}}
\newcommand{\Z}{\mathbb{Z}}
\newcommand{\Proj}{\mathbb{P}}
\newcommand{\Aff}{\mathbb{A}}

\DeclareMathOperator{\id}{id}
\DeclareMathOperator{\im}{im}
\DeclareMathOperator{\GL}{GL}
\DeclareMathOperator{\sgn}{sgn}
\DeclareMathOperator{\Tor}{Tor}
\DeclareMathOperator{\Sym}{Sym}
\DeclareMathOperator{\coker}{coker}
\DeclareMathOperator{\Quot}{Quot}
\DeclareMathOperator{\supp}{supp}
\DeclareMathOperator{\Hom}{Hom}
\DeclareMathOperator{\Spec}{Spec}
\DeclareMathOperator{\MinSpec}{MinSpec}
\DeclareMathOperator{\MaxSpec}{MaxSpec}
\DeclareMathOperator{\diag}{diag}
\DeclareMathOperator{\BL}{BL}
\DeclareMathOperator{\Ouv}{Ouv}
\DeclareMathOperator{\Sh}{Sh}
\DeclareMathOperator{\PSh}{PSh}
\DeclareMathOperator{\Eq}{Eq}
\DeclareMathOperator{\colim}{colim}
\DeclareMathOperator{\Pic}{Pic}
\DeclareMathOperator{\CL}{CL}
\DeclareMathOperator{\eq}{eq}
\DeclareMathOperator{\codim}{codim}

\newenvironment{exercise}[1] {
  \vspace{0.5cm}
  \noindent \textbf{Exercise~{#1}.}
} {
  \vspace{0.5cm}
}
\newenvironment{claim} {
  \par\noindent\textbf{Claim.}
} { }

\newenvironment{proof_claim} {
  \par\noindent\textbf{Proof of claim.}
} {
    \qed (of claim)
}

\title{Algebraic geometry 1\\Exercise sheet \sheet}
\author{Solutions by: Eric Rudolph and David Čadež}

\date{\today}


\begin{document}

\maketitle{}

\begin{exercise}{1}
    \begin{enumerate}
        \item Define 
        \begin{align*}
            \pi^{-1}:U &\longrightarrow \pi^{-1}(U) \\
            (x_1,...,x_n)  &\mapsto (x_1,...,x_n)[x_1:...:x_n].
        \end{align*}
        This is well-defined, because by definition of $U$, not all
        $x_i$ can be zero at the same time, so $[x_1:...:x_n]$ is actually a
        point in projective space. We also have 
            $(x_1,...,x_n)[x_1:...:x_n]\in Z$ for $(x_1,...,x_n)\in U$,
            because $x_ix_j=x_jx_i$ for all $1\leq i,j \leq n.$
            To see injectivity of $\pi^{-1}$, let
            $(x_1,...,x_n)\in U$ with $x_j\not = 0$. Then we have $y_j\not = 0$, because
            if we assume $x_j\not =0$ and $y_j=0$, then for some $y_i\not =0$ 
            (which exists since $[y_1:...:y_n]$ is a point in projective space) 
            we have $0\not = x_jy_i= x_iy_j=0.$
            Therefore, we can just set $y_j=1$. Then 
            \begin{align*}
                x_iy_j=x_jy_i \implies y_1=\frac{x_1y_j}{x_j}=\frac{x_1}{x_j},
            \end{align*}
            showing that all the $y_i$ are fixed up to a scalar after fixing all the $x_i$.
        \item    
            Define 
            \begin{align*}
                \phi: V_i &\to \mathbb{A}_n^k\\
                (x,y)&\mapsto (\frac{x_1}{y_i},\dots, x_i,\dots,\frac{x_n}{y_i}),
            \end{align*}
            where the inverse map is given by 
            \begin{align*}
                \phi^{-1}:\mathbb{A}_n^k &\to V_i\\
                (x_1,\dots,x_n)&\mapsto (x_1x_i,\dots,x_i,\dots,x_nx_i)[x_1:\dots:x_{i-1}:1:\dots:x_n].
            \end{align*}
    \end{enumerate}
\end{exercise}

\begin{exercise}{2}
    The part which to me seemed the hardest was to calculate the closure of
    $\pi^{-1}(Y \setminus \{t\})$ in $Z$. So to for that we look at
    $\pi^{-1}(Y)$ and decompose it into irreducible components. Or actually we
    first cover it $V_i$ and then look at them inside each $V_i$. In these two
    cases it decomposed into nice irreducible components, one of which is the
    blow-up and the other whole $\pi^{-1}(t)$.
    \begin{enumerate}
        \item{Let $Y = V(x^2_1 - x^3_2) \subseteq \mathbb{A}^2(k)$. We look at
            \begin{equation}
                \pi^{-1}(Y) = \{ ((x_1, x_2), [y_1 : y_2]) \mid x_1 y_2 = x_2
                y_1, x^2_1 - x^3_2 = 0 \}.
            \end{equation}
            We can cover it with $V_i$ ($i=1,2$). Lets first look inside $V_1
            \cong \mathbb{A}^2(k) \times \mathbb{A}^1(k)$, where $y_1 = 1$.
            Equations then become $x_1 y_2 = x_2$ and $x^2_1 (1 - x_1
            y^3_2) = 0$. The latter equation can be decomposed, so we get two
            closed subsets:
            \begin{itemize}
                \item{
                        $\{ x_1 = x_2 = 0 \} \subseteq
                        V_1 \subseteq \mathbb{A}^2(k) \times \mathbb{P}^1(k)$
                    }
                \item{
                        $\{ x_1 y^3_2 - 1 = 0, x_2 y^2_2 - 1 = 0 \} \subseteq
                        V_1 \subseteq \mathbb{A}^2(k) \times \mathbb{P}^1(k)$
                    }
            \end{itemize}
            First one lies in $\pi^{-1}(t)$, so $\{ x_1 = x_2 = 0 \} \cap
            \pi^{-1}(Y \setminus \{t\}) = \emptyset$. Therefore
            \begin{equation}
                \pi^{-1}(Y \setminus \{t\}) \cap V_1 = \{ x_1 y^3_2 - 1 = 0, x_2
                y^2_2 - 1 = 0, x_1 \not= 0, x_2 \not= 0 \} \cap V_1
            \end{equation}
            Therefore taking the closure inside $V_1$:
            \begin{equation}
                \BL_t(Y) \cap V_1 = \{ x_1 y^3_2 - 1 = 0, x_2 y^2_2 - 1 = 0 \}
                \cap V_1.
            \end{equation}
            Before definition a morphism $\BL_t(Y) \rightarrow \mathbb{A}^1(k)$,
            lets look at $\BL_t(Y) \cap V_2$. Similar as before we set $y_2 = 1$
            and get that $\pi^{-1}(Y) \cap V_2$ is made up of two closed subsets
            \begin{itemize}
                \item{
                        $\{ x_1 = x_2 = 0 \} \subseteq
                        V_2 \subseteq \mathbb{A}^2(k) \times \mathbb{P}^1(k)$
                    }
                \item{
                        $\{ x_1 = y^3_1, x_2 = y^2_1 \} \subseteq
                        V_2 \subseteq \mathbb{A}^2(k) \times \mathbb{P}^1(k)$
                    }
            \end{itemize}
            This intersection contains more information, since it also contains
            $\BL_t(Y) \cap \pi^{-1}(t)$.

            Define a morphism $\phi \colon \BL_t(Y) \rightarrow \mathbb{A}^2(k)$

            }
        \item{}
    \end{enumerate}
\end{exercise}

\begin{exercise}{4}
    \begin{enumerate}
        \item{
                Lets first prove that $V_{U}$ are stable under
                intersections:
                \begin{claim}
                    Take $U, W \subseteq X$ open subsets. Then $V_{U \cap W} =
                    V_U \cap V_W$.
                \end{claim}
                \begin{proof_claim}
                    Inclusion $V_{U \cap W} \subseteq V_U \cap
                    V_W$ is clear.

                    For the other inclusion take $Z \in V_U \cap V_W$. By
                    definition $Z \cap U \not= \emptyset$ and $Z \cap V \not=
                    \emptyset$. Suppose $Z \cap (U \cap V) = \emptyset$. Then
                    $(Z \cap U)^c \cup (Z \cap V)^c = X$. But since $Z$ is
                    irreducible, and is covered by $U^c \cup V^c$, we must have
                    (WLOG) $Z \subseteq U^c$. That is in contradiction with
                    $Z \cap U \not= \emptyset$. 
                \end{proof_claim}

                It also behaves well under unions:
                \begin{align*}
                    V_{U \cup W} &= \{ Z\ \text{cl. irred.} \mid Z \cap (U \cup
                    W) \not= \emptyset \} \\
                    &= \{ Z\ \text{cl. irred.} \mid (Z \cap U) \not= \emptyset
                    \ \text{or}\ (Z \cap W) \not= \emptyset \} \\
                    &= \{ Z\ \text{cl. irred.} \mid (Z \cap U) \not= \emptyset
                    \} \cup \{ Z\ \text{cl. irred.} \mid (Z \cap W) \not=
                    \emptyset \} \\
                    &= V_U \cup V_W
                \end{align*}
                and practically same argument applies to infinite unions.

                Therefore every open subset of $X^{\text{sob}}$ can be written
                as $V_U$ for some open $U \subseteq X$ (in general it could've
                been just a base of topology, but this shows it is the whole
                topology).

                \begin{claim}
                    If $V_{U_1} = V_{U_2}$ then $U_1 = U_2$.
                \end{claim}
                \begin{proof_claim}
                    Suppose $x \in U_1 \setminus U_2$, then $\overline{\{x\}}
                    \in V_{U_1} \setminus V_{U_2}$. This proves the claim.
                \end{proof_claim}
                Next claim is a direct consequence of one above.
                \begin{claim}
                    Closed irreducible subsets of $X^{\text{sob}}$ are exactly
                    $V^c_U$ for open $U \subseteq X$ such that (closed) subset
                    $U^c \subseteq X$ is irreducible.
                \end{claim}
                \begin{proof_claim}
                    Let $V^c_U$ be irreducible and $U^c = U^c_1 \cup U^c_2
                    \subseteq X$. Then $V_U = V_{U_1 \cap U_2} = V_{U_1} \cap
                    V_{U_2}$ and thus $V^c_U = V^c_{U_1} \cup V_{U_2}$. Since
                    $V^c_U$ is irreducible, we must have $V^c_U = V^c_{U_1}$ and
                    thus $U = U_1$, which proves irreducibility of $U^c$.

                    For the other implication, let $U^c$ be irreducible and
                    $V^c_U = V^c_{U_1} \cup V^c_{U_2}$. Then $U_1 \cap U_2 = U$.
                    Since $U^c$ is irreducible, we must have (WLOG) $U_1 = U$
                    and therefore $V_U = V_{U_1}$.
                \end{proof_claim}

                Let us show $X^{\text{sob}}$ is sober. Let $V_U^c$ be closed
                irreducible. Then by last claim $U^c$ is closed and irreducible.
                The set $U^c$ is the generic point with $\overline{\{U^c\}} =
                V_U^c$. The inclusion $\overline{\{U^c\}} \subseteq V_U^c$ is
                obvious, because $V^c_U$ contains the point $U^c$ and is a
                closed set. For the other inclusion take a closed set that
                $V^c_W$ that contains $U^c$. That means $U^c \cap W = \emptyset$
                and thus $W \subseteq U$. Then we have $V_W \subseteq V_U$ and
                $V^c_U \subseteq V^c_W$. This proves that $V^c_U$ is the closure
                of the point $U^c$.
            }
        \item{
                Define
                \begin{align*}
                    h \colon X^{\text{sob}} &\rightarrow Z \\
                    W &\mapsto \text{unique generic point of}\ \overline{g(W)}.
                \end{align*}
                Note that: a continuous image of an irreducible set is
                irreducible and the closure of an irreducible set is
                irreducible.  
                So $\overline{g(W)} \subseteq Z$ is a closed irreducible subset and thus
                has a unique generic point in $Z$.

                Let's now prove $g = h \circ f$. Take $x \in X$. We have to
                prove $g(x)$ is the unique generic point of
                $\overline{g\left(\overline{\{x\}}\right)}$. Clearly $g(x) \in
                \overline{g\left(\overline{\{x\}}\right)}$. Take any closed $W
                \subseteq Z$ with $g(x) \in W$. Then, by definition, $x \in
                g^{-1}(W)$. Because $g^{-1}(W)$ is closed, also
                $\overline{\{x\}} \subseteq g^{-1}(W)$. So
                $g\left(\overline{\{x\}}\right) \subseteq W$. But since $W$ is
                closed we have $\overline{g\left(\overline{\{x\}}\right)}
                \subseteq W$. This proves that $g(x)$ is indeed a generic point
                of $\overline{g\left(\overline{\{x\}}\right)}$. So we have $g =
                h \circ f$.

                To prove $h$ is continuous we take an open set $U \subseteq Z$.
                we want to see that $h^{-1}(U)$ is open. Since $g^{-1}(U) =
                f^{-1}(h^{-1}(U))$ is open and $f^{-1}$ induces a bijection of
                open sets, the set $h^{-1}(U)$ is open as well. So $h$ is
                continuous.

                We should also argue why $h$ is unique. Take $h, h' \colon
                X^{\text{sob}} \rightarrow Z$ both continuous and satisfying $g
                = h \circ f = h' \circ f$. Pick any closed irreducible $W
                \subseteq X$. Suppose $h(W) \not= h'(W)$. WLOG there exists open
                $U \subseteq Z$ such that $h(W) \in U$ and $h'(W) \notin U$
                (because requiring \emph{unique} generic point implies $T_0$
                property). Open sets $h^{-1}(U)$ and $h'^{-1}(U)$ therefore
                differ. Using one of the claims above, they are of the form
                $V_{U_1} = h^{-1}(U)$ and $V_{U_2} = h'^{-1}(U)$. So we have $W
                \in V_{U_1}$ and $W \notin V_{U_2}$. Then there exists $w \in W
                \cap U_1$, for which $\overline{\{w\}} \in V_{U_1}$ and
                $\overline{\{w\}} \notin V_{U_2}$. By definition
                $\overline{\{w\}} \in h^{-1}(U)$ and $\overline{\{w\}} \notin
                h'^{-1}(U)$ which means that $h(\overline{\{w\}}) \in U$ and
                $h'(\overline{\{w\}}) \notin U$. But that is a contradiction
                with assumption $g = h \circ f = h' \circ f$.
            }
        \item{We can define $h \colon V \rightarrow \MaxSpec(A)$ with $h(x_1,
            \dots, x_n) = (X_1 - x_1, \dots, X_n - x_n)$. Due to Hilberts
            Nullstellensatz this is a bijection (because $k$ alg.\ closed). Take
            a closed subset $C \subseteq V$. Again by Hilberts Nullstellensatz
            we get a radical ideal $I$ such that $V = V(I)$. So $h(C) = \{ m \in
            \MaxSpec \mid I \subseteq m \}$, which is a closed set. And if we
            take a closed set $V(I) \subseteq \MaxSpec(A)$ for some $I \subseteq
            A$ we have $h^{-1}(V(I)) = \{ x \in V \mid \forall f \in I \colon
            f(x) = 0 \}$. So $h$ is a homeomorphism.

            Observe that if we take $X = \MaxSpec(A)$, then $X^{\text{sob}} =
            \Spec(A)$. As sets that is clear, because all irreducible closed
            sets of $V$ are exactly vanishing sets of prime ideals in $A$. And
            topology on $V \cong \MaxSpec(A)$ is defined by the basis of closed
            sets being the vanishing sets of prime ideals, so $X^{\text{sob}} =
            \Spec(A)$.

            Then we have a diagram

            \begin{tikzcd}
                &V \arrow{d}{i} \arrow{r}{h} &\MaxSpec(A) \arrow{d}{j} \\
                &V^{\text{sob}} \arrow{r} &\Spec(A)
            \end{tikzcd}

            where $i \colon x \mapsto \overline{\{x\}}$ and $j \colon m \mapsto
            \{m\}$ (since maximal ideals are already closed points).

            By 2. part we know there exists $g \colon V^{\text{sob}} \rightarrow
            \Spec(A)$ with $j \circ h = g \circ i$. And similar there exists $f
            \colon \Spec(A) \rightarrow V^{\text{sob}}$ with $i \circ h^{-1} = f
            \circ j$. Combining these two equations gives us $j = g \circ f
            \circ j$ and $i = f \circ g \circ i$.

            Observe that the map $g \circ f$ satisfies the ``universal
            condition'' from 2. part for the map $i \colon V \rightarrow
            V^{\text{sob}}$. So it is the unique map $\pi \colon V \rightarrow
            V^{\text{sob}}$ that makes

            \begin{tikzcd}
                &V \arrow{d}{i} \arrow{r}{i} &V^{\text{sob}} \\
                &V^{\text{sob}} \arrow{ru}
            \end{tikzcd}

            And identity would also satisfy that condition, so
            $g \circ f$ must be the identity on $V^{\text{sob}}$.

            We argue exactly the same that $f \circ g$ is the identity on
            $\Spec(A)$.

            This proves that $V^{\text{sob}} \cong \Spec(A)$.
            }
    \end{enumerate}
\end{exercise}

\end{document}
