\newcommand{\sheet}{3}
\documentclass{article}
\usepackage[english, german]{babel}
\usepackage{amsthm,amssymb,amsmath,mathrsfs,mathtools}
\usepackage[shortlabels]{enumitem}
\usepackage{hyperref}
\usepackage{biblatex}
\usepackage{tikz}
\usepackage{tikz-cd}

% \usepackage[tmargin=1.25in,bmargin=1.25in,lmargin=1.2in,rmargin=1.2in]{geometry}


\newcommand{\C}{\mathbb{C}}
\newcommand{\R}{\mathbb{R}}
\newcommand{\N}{\mathbb{N}}
\newcommand{\Q}{\mathbb{Q}}
\newcommand{\Z}{\mathbb{Z}}
\newcommand{\Proj}{\mathbb{P}}
\newcommand{\Aff}{\mathbb{A}}

\DeclareMathOperator{\id}{id}
\DeclareMathOperator{\im}{im}
\DeclareMathOperator{\GL}{GL}
\DeclareMathOperator{\sgn}{sgn}
\DeclareMathOperator{\Tor}{Tor}
\DeclareMathOperator{\Sym}{Sym}
\DeclareMathOperator{\coker}{coker}
\DeclareMathOperator{\Quot}{Quot}
\DeclareMathOperator{\supp}{supp}
\DeclareMathOperator{\Hom}{Hom}
\DeclareMathOperator{\Spec}{Spec}
\DeclareMathOperator{\MinSpec}{MinSpec}
\DeclareMathOperator{\MaxSpec}{MaxSpec}
\DeclareMathOperator{\diag}{diag}
\DeclareMathOperator{\BL}{BL}
\DeclareMathOperator{\Ouv}{Ouv}
\DeclareMathOperator{\Sh}{Sh}
\DeclareMathOperator{\PSh}{PSh}
\DeclareMathOperator{\Eq}{Eq}
\DeclareMathOperator{\colim}{colim}
\DeclareMathOperator{\Pic}{Pic}
\DeclareMathOperator{\CL}{CL}
\DeclareMathOperator{\eq}{eq}
\DeclareMathOperator{\codim}{codim}

\newenvironment{exercise}[1] {
  \vspace{0.5cm}
  \noindent \textbf{Exercise~{#1}.}
} {
  \vspace{0.5cm}
}
\newenvironment{claim} {
  \par\noindent\textbf{Claim.}
} { }

\newenvironment{proof_claim} {
  \par\noindent\textbf{Proof of claim.}
} {
    \qed (of claim)
}

\title{Algebraic geometry 1\\Exercise sheet \sheet}
\author{Solutions by: Eric Rudolph and David Čadež}

\date{\today}


\begin{document}

\maketitle{}

\begin{exercise}{1}
    \begin{enumerate}
        \item Define 
        \begin{align*}
            \pi^{-1}:U &\longrightarrow \pi^{-1}(U) \\
            (x_1,...,x_n)  &\mapsto (x_1,...,x_n)[x_1:...:x_n].
        \end{align*}
        This is well-defined, because by definition of $U$, not all
        $x_i$ can be zero at the same time, so $[x_1:...:x_n]$ is actually a
        point in projective space. We also have 
            $(x_1,...,x_n)[x_1:...:x_n]\in Z$ for $(x_1,...,x_n)\in U$,
            because $x_ix_j=x_jx_i$ for all $1\leq i,j \leq n.$
            To see injectivity of $\pi^{-1}$, let
            $(x_1,...,x_n)\in U$ with $x_j\not = 0$. Then we have $y_j\not = 0$, because
            if we assume $x_j\not =0$ and $y_j=0$, then for some $y_i\not =0$ 
            (which exists since $[y_1:...:y_n]$ is a point in projective space) 
            we have $0\not = x_jy_i= x_iy_j=0.$
            Therefore, we can just set $y_j=1$. Then 
            \begin{align*}
                x_iy_j=x_jy_i \implies y_1=\frac{x_1y_j}{x_j}=\frac{x_1}{x_j},
            \end{align*}
            showing that all the $y_i$ are fixed up to a scalar after fixing all the $x_i$.
        \item    
            Define 
            \begin{align*}
                \phi: V_i &\to \mathbb{A}_n^k\\
                (x,y)&\mapsto (\frac{x_1}{y_i},\dots, x_i,\dots,\frac{x_n}{y_i}),
            \end{align*}
            where the inverse map is given by 
            \begin{align*}
                \phi^{-1}:\mathbb{A}_n^k &\to V_i\\
                (x_1,\dots,x_n)&\mapsto (x_1x_i,\dots,x_i,\dots,x_nx_i)[x_1:\dots:x_{i-1}:1:\dots:x_n].
            \end{align*}
    \end{enumerate}
\end{exercise}

\begin{exercise}{4}
    \begin{enumerate}
        \item{
                Lets first prove that $V_{U}$ are stable under
                intersections:
                \begin{claim}
                    Take $U, W \subseteq X$ open subsets. Then $V_{U \cap W} =
                    V_U \cap V_W$.
                \end{claim}
                \begin{proof_claim}
                    Inclusion $V_{U \cap W} \subseteq V_U \cap
                    V_W$ is clear.

                    For the other inclusion take $Z \in V_U \cap V_W$. By
                    definition $Z \cap U \not= \emptyset$ and $Z \cap V \not=
                    \emptyset$. Suppose $Z \cap (U \cap V) = \emptyset$. Then
                    $(Z \cap U)^c \cup (Z \cap V)^c = X$. But since $Z$ is
                    irreducible, and is covered by $U^c \cup V^c$, we must have
                    (WLOG) $Z \subseteq U^c$. That is in contradiction with
                    $Z \cap U \not= \emptyset$. 
                \end{proof_claim}

                It also behaves well under unions:
                \begin{align*}
                    V_{U \cup W} &= \{ Z\ \text{cl. irred.} \mid Z \cap (U \cup
                    W) \not= \emptyset \} \\
                    &= \{ Z\ \text{cl. irred.} \mid (Z \cap U) \not= \emptyset
                    \ \text{or}\ (Z \cap W) \not= \emptyset \} \\
                    &= \{ Z\ \text{cl. irred.} \mid (Z \cap U) \not= \emptyset
                    \} \cup \{ Z\ \text{cl. irred.} \mid (Z \cap W) \not=
                    \emptyset \} \\
                    &= V_U \cup V_W
                \end{align*}
                and practically same argument applies to infinite unions.

                Therefore every open subset of $X^{\text{sob}}$ can be written
                as $V_U$ for some open $U \subseteq X$ (in general it could've
                been just a base of topology, but it is the whole topology).

                \begin{claim}
                    Closed irreducible subsets of $X^{\text{sob}}$ are exactly
                    $V^c_U$ for open $U \subseteq X$ such that (closed) subset
                    $U^c \subseteq X$ is irreducible.
                \end{claim}
                \begin{proof_claim}
                    Take $V^c_U$ such that $U^c$ is not irreducible. Then there
                    exist closed subsets $U^c_1, U^c_2 \subseteq X$ with $U^c =
                    U^c_1 \cup U^c_2$ meanwhile $U^c \not= U^c_1$ and $U^c \not=
                    U^c_2$. Then $V_U = V_{U_1 \cup U_2} = V_{U_1} \cup V_{U_2}$
                    and we can thus cover $V_U$ with $V_{U_1}$ and $V_{U_2}$. We
                    just have to show $V_U \not= V_{U_1}$ and $V_U \not= V_{U_2}$.
                    Take $x \in U^c \setminus U^c_1$. Then $\overline{\{x\}} \in
                    V^c_U \setminus V^c_{U_1}$, so $V_U \not= V_{U_1}$, which
                    proves the claim.
                \end{proof_claim}

                Let us show $X^{\text{sob}}$ is sober. Let $V_U^c$ be closed
                irreducible. Then by last claim $U^c$ is closed and irreducible.
                The set $U^c$ is the generic point with $\overline{\{U^c\}} =
                V_U^c$. The inclusion $\overline{\{U^c\}} \subseteq V_U^c$ is
                obvious, because $V^c_U$ contains the point $U^c$ and is a
                closed set. For the other inclusion take a closed set that
                $V^c_W$ that contains $U^c$. That means $U^c \cap W = \emptyset$
                and thus $W \subseteq U$. Then we have $V_W \subseteq V_U$ and
                $V^c_U \subseteq V^c_W$. This proves that $V^c_U$ is the closure
                of the point $U^c$.
            }
        \item{
                Define
                \begin{align*}
                    h \colon X^{\text{sob}} &\rightarrow Z \\
                    W &\mapsto \text{unique generic point of}\ \overline{g(W)}.
                \end{align*}
                Note that: a continuous image of an irreducible set is
                irreducible and the closure of an irreducible set is
                irreducible.  
                So $\overline{g(W)} \subseteq Z$ is a closed irreducible subset and thus
                has a unique generic point in $Z$.

                Let's now prove $g = h \circ f$. Take $x \in X$. We have to
                prove $g(x)$ is the unique generic point of
                $\overline{g\left(\overline{\{x\}}\right)}$. Clearly $g(x) \in
                \overline{g\left(\overline{\{x\}}\right)}$. Take any closed $W
                \subseteq Z$ with $g(x) \in W$. Then, by definition, $x \in
                g^{-1}(W)$. Because $g^{-1}(W)$ is closed, also
                $\overline{\{x\}} \subseteq g^{-1}(W)$. So
                $g\left(\overline{\{x\}}\right) \subseteq W$. But since $W$ is
                closed we have $\overline{g\left(\overline{\{x\}}\right)}
                \subseteq W$. This proves that $g(x)$ is indeed a generic point
                of $\overline{g\left(\overline{\{x\}}\right)}$. So we have $g =
                h \circ f$.

                To prove $h$ is continuous we take an open set $U \subseteq Z$.
                we want to see that $h^{-1}(U)$ is open. Since $g^{-1}(U) =
                f^{-1}(h^{-1}(U))$ is open and $f^{-1}$ induces a bijection of
                open sets, the set $h^{-1}(U)$ is open as well. So $h$ is
                continuous.

                We should also argue why $h$ is unique. Take $h, h' \colon
                X^{\text{sob}} \rightarrow Z$ both continuous and satisfying $g
                = h \circ f = h' \circ f$. Pick any closed irreducible $W
                \subseteq X$. Suppose $h(W) \not= h'(W)$. WLOG there exists open
                $U \subseteq Z$ such that $h(W) \in U$ and $h'(W) \notin U$
                (because requiring \emph{unique} generic point implies $T_0$
                property). Open sets $h^{-1}(U)$ and $h'^{-1}(U)$ therefore
                differ. Using one of the claims above, they are of the form
                $V_{U_1} = h^{-1}(U)$ and $V_{U_2} = h'^{-1}(U)$. So we have $W
                \in V_{U_1}$ and $W \notin V_{U_2}$. Then there exists $w \in W
                \cap U_1$, for which $\overline{\{w\}} \in V_{U_1}$ and
                $\overline{\{w\}} \notin V_{U_2}$. By definition
                $\overline{\{w\}} \in h^{-1}(U)$ and $\overline{\{w\}} \notin
                h'^{-1}(U)$ which means that $h(\overline{\{w\}}) \in U$ and
                $h'(\overline{\{w\}}) \notin U$. But that is a contradiction
                with assumption $g = h \circ f = h' \circ f$.
            }
        \item{
            }
    \end{enumerate}
\end{exercise}

\end{document}
