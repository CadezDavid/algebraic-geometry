\newcommand{\sheet}{3}
\documentclass{article}
\usepackage[english, german]{babel}
\usepackage{amsthm,amssymb,amsmath,mathrsfs,mathtools}
\usepackage[shortlabels]{enumitem}
\usepackage{tikz}
\usepackage{tikz-cd}

% \usepackage[tmargin=1.25in,bmargin=1.25in,lmargin=1.2in,rmargin=1.2in]{geometry}


\newcommand{\C}{\mathbb{C}}
\newcommand{\R}{\mathbb{R}}
\newcommand{\N}{\mathbb{N}}
\newcommand{\Q}{\mathbb{Q}}
\newcommand{\Z}{\mathbb{Z}}

\DeclareMathOperator{\id}{id}
\DeclareMathOperator{\im}{im}
\DeclareMathOperator{\GL}{GL}
\DeclareMathOperator{\sgn}{sgn}
\DeclareMathOperator{\Tor}{Tor}
\DeclareMathOperator{\Sym}{Sym}
\DeclareMathOperator{\coker}{coker}
\DeclareMathOperator{\Quot}{Quot}
\DeclareMathOperator{\supp}{supp}
\DeclareMathOperator{\Hom}{Hom}
\DeclareMathOperator{\Spec}{Spec}
\DeclareMathOperator{\MinSpec}{MinSpec}
\DeclareMathOperator{\diag}{diag}


\newenvironment{exercise}[1] {
  \vspace{0.5cm}
  \noindent \textbf{Exercise~{#1}.}
} {
  \vspace{0.5cm}
}
\newenvironment{claim} {
  \noindent \textbf{Claim.}
} {
}

\title{Algebraic geometry 1\\Exercise sheet \sheet}
\author{Solutions by: Eric Rudolph and David Čadež}

\date{\today}


\begin{document}

\maketitle{}

\begin{exercise}{1}
    \begin{enumerate}[a)]
        \item{
                % Let us first create a map
                % \begin{equation*}
                %     \begin{split}
                %         F \colon \Hom_{k-group}(A_1, A_2) &\to \Hom_{Y-group}(Y
                %         \times_{\Spec k} A_1, Y \times_{\Spec k} A_2) \\ g
                %         &\mapsto (\id, g)
                %     \end{split}
                % \end{equation*}
                % First we check that restriction $(\id, g)\mid_{\{x\} \times
                % A_1}$ recovers $g$. Since we are identifying $A_1 \cong \{x\}
                % \times A_1$ and $A_2 \cong \{x\} \times A_2$, it is clear that
                % we obtain a map $g \colon A_1 \to A_2$.

                % What seems to be the harder part of the exercise is to show that
                % $F$ is indeed surjective (or, equivalently, that the map given
                % in the exercise sheet is injective).

                % Our assumptions conveniently fit conditions of Rigidity theorem,
                % so thats what we use.
                
                Let
                \begin{equation*}
                    \begin{split}
                        F \colon  \Hom_{Y-group}(Y \times_{\Spec k} A_1, Y
                        \times_{\Spec k} A_2) &\to \Hom_{k-group}(A_1, A_2) \\
                        g &\mapsto g \mid_{\{x\} \times A_1} 
                    \end{split}
                \end{equation*}
                be the map given in the exercise.

                We can check injectivity and surjectivity of $F$ by hand.
                Since both have hom sets have group structure, we can take $f$
                that gets mapped to $0$ (i.e. $f \mid_{\{x\} \times A_1}$ is the
                unique map $A_1 \to A_2$ that factors through $e \colon \Spec k
                \to A_2$).
                In particular that means that the composition $Y \times A_1 \to
                Y \times A_2$ is constant when
                restricted to $\{x\} \times A_1$. Since all our assumptions
                conveniently fit Rigity theorem, we can use that to get that $f$ factors
                through $Y \to Y \times A_2$, which shows that $Y \times A_1 \to
                Y \times A_2$ is the identity element in
                $\Hom_{Y-\text{group}}(Y \times A_1, Y \times A_2)$ (i.e. the
                ``zero map'').
                Therefore $F$ is injective.
                It is clearly surjective; given a map $g \colon A_1 \to A_2$ we
                can do base change to a map $(\id, g) \colon Y \times A_1 \to Y
                \times A_2$, which restricts to $g$.
            }
        \item{

            }
    \end{enumerate}
\end{exercise}

\begin{exercise}{2}
    \begin{enumerate}[a)]
        \item{
                Lets use primitive element theorem and write $K = k(a)$ for some $a
                \in K$.
                Written differently we have $K = \Quot(k[x]/f(x))$ where $f$ is
                the minimal polynomial of $a \in K$.
                Denote $A = k[x]/f(x)$.
                Note that minimal polynomial is irreducible and thus $A$ a domain.

                From Kähler arithmetic we have that
                \begin{equation*}
                    \Omega^1_{K/k} = K \otimes_A \Omega^1_{A/k}
                \end{equation*}
                So it is enough to calculate $\Omega^1_{A/k}$.

                Suppose $K/k$ is separable. That implies $x$ is separable and
                $f(x)$ has no multiple roots. Therefore $f'(x)$ and $f(x)$ are
                coprime and thus generate whole $k[x]$. That means $f'(x)$ is
                invertible as element in $A$.

                Let $M$ be any $A$-module and $\delta \in \Der_k(A, M)$.
                Derivation $\delta$ has to be $k$-linear, so it is uniquely
                defined by its value in $x$.
                Since $f(x)$ is $0$ in $A$, we must have
                \begin{equation*}
                    \delta(f(x)) = f'(x) \delta(x) = 0
                \end{equation*}
                But $f'(x)$ is invertible, so we can simply multiple by its
                inverse and obtain $\delta(x) = 0$.
                We've shown that for every $A$-module $M$, there exist only
                derivation constantly $0$. Therefore $\Omega^1_{A/k} = 0$ and
                thus also $\Omega^1_{K/k} = 0$.

            }
        \item{
                So $K = \Quot(k[x_1, \dots, x_n] / I)$.
                Denote $B = k[x_1, \dots, x_n] / I$ and let $A = k[y_1, \dots,
                y_d]$ be Noether normalization of $B$ (so $A \to B$ is finite).
                By definition $d = \trdeg(K/k)$.

                So we have maps $k \to A \to B$.
                Using Kähler arithmetic we get exact sequence
                \begin{equation*}
                    B \otimes_A \Omega^1_{A/k} \to \Omega^1_{B/k} \to
                    \Omega^1_{B/A} \to 0.
                \end{equation*}
                Because $\characteristic(k) = 0$ and $A \to B$ finite, we have
                by previous part $\Omega^1_{B/A} = 0$. So we get a surjection $B
                \otimes_A \Omega^1_{A/k} \to \Omega^1_{B/k}$.
                Since $A \to B$ is injective, the map $B \otimes_A
                \Omega^1_{A/k} \to \Omega^1_{B/k}$ is also injective.
                So we have 
                \begin{equation*}
                    B \otimes_A \Omega^1_{A/k} \xrightarrow{\sim} \Omega^1_{B/k}
                \end{equation*}

                We've shown during lectures that $\Omega^1_{A/k} = A^d$.
                So $B \otimes_A \Omega^1_{A/k} = B^d$.

                Again using Kähler arithmetic for localization we have
                \begin{equation*}
                    K \otimes_B \Omega^1_{B/k} \cong \Omega^1_{K/k}
                \end{equation*}
                So $\Omega^1_{K/k} \cong K^d$.
            }
        \item{
                It suffices to find any non-empty open subscheme, as $X$ is
                integral.

                Suppose $\Omega^1_{X/k}$ has rank $n$ at the generic point.
                Since $\characteristic(k) = 0$, that is equal to the local
                dimension.

                It has rank at least $n$ everywhere else. And using upper
                semicontinuity we get that it has rank exactly $n$ on an open
                neighbourhood of the generic point. So $\Omega^1_{X/k}$ is
                therefore locally free of rank $n$ on some non-empty
                neighbourhood of the generic point, which is where $X$ is then
                smooth.
            }
    \end{enumerate}
\end{exercise}


\end{document}
