\newcommand{\sheet}{11}
\documentclass{article}
\usepackage[english, german]{babel}
\usepackage{amsthm,amssymb,amsmath,mathrsfs,mathtools}
\usepackage[shortlabels]{enumitem}
\usepackage{tikz}
\usepackage{tikz-cd}

% \usepackage[tmargin=1.25in,bmargin=1.25in,lmargin=1.2in,rmargin=1.2in]{geometry}


\newcommand{\C}{\mathbb{C}}
\newcommand{\R}{\mathbb{R}}
\newcommand{\N}{\mathbb{N}}
\newcommand{\Q}{\mathbb{Q}}
\newcommand{\Z}{\mathbb{Z}}

\DeclareMathOperator{\id}{id}
\DeclareMathOperator{\im}{im}
\DeclareMathOperator{\GL}{GL}
\DeclareMathOperator{\sgn}{sgn}
\DeclareMathOperator{\Tor}{Tor}
\DeclareMathOperator{\Sym}{Sym}
\DeclareMathOperator{\coker}{coker}
\DeclareMathOperator{\Quot}{Quot}
\DeclareMathOperator{\supp}{supp}
\DeclareMathOperator{\Hom}{Hom}
\DeclareMathOperator{\Spec}{Spec}
\DeclareMathOperator{\MinSpec}{MinSpec}
\DeclareMathOperator{\diag}{diag}


\newenvironment{exercise}[1] {
  \vspace{0.5cm}
  \noindent \textbf{Exercise~{#1}.}
} {
  \vspace{0.5cm}
}
\newenvironment{claim} {
  \noindent \textbf{Claim.}
} {
}

\title{Algebraic geometry 1\\Exercise sheet \sheet}
\author{Solutions by: Eric Rudolph and David Čadež}

\date{\today}


\begin{document}

\maketitle{}

\begin{exercise}{3}
    \begin{enumerate}
        \item Since $k$ is algebraically closed, the only irreducible polynomials $f\in k[x,y]$ are of degree $1$.
        
        Hence, we can write 
        \begin{align*}
            f_r=l_1  \dots  l_r,
        \end{align*}
        where $l_i\in k[x,y]$ is of degree $1$. From the assumption that $f_r$ is homogenous it follows that the $l_i$ are homogenous.

        Therefore, we can write
        \begin{align*}
            Z=V(f_r)=V(l_1 \dots  l_r)=\cup_i V(l_i)
        \end{align*}
        and since $V(l_i)$ is a line through the origin, $Z$ can be written as the finite union of lines through the origin.

        \item 
        We first want to prove that $\dim(\mathcal{O}_{X,(x,y)})=1$ for all $r$. The prime ideals $p$ in this ring 
        fulfil $(f)\subset p \subset (x,y)$. Remember that we can write down these prime ideals explicitely as in 
        \href{https://math.stackexchange.com/questions/56916/what-do-prime-ideals-in-kx-y-look-like}{"What do primes of k[x,y] look like"}.
        From this the claim follows.

        We know that $\dim_k(m_{\mathcal{O}_{X,(x,y)}}/m_{\mathcal{O}_{X,(x,y)}}^2)$ is the number of generators 
        of $m_{\mathcal{O}_{X,(x,y)}}$.

        Now if $r=1$, then we can write $f=g(x,y)x+h(x,y)y$ and w.l.o.g. we have $g(0,0)=1$, meaning that it is invertible (after localizing
        ). Therefore $f=x+h(x,y)y$, so $y\mid x$ meaning $(x,y)=(y)$
        On the other hand, if $r>1$, then $x\nmid y$ and $y\nmid x$ meaning that $m$ is no principal ideal showing that $X$ is 
        singular at zero in this case. (This can be seen by writing $f$ as $f=x^2h_1(x,y)+xyh_2(x,y)+y^2h_3(x,y))$.

        \item By part two of this exercise, all the schemes have a singular point at the origin. I don't know why they do not have singular
        points anywhere else.
    \end{enumerate}

\end{exercise}

\end{document}
