\newcommand{\sheet}{3}
\documentclass{article}
\usepackage[english, german]{babel}
\usepackage{amsthm,amssymb,amsmath,mathrsfs,mathtools}
\usepackage[shortlabels]{enumitem}
\usepackage{tikz}
\usepackage{tikz-cd}

% \usepackage[tmargin=1.25in,bmargin=1.25in,lmargin=1.2in,rmargin=1.2in]{geometry}


\newcommand{\C}{\mathbb{C}}
\newcommand{\R}{\mathbb{R}}
\newcommand{\N}{\mathbb{N}}
\newcommand{\Q}{\mathbb{Q}}
\newcommand{\Z}{\mathbb{Z}}

\DeclareMathOperator{\id}{id}
\DeclareMathOperator{\im}{im}
\DeclareMathOperator{\GL}{GL}
\DeclareMathOperator{\sgn}{sgn}
\DeclareMathOperator{\Tor}{Tor}
\DeclareMathOperator{\Sym}{Sym}
\DeclareMathOperator{\coker}{coker}
\DeclareMathOperator{\Quot}{Quot}
\DeclareMathOperator{\supp}{supp}
\DeclareMathOperator{\Hom}{Hom}
\DeclareMathOperator{\Spec}{Spec}
\DeclareMathOperator{\MinSpec}{MinSpec}
\DeclareMathOperator{\diag}{diag}


\newenvironment{exercise}[1] {
  \vspace{0.5cm}
  \noindent \textbf{Exercise~{#1}.}
} {
  \vspace{0.5cm}
}
\newenvironment{claim} {
  \noindent \textbf{Claim.}
} {
}

\title{Algebraic geometry 1\\Exercise sheet \sheet}
\author{Solutions by: Eric Rudolph and David Čadež}

\date{\today}


\begin{document}

\maketitle{}

\begin{exercise}{1}
    \begin{enumerate}
        \item Let $X$ be a finite set that is irreducible with 
        respect to some topology $\mathcal{F}$ on $X$. Then we get
        $\mid \mathcal{F} \mid<\infty$ and since finite unions of
        closed sets are closed again, we get that
        \begin{align*}
            X^{\prime}:=\bigcup_{U \subsetneq X \text{closed}}{U}
        \end{align*}
        is closed in X. Since $X$ is by assumption irreducible,
        $X\neq X^{\prime}$, so we can pick $x_0\in X \backslash X^{\prime}$,
        which is by construction generic. For the second part of the exercise we use 
        part 2 of Hochster's Theorem. As a finite set, $X$ is 
        quasicompact and as a basis $\mathcal{B}$ consisting of quasicompact 
        open sets stable under finite intersections take all of the open sets. \\
        It remains to show that $X$ is sober. We need to check that every irreducible
        subset of $X$ has a unique generic point. The existence of a generic point comes
        from part of of this exercise. Uniqueness of this point is due to the fact
        that generic points in $T_0$ spaces are unique if they exist, which follows directly
        from the definition of $T_0$.
    \end{enumerate}
    
    \begin{enumerate}
        \item{
                Let $X$ be a finite irreducible topological space. Since
                \begin{equation*} X = \bigcup_{x \in X} \overline{\{x\}}
                \end{equation*} is a finite decomposition into closed sets, we
                must have $\overline{\{x\}} = X$ for some $x \in X$. This $x$ is
                a generic point of $X$.

                If we additionally assumed $X$ is $T_0$, then this point $x$
                would be unique, since in a $T_0$ space we have
                $\overline{\{x\}} \not= \overline{\{y\}}$ for $x \not= y$. Also
                in a finite space the conditions of quasicompactness and the
                basis being stable under finite intersections are clearly
                fulfilled. So finite $T_0$ spaces are spectral.
            }
        \item{
                Let us first describe what $\Spec(\Z)$ looks like. The ring $\Z$ a PID
                with prime ideals being those $(a)$, for which $a \in \Z$ is
                either a prime number or $a = 0$. So
                \begin{equation*}
                    \Spec(\Z) = \{ (p) \mid p\ \text{prime} \} \cup \{ (0) \}.
                \end{equation*}
                In this space $(0)$ is a unique generic point.

                Closed sets in $\Spec(\Z)$ are by definiton
                \begin{equation*}
                    V((a)) = \{ (p) \in \Spec(\Z) \mid (a) \subseteq (p) \} = 
                    \{ (p) \in \Spec(\Z) \mid p\ \text{divides}\ a \}.
                \end{equation*}
                So if $a = \Pi_{i} p^{k_i}_i$, then $V((a)) = {\{ (p_i) \}}_i
                \subseteq \Spec(\Z)$. Since any $a \in \Z$ is only divisible by
                finitely many prime numbers, we get the cofinite topology on
                $\Spec(\Z) \setminus \{(0)\}$.

                Adding generic point $(0)$ to $\Spec(\Z) \setminus \{(0)\}$ is
                the construction $X \rightarrow X^{\text{sob}}$ which we did
                last week.

                So closed sets in $\Spec(\Z)$ are exactly all finite subsets of
                $\Spec(\Z) \setminus \{(0)\}$, the whole space and the empty
                set.

                Define the diagram with objects
                \begin{equation*}
                    F_i := \{ (p_1), \dots, (p_i) \} \cup \{ (0) \}
                \end{equation*}
                with subspace topology, as $F_i \subseteq \Spec(\Z)$,
                and morphisms
                \begin{equation*}
                    f_i \colon F_{i+1} \rightarrow F_i
                \end{equation*}
                mapping $f_i((p_j)) = (p_j)$ for $j \leq i$ and $f_i((p_{i+1}))
                = f_i((0)) = (0)$.

                We view $F_i$ as a subspace of $F_j$ for $i < j$.

                First we have to check that $f_i$ are continuous. Take a closed
                subset $U \subseteq F_i$. If $(0) \in U$, then $U = F_i$ and
                $f^{-1}_i(U) = F_{i+1}$. And if $U \subseteq \{ (p_1), \dots,
                (p_i) \}$, then $f^{-1}_i(U) = \{ (p_1), \dots, (p_i) \}
                \subseteq F_{i+1}$.

                We claim $\Spec(\Z)$ is the limit of this diagram. For that we
                define $\alpha_i \colon \Spec(\Z) \rightarrow F_i$ by
                $\alpha_i((p_j)) = (p_j)$ for $j \leq i$ and $\alpha_i((p_j)) =
                \alpha_i((0)) = (0)$ for $j > i$.
                Clearly we have $\alpha_i = f_i \circ \alpha_{i+1}$.

                Suppose now there exists an object $B$ with $\beta_i \colon B
                \rightarrow F_i$. Define $\alpha \colon B \rightarrow \Spec(\Z)$
                in the following way. Take $b \in B$ and look at $\beta_i(b)$
                for different $i = 1, 2, \dots$. If $\beta_i(b) = (0)$ for all
                $i$, then define $\alpha(b) = (0) \in \Spec(\Z)$. If
                $\beta_{i_0}(b) \not= (0)$ for some $i_0$, then $\beta_j(b) =
                \beta_{i_0}(b)$ for all $j > i_0$ using commutativity of the
                diagram.

                Lets show that $\alpha$ is continuous. Take a closed subset $A
                \subseteq \Spec(\Z)$. If $(0) \in A$, then $\alpha^{-1}(A) = B$.
                Else $A$ is finite subset of $\Spec(\Z) \setminus \{(0)\}$. Then
                $A$ is contained in some $F_i$ (pick $i$ to be largest such that
                $(p_i) \in A$) and $(\alpha_i^{-1} \circ \alpha_i)(A) = A$
                holds. Therefore
                \begin{equation*}
                    \alpha^{-1}(A) = (\alpha^{-1} \circ \alpha^{-1}_i \circ
                    \alpha_i)(A) = (\beta^{-1}_i \circ \alpha_i)(A) =
                    \beta^{-1}_i(\alpha_i(A)).
                \end{equation*}
                The set $\alpha_i(A)$ is a finite subset of $F_i$ that does not
                contain $(0)$, so it is closed. Since $\beta_i$ is continuous,
                the set $\alpha^{-1}(A)$ is then closed, which proves continuity
                of $\alpha$.

                The map $\alpha$ was defined using condition that the diagram
                must commute. It is clear that any other definition of $\alpha$
                would not satisfy commutativity conditions $\beta_i = \alpha_i
                \circ \alpha$ for all $i$. So $\alpha$ was unique and thus
                $\Spec(\Z)$ is the limit of the constructed diagram.
            }
    \end{enumerate}
\end{exercise}

\begin{exercise}{2}
    Denote $A = \lim A_i$, $B = \lim B_i$ and $C = \lim C_i$. Also denote maps
    $A_i \rightarrow A$ with $f_i$, $B_i \rightarrow B$ with $g_i$ and $C_i
    \rightarrow C$ with $h_i$.

    By composing $\alpha_i$ and $g_i$ we get $A_i \rightarrow B$ defined as $g_i
    \circ \alpha_i$. Then by the definition of a colimit we have a unique map $A
    \rightarrow B$, denoted by $\alpha$, such that $g_i \circ \alpha_i = \alpha
    \circ f_i$. In the same way we obtain $\beta \colon B \rightarrow C$. With
    these definitions the whole diagram commutes.

\end{exercise}

\end{document}
