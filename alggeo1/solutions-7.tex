\newcommand{\sheet}{7}
\documentclass{article}
\usepackage[english, german]{babel}
\usepackage{amsthm,amssymb,amsmath,mathrsfs,mathtools}
\usepackage[shortlabels]{enumitem}
\usepackage{hyperref}
\usepackage{biblatex}
\usepackage{tikz}
\usepackage{tikz-cd}

% \usepackage[tmargin=1.25in,bmargin=1.25in,lmargin=1.2in,rmargin=1.2in]{geometry}


\newcommand{\C}{\mathbb{C}}
\newcommand{\R}{\mathbb{R}}
\newcommand{\N}{\mathbb{N}}
\newcommand{\Q}{\mathbb{Q}}
\newcommand{\Z}{\mathbb{Z}}
\newcommand{\Proj}{\mathbb{P}}
\newcommand{\Aff}{\mathbb{A}}

\DeclareMathOperator{\id}{id}
\DeclareMathOperator{\im}{im}
\DeclareMathOperator{\GL}{GL}
\DeclareMathOperator{\sgn}{sgn}
\DeclareMathOperator{\Tor}{Tor}
\DeclareMathOperator{\Sym}{Sym}
\DeclareMathOperator{\coker}{coker}
\DeclareMathOperator{\Quot}{Quot}
\DeclareMathOperator{\supp}{supp}
\DeclareMathOperator{\Hom}{Hom}
\DeclareMathOperator{\Spec}{Spec}
\DeclareMathOperator{\MinSpec}{MinSpec}
\DeclareMathOperator{\MaxSpec}{MaxSpec}
\DeclareMathOperator{\diag}{diag}
\DeclareMathOperator{\BL}{BL}
\DeclareMathOperator{\Ouv}{Ouv}
\DeclareMathOperator{\Sh}{Sh}
\DeclareMathOperator{\PSh}{PSh}
\DeclareMathOperator{\Eq}{Eq}
\DeclareMathOperator{\colim}{colim}
\DeclareMathOperator{\Pic}{Pic}
\DeclareMathOperator{\CL}{CL}
\DeclareMathOperator{\eq}{eq}
\DeclareMathOperator{\codim}{codim}

\newenvironment{exercise}[1] {
  \vspace{0.5cm}
  \noindent \textbf{Exercise~{#1}.}
} {
  \vspace{0.5cm}
}
\newenvironment{claim} {
  \par\noindent\textbf{Claim.}
} { }

\newenvironment{proof_claim} {
  \par\noindent\textbf{Proof of claim.}
} {
    \qed (of claim)
}

\title{Algebraic geometry 1\\Exercise sheet \sheet}
\author{Solutions by: Eric Rudolph and David Čadež}

\date{\today}


\begin{document}

\maketitle{}

\begin{exercise}{1}
    \begin{enumerate}
        \item
        Pick any $\mathcal{O}_X$-module $\mathcal{M}$. Then we have
        \begin{align*}
            \Hom_{\mathcal{O}_X} (f_* \widetilde{N}, \mathcal{M}) &\cong
            \Hom_A(f_* \widetilde{N}(X), \mathcal{M}(X)) \\
            &= \Hom_A(N|_A, \mathcal{M}(X)) \\
            &\cong \Hom_{\mathcal{O}_X}(\widetilde{N|_A}, \mathcal{M}).
        \end{align*}
        By the Yoneda lemma, this implies that $f_*\widetilde{N}\cong
    \tilde{N}_{|A}$. 
    \item For the second part of this exercise, we extend the
        first part as follows, using that $f^*$ is left-adjoint to $f_*$
        \begin{align*}
            \Hom_{\mathcal{O}_Y}(f^*\tilde{\mathcal{M}},\tilde{N})
            \cong \Hom_{\mathcal{O}_X}(\widetilde{\mathcal{M}},f_*\widetilde{\mathcal{N}})
            \cong \Hom_A (\widetilde{\mathcal{M}}(B),f_*\widetilde{\mathcal{N}}(B))\\
            = \Hom_A(M,\widetilde{\mathcal{N}}(A))
            \cong \Hom_B(N_{|A}\otimes_A B, \tilde{N}(A))
            \cong \Hom_{\mathcal{O}_Y}(\widetilde{\mathcal{M}\otimes_A B}, \widetilde{\mathcal{N}}).
        \end{align*} 
        Now, by the Yoneda lemma we again obtain that 
        \begin{align*}
            \widetilde{\mathcal{M}\otimes_A B} 
            \cong f^*\tilde{\mathcal{M}}.
        \end{align*}
        Next, we want to show that we can extend this exercise from affine schemes to schemes.
        
        Let $S_i$ with $i\in I$ be a cover of $S$ by open affines. Then for each $i\in I$ we get that
        $g^{-1}(S_i)$ is a subscheme of $Z_i\subset Z$ (unfortunately not necessarilary affine). Now, we cover
        each of these subschemes $Z_i$ by open affines $Z_{ij}$. By construction $g$ maps $Z_{ij}$ into $S_i$.
        Hence, 
        \begin{align*}
            (g^*\mathcal{M})_{Z_{ij}}=f^*\mathcal{M}_{Z_{ij}}\cong \widetilde{M\otimes_A B},
        \end{align*}
        showing that $g^*$ preserves quasi-coherence.
    \end{enumerate}
\end{exercise}

\begin{exercise}{2}
    For every homogenous polynomial $F(X_0, \dots, X_n)$ of degree $m$ we attach
    $\{f_i\}_{i=0,\dots,n}$, where $f_i(X_{0/i}, \dots, X_{i-1/i}, X_{i+1/i},
    X_{n/i})$ is the unique polynomial such that $\beta_i(f_i) = \frac{F(X_0,
    \dots, X_n)}{X^m_i}$, where
    \begin{align*}
        \beta_i \colon \Z[X_{0/i}, \dots, X_{n/i}] &\to \Z[X_0, \dots, X_n,
        X^{-1}_i] \\
        X_{j/i} &\mapsto \frac{X_j}{X_i}
    \end{align*}
    
    First we check that $\alpha^m_{i, j} (f_i) = f_j$. Due to linearity, it is
    enough to check the case when $F$ is a monomial. Let $F = \prod^n_{i = 0}
    X^{m_i}_i$. Then
    \begin{equation*}
        f_i = \prod^n_{k = 0, k \not= i} X^{m_k}_{k/i} \quad\text{and}\quad f_j
        = \prod^n_{k = 0, k \not= j} X^{m_k}_{j/k}.
    \end{equation*}
    Simply check
    \begin{equation*}
        \alpha^m_{i, j} (f_i) =
        % \alpha^m_{i, j} (\prod^n_{k = 0, k \not= i} X^{m_j}_{k/i}) =
        \prod^n_{k = 0, k \not= i} \alpha^m_{i, j} (X^{m_j}_{k/i}) =
        X^m_{j/i} X^{-1}_{j/i} \prod^n_{k = 0, k \not= i, j} (X_{k/j}
        X^{-1}_{j/i})^{m_k} = f_j
    \end{equation*}
    where we used $m = \sum^n_{k=0} m_k$ in the last equality.
    So $\{f_i\}_{i = 0, \dots, n}$ define a global section on $\Proj^n_k$.

    To show injectivity, suppose $f_i = 0$ for all $i$. Then $\beta_i(f_i) = 0$
    for any fixed $i$ and thus $\frac{F(X_0, \dots, X_n)}{X^n_i} = 0$, so
    $F(X_0, \dots, X_n) = 0$.

    From this we actually get that if global section $\{f_i\}_{i = 0, \dots, n}$
    vanishes on some $U_i$, then it is $0$, which aligns with intuition that a
    homogenous polynomial will be defined uniquely if we set one variable to be
    $1$ (and remember its degree).

    To show surjectivity, take $\{f_i\}_{i = 0, \dots, n}$ for which
    $\alpha^m_{i, j} (f_i) = f_j$ holds for every $i, j$. Then simply choose
    some $i$ and let $F(X_0, \dots X_n) = X^m_i \beta_i(f_i)$. So we found $F$
    that maps to $\{f_i\}_{i = 0, \dots, n}$.
\end{exercise}

\begin{exercise}{3}
    \begin{enumerate}
        \item{
            We have a cover $\Proj^n_\Z = \cup_i U_i$, where $U_i = \Spec(\Z[X_{j/i}, j
            \not= i])$. We defined $\Proj^n_k$ to be simply the fibered product
            $\Proj^n_\Z \times_{\Spec(\Z)} \Spec(k)$. We can use 1st exercise from sheet
            6, to get a cover
            \begin{align*}
                \Proj^n_k &= \bigcup_i U_i \times_{\Spec(\Z)} \Spec(k) \\
                &= \bigcup_i \Spec(\Z[X_{j/i}, j \not= i]) \times_{\Spec(\Z)} \Spec(k) \\
                &= \bigcup_i \Spec(\Z[X_{j/i}, j \not= i] \otimes_{\Z} k) \\
                &= \bigcup_i \Spec(k[X_{j/i}, j \not= i]).
            \end{align*}

            Define morphism $\Proj^n_k \to (\Proj^n_k(k))^{\text{sob}}$ on the
            cover.

            % this is wrong, for sure
            % We can show that soberification of $(\Proj^n_k(k))^{\text{sob}}$ is same as
            % soberification on each open set of the cover and then gluing.

            \href{https://stacks.math.columbia.edu/tag/06N9}{Lemma 06N9}
            We have that for a space $X$ and a covering $X = \bigcup_i X_i$, the
            space $X$ is sober if and only if $X_i$ is sober for every $i$.

            We showed on sheet 3 that soberification of an $\Aff^n_k(k)$ is
            $\Spec(k[X_1, \dots, X_n])$.
            
            So we have $(\Proj^n_k(k))^{\text{sob}} = \bigcup_i
            (\Aff^n_k(k))^{\text{sob}} = \bigcup_i \Spec(k[X_1, \dots, X_n])$.

            Define morphism
            \begin{equation*}
                \Proj^n_k = \bigcup_i \Spec(k[X_{j/i}, j \not= i]) \to
                (\Proj^n_k(k))^{\text{sob}} = \bigcup_i \Spec(k[X_1, \dots, X_n])
            \end{equation*}
            with isomorphism for every $i$.
            }

        \item{
                We defined $V(s) \subseteq \Proj^n_k$ locally on affine
                subschemes. Our definition assumed we have a line bundle
                $\mathcal{L}$ on $(X, \mathcal{O}_X)$. In our case $\mathcal{L}
                = \mathcal{O}_{\Proj^n_k}(d)$

                Locally on $U_i = \Spec(k[X_{j/i}, j \not= i])$ we have equality
                $\mathcal{O}_{\Proj^n_k}(d)|_{U_i} = \mathcal{O}_{U_i}$ by
                definition.

                So we have
                \begin{equation*}
                    V(s) \cap U_i = \Spec(k[X_{j/i}, j \not= i] / (f_i)),
                \end{equation*}
                where $f_i \in k[x_{0/i}, \dots x_{i-1/i}, x_{i+1/i}, x_{n/i}]$
                is the polynomial from exercise $2$ applied on $f$.

                On the other hand we had $V^+(f) \subseteq \Proj^n_k(k)$. We can
                intersect it with $V_i \cong \Aff^n_k(k)$ and get $V(f_i) =
                V^+(f) \cap V_i \cong \MaxSpec((k[x_1, \dots,
                x_n]/(f))_{\text{red}})$, where $f_i$ as in exercise $2$ (rename
                $x_1, \dots, x_n$ appropriately). Then we can do $^{\text{sch}}$
                on this to get $(V(f_i))^{\text{sch}} \cong \Spec((k[x_1, \dots,
                x_n]/(f_i))_{\text{red}})$.

                So we have that $V(s)_{\text{red}} = V^+(f)^{\text{sch}}$ as
                topological subsets of $\Proj^n_k \cong
                (\Proj^n_k(k))^{\text{sch}}$ are the same. And clearly their
                structure sheaves are defined as simply restrictions of
                structure sheaves of $\Proj^n_k$ and
                $(\Proj^n_k(k))^{\text{sch}}$ respectively.
            }
    \end{enumerate}
\end{exercise}

\begin{exercise}{4}
    We don't really want to do all the explicit calculations, so we only show what we think
    is maybe the main takeaway of this exercise.

    For some polynomial $f\in \mathbb{R}[x,y]$ we have that 
    \begin{align*}
        &V(f) \times_{\Spec(\mathbb{R})} \Spec(\mathbb{C})\\
        &\cong \Spec(\mathbb{R}[x,y]/(f)) \otimes_{Spec(\mathbb{R})}\Spec(\mathbb{C})\\
        &\cong \Spec(\mathbb{R}[x,y]/(f)\times_\mathbb{R} \mathbb{C})\\
        &\cong \Spec(\mathbb{C}[x,y]/(f)).
    \end{align*}
    In the following, we take $f(x,y):=xy-1$ and $g(x,y):=x^2+y^2-1$. We know from 
    the first sheet, that 
    \begin{align*}
        \mathbb{C}[x,y]/(f)\cong \mathbb{C}[x,y]/(g),
    \end{align*}
    but one can easily check that
    \begin{align*}
        \mathbb{R}[x,y]/(f) \not \cong \mathbb{R}[x,y]/(g),
    \end{align*}
    since the left side has strictly more units than the right side.

    Therefore, this is an example showing that varieties, so in particular schemes being isomorphic 
    is not stable under base change.
\end{exercise}

\end{document}
