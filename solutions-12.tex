\newcommand{\sheet}{12}
\documentclass{article}
\usepackage[english, german]{babel}
\usepackage{amsthm,amssymb,amsmath,mathrsfs,mathtools}
\usepackage[shortlabels]{enumitem}
\usepackage{hyperref}
\usepackage{biblatex}
\usepackage{tikz}
\usepackage{tikz-cd}

% \usepackage[tmargin=1.25in,bmargin=1.25in,lmargin=1.2in,rmargin=1.2in]{geometry}


\newcommand{\C}{\mathbb{C}}
\newcommand{\R}{\mathbb{R}}
\newcommand{\N}{\mathbb{N}}
\newcommand{\Q}{\mathbb{Q}}
\newcommand{\Z}{\mathbb{Z}}
\newcommand{\Proj}{\mathbb{P}}
\newcommand{\Aff}{\mathbb{A}}

\DeclareMathOperator{\id}{id}
\DeclareMathOperator{\im}{im}
\DeclareMathOperator{\GL}{GL}
\DeclareMathOperator{\sgn}{sgn}
\DeclareMathOperator{\Tor}{Tor}
\DeclareMathOperator{\Sym}{Sym}
\DeclareMathOperator{\coker}{coker}
\DeclareMathOperator{\Quot}{Quot}
\DeclareMathOperator{\supp}{supp}
\DeclareMathOperator{\Hom}{Hom}
\DeclareMathOperator{\Spec}{Spec}
\DeclareMathOperator{\MinSpec}{MinSpec}
\DeclareMathOperator{\MaxSpec}{MaxSpec}
\DeclareMathOperator{\diag}{diag}
\DeclareMathOperator{\BL}{BL}
\DeclareMathOperator{\Ouv}{Ouv}
\DeclareMathOperator{\Sh}{Sh}
\DeclareMathOperator{\PSh}{PSh}
\DeclareMathOperator{\Eq}{Eq}
\DeclareMathOperator{\colim}{colim}
\DeclareMathOperator{\Pic}{Pic}
\DeclareMathOperator{\CL}{CL}
\DeclareMathOperator{\eq}{eq}
\DeclareMathOperator{\codim}{codim}

\newenvironment{exercise}[1] {
  \vspace{0.5cm}
  \noindent \textbf{Exercise~{#1}.}
} {
  \vspace{0.5cm}
}
\newenvironment{claim} {
  \par\noindent\textbf{Claim.}
} { }

\newenvironment{proof_claim} {
  \par\noindent\textbf{Proof of claim.}
} {
    \qed (of claim)
}

\title{Algebraic geometry 1\\Exercise sheet \sheet}
\author{Solutions by: Eric Rudolph and David Čadež}

\date{\today}


\begin{document}

\maketitle{}

\begin{exercise}{3}
    \begin{enumerate}
        \item Since $X$ is smooth, all local rings are UFD's. So in this exercise we will show that locally factorial is not affine-local
        \item We can write every $a\in A$ as 
        \begin{align*}
            a=c_o+c_1a_1(x)+c_2ya_2(x),
        \end{align*}
        so as a polynomial of degree 1 over $R$. Since the sum of two integral elements is integral again (\href{https://stacks.math.columbia.edu/tag/00GO}{StacksProject})
        it is enough to show that $c_2ya(x)$ is integral over $R$. This is clearly true since $(c_2ya(x))^2\in R$.
        \item We need to check that zero gets mapped to zero. This follows from
        \begin{align*}
            (-y)^2-x^3+x=y^2-x^3+x
        \end{align*}
        \item Write an element $a(x,y)\in A$ as 
        \begin{align*}
            a(x,y)=c+a_1(x)+ya_2(x).
        \end{align*}
        We compute 
        \begin{align*}
            N(a)=a(x,y)a(x,-y)=(c+a_1(x))^2-(ya_2(x))^2=(c+a_1(x))^2-(x^3-x)a_2(x)^2\in R.
        \end{align*}
        Multiplicativity of $N$ follows from the properties of ring maps.
        \item Take $a\in A^*$. Then $N(a)\in k[x]^*=k^*$ and since $\deg(a\sigma(a))\leq 1$ implies $deg(a)\leq 1$ this implies $a\in k^*$.

        We can check irreducibility using the map $N.$

        Assume that $x=fg$ for $f,g\in A$. Then
        \begin{align*}
            x^2=N(x)=N(fg).
        \end{align*}
        Write $fg=a_1(x)$. Then by the calculations from above,
        \begin{align*}
            a_1(x)^2=x^2 \in R.
        \end{align*}
        Therefore $a_1(x)=\pm x,$ but $x$ is irreducible in $R$, so w.l.o.g. $f\in A^*=k^*$. 

        I guess a similar argument works to show that $y$ is irreducible.

        This shows that $y^2=x(x^2-1)$ are two decompositions into irreducible elements, showing that $A$ is not 
        factorial.
    \end{enumerate}
\end{exercise}

\end{document}
