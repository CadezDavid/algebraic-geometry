\newcommand{\sheet}{9}
\documentclass{article}
\usepackage[english, german]{babel}
\usepackage{amsthm,amssymb,amsmath,mathrsfs,mathtools}
\usepackage[shortlabels]{enumitem}
\usepackage{hyperref}
\usepackage{biblatex}
\usepackage{tikz}
\usepackage{tikz-cd}

% \usepackage[tmargin=1.25in,bmargin=1.25in,lmargin=1.2in,rmargin=1.2in]{geometry}


\newcommand{\C}{\mathbb{C}}
\newcommand{\R}{\mathbb{R}}
\newcommand{\N}{\mathbb{N}}
\newcommand{\Q}{\mathbb{Q}}
\newcommand{\Z}{\mathbb{Z}}
\newcommand{\Proj}{\mathbb{P}}
\newcommand{\Aff}{\mathbb{A}}

\DeclareMathOperator{\id}{id}
\DeclareMathOperator{\im}{im}
\DeclareMathOperator{\GL}{GL}
\DeclareMathOperator{\sgn}{sgn}
\DeclareMathOperator{\Tor}{Tor}
\DeclareMathOperator{\Sym}{Sym}
\DeclareMathOperator{\coker}{coker}
\DeclareMathOperator{\Quot}{Quot}
\DeclareMathOperator{\supp}{supp}
\DeclareMathOperator{\Hom}{Hom}
\DeclareMathOperator{\Spec}{Spec}
\DeclareMathOperator{\MinSpec}{MinSpec}
\DeclareMathOperator{\MaxSpec}{MaxSpec}
\DeclareMathOperator{\diag}{diag}
\DeclareMathOperator{\BL}{BL}
\DeclareMathOperator{\Ouv}{Ouv}
\DeclareMathOperator{\Sh}{Sh}
\DeclareMathOperator{\PSh}{PSh}
\DeclareMathOperator{\Eq}{Eq}
\DeclareMathOperator{\colim}{colim}
\DeclareMathOperator{\Pic}{Pic}
\DeclareMathOperator{\CL}{CL}
\DeclareMathOperator{\eq}{eq}
\DeclareMathOperator{\codim}{codim}

\newenvironment{exercise}[1] {
  \vspace{0.5cm}
  \noindent \textbf{Exercise~{#1}.}
} {
  \vspace{0.5cm}
}
\newenvironment{claim} {
  \par\noindent\textbf{Claim.}
} { }

\newenvironment{proof_claim} {
  \par\noindent\textbf{Proof of claim.}
} {
    \qed (of claim)
}

\title{Algebraic geometry 1\\Exercise sheet \sheet}
\author{Solutions by: Eric Rudolph and David Čadež}

\date{\today}


\begin{document}

\maketitle{}

\begin{exercise}{2}
    \begin{enumerate}
        \item{
            Suppose we have a vector bundle of rank $n$ on $\Proj^1_k$. How do
            we construct a matrix $\in \GL_n(k[T^{\pm 1}])$?

            Take a rank $n$ vector bundle $\mathcal{E}$. Since Picard group of
            $U_0$ and $U_1$ are trivial, we have isomorphisms $\alpha, \beta$.
            So on $\Spec(k[T^{\pm 1}]) \subseteq U_0$ we have an isomorphism
            $\Gamma(U_0 \cap U_1, \mathcal{O}^n_{U_0}) = (k[T^{\pm 1}])^n \cong
            \mathcal{E}(U_0 \cap U_1)$.

            Combining this with an isomorphism $\mathcal{E}(U_0 \cup U_1) \cong
            (k[T^{\pm 1}])^n = \Gamma(U_0 \cap U_1, \mathcal{O}^n_{U_1})$, we
            get an isomorpism $(k[T^{\pm 1}])^n \cong (k[T^{\pm 1}])^n$.

            Let $\mathcal{D}$ be another rank $n$ vector bundle on $\Proj^1_k$,
            and let $\varphi \colon \mathcal{E} \to \mathcal{D}$ be an
            isomorphism between them.
            On $U_0$ and $U_1$ we get induced isomorphisms
            \begin{equation*}
                (k[T])^n = \mathcal{E}(U_0) \to \mathcal{D}(U_0) = (k[T])^n
            \end{equation*}
            and
            \begin{equation*}
                (k[T^{-1}])^n = \mathcal{E}(U_1) \to \mathcal{D}(U_1) = (k[T^{-1}])^n
            \end{equation*}
            \dots

            On the right side, we are given transition maps. 
        
                We have that
                \begin{align*}
                    \alpha_{| U_0\cap U_1}^{-1} \circ \beta_{| U_0\cap U_1}
                \end{align*}
                is invertible, because by assumption $\alpha$ and $\beta$ are
                isomorphisms. To see injectivity, remember that given sheaves on
                a cover and transition maps, we can uniquely (up to isomorphism)
                glue them to get a sheaf on the whole space.

                Well definedness of this map comes from the fact that if two vector bundles 
                $V_1\cong V_2$ are ismorphic, then the transition map is the same.

                It remains to show surjectivity.


        }
    \item
    By left multiplication we get a diagonal matrix we get that 
    \begin{align*}
        A=\begin{bmatrix}
            a_{11}&a_{12}\\
            a_{21}&a_{22}
        \end{bmatrix}
        \in \GL_2(k[x]),
    \end{align*}
    so assume this without loss of generality.

    Next, observe that the determinant of $A$ in this case is in $k[x,x^{-1}]\cap k[x]$, so we can write 
    \begin{align*}
        det(A)=cx^n.
    \end{align*}
    Since $k[x]$ is an euclidean domain, we can now assume that $a_{12}=0$. Using our observation regarding
    the derterminant, we conclude that $a_{11}$ and $a_{22}$ are monomials (with non-negative degree).

    We think that we can also assume w.l.o.g that $\deg(a_{11})\geq \deg(a_{22})$.


    Now we can eliminate $a_{21}$ by adding $k[x]$ mulitples of the second column to the first column, to 
    eliminate all the terms in $a_{21}$ with degree greater $\deg(a_{22})$ and all terms in
    $a_{12}$ with degree smaller $\deg(a_{21})$ by adding $k[x^{-1}]$ multiples of the first row 
    to the second row (to eliminate all terms in $a_{12}$ with degree smaller $\deg(a_{11})$).

    \item[3.]{
        Claim: Every line bundle on $\Proj^1$ can be written as 
        \begin{align*}
            \mathcal{O}_{\Proj^1}(d_n)\oplus \dots \oplus \mathcal{O}_{\Proj ^1}(d_n).
        \end{align*}
        
        Proof of claim:

        By part 1 of this exercise, we can characterize the isomorphism classes of rank n vector bundles by looking at the transition functions. 

        In the second part of this exercise, we showed that (for $n=2$, but actually inductively for all $n$) these 
        transition functions can be writte as $T^d$. The claim now follows from observing that the transition matrix of 
        \begin{align*}
            \mathcal{O}_{\Proj^1}(d_n)\oplus \dots \oplus \mathcal{O}_{\Proj ^1}(d_n)
        \end{align*}
        is given by
        \begin{align*}
            T^{(d_1,\dots, d_n)}.
        \end{align*}
    }
    \end{enumerate}
\end{exercise}

\begin{exercise}{3}
    \begin{enumerate}
        \item By exercise 1 on sheet 3 we get that 
        \begin{align*}
            X \to \Aff_R^{n+1} \times_{\Spec(R)} \Proj^n_R
        \end{align*}
        is a closed immersion, so it remains to check that 
        \begin{align*}
            \mathcal{O}_x \to f_* \mathcal{O}_{\Aff_R^{n+1} \times_{\Spec(R)} \Proj ^n_R}
        \end{align*}
        is a surjection.
    \end{enumerate}
\end{exercise}
\end{document}
